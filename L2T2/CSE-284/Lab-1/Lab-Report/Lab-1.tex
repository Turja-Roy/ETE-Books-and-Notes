\documentclass[12pt]{article}
\usepackage[a4paper, margin=0.75in]{geometry}
\usepackage[document]{ragged2e}
\usepackage{graphicx}
\usepackage{placeins}
\graphicspath{ {./images/} }
% \usepackage[tmargin=2cm,rmargin=1in,lmargin=1in,margin=0.85in,bmargin=2cm,footskip=.2in]{geometry}
\usepackage{enumerate}
\usepackage{framed}
\usepackage{amsmath,amsfonts,amsthm,thmtools,amssymb,mathtools,commath}
\usepackage{tikz}
\usepackage{xcolor}
\usepackage[most]{tcolorbox}


\tcbuselibrary{theorems}
\newtcbtheorem[number within=section]{example}{Example}%
{
    colback=green!5,
    frame hidden,
    detach title,
    before upper = \tcbtitle\par\smallskip,
    coltitle=green!35!black,
    % colframe=green!35!black,
    fonttitle=\bfseries\sffamily
    % description font=\mdseries
}{th}


\title{
    CSE-284: Object Oriented Programming \\
    Experiment 1: Introduction to Class and Objects in OOP
}

\author{
    Turja Roy \\ 
    ID: 2108052 \\ 
    Group: G-2
}
\date{}

\begin{document}
\maketitle

\section*{Objectives:}
\begin{itemize}
    \item Introduction with Classes and Objects in C++. 
    \item To create data member and member functions (Methods) in a class. 
    \item To understand the concept of visibility of data member and member functions (Public and Private access).
\end{itemize}


\FloatBarrier
\section*{Example 1}
Write a C++ program to define a class \texttt{Box} and create objects of this class.

\subsection*{Code}
\lstinputlisting[language=C++]{../Exp-1.cpp}

\subsection*{Output}
\begin{figure}[htpb]
    \centering
    \includegraphics[width=0.8\textwidth]{Exp-1.png}
    \caption{Output of Exp-1.cpp}
\end{figure}


\FloatBarrier
\section*{Example 2}
Write a C++ program to define a class \texttt{Box} with member functions.

\subsection*{Code}
\lstinputlisting[language=C++]{../Exp-2.cpp}

\subsection*{Output}
\begin{figure}[htpb]
    \centering
    \includegraphics[width=0.8\textwidth]{Exp-2.png}
    \caption{Output of Exp-2.cpp}
\end{figure}


\FloatBarrier
\section*{Example 3}
Write a C++ program to define a class \texttt{Box} with member functions.

\subsection*{Code}
\lstinputlisting[language=C++]{../Exp-3.cpp}

\subsection*{Output}
\begin{figure}[htpb]
    \centering
    \includegraphics[width=0.7\textwidth]{Exp-3-Error.png}
    \caption{Exp-3 error log}
\end{figure}
\begin{figure}[htpb]
    \centering
    \includegraphics[width=0.8\textwidth]{Exp-3-Error-fixed.png}
    \caption{Output of Exp-3.cpp After fixing error}
\end{figure}


\FloatBarrier
\section*{Example 4}
Write a C++ program to understand public and private access of class members.

\subsection*{Code}
\lstinputlisting[language=C++]{../Exp-4.cpp}

\subsection*{Output}
\begin{figure}[htpb]
    \centering
    \includegraphics[width=0.7\textwidth]{Exp-4-Error.png}
    \caption{Exp-4 error log}
\end{figure}
\begin{figure}[htpb]
    \centering
    \includegraphics[width=0.8\textwidth]{Exp-4-Error-fixed.png}
    \caption{Output of Exp-4.cpp After fixing error}
\end{figure}


\FloatBarrier
\section*{Lab Task}
Write a C++ program to calculate the CGPA of a student using class and objects.

\subsection*{Code}
\lstinputlisting[language=C++]{../Lab-Test.cpp}

\subsection*{Output}
\begin{figure}[htpb]
    \centering
    \includegraphics[width=0.8\textwidth]{Lab-Test.png}
    \caption{Output of CGPA Calculator}
\end{figure}


\FloatBarrier
\section*{Practice Exercise 1}
Write a class having two private variables and one member function which will return the area and parimeter of the rectangle. 

\subsection*{Code}
\lstinputlisting[language=C++]{../Prac-1.cpp}

\subsection*{Output}
\begin{figure}[htpb]
    \centering
    \includegraphics[width=0.8\textwidth]{Prac-1.png}
    \caption{Output of Prac-1.cpp}
\end{figure}


\FloatBarrier
\section*{Practice Exercise 2}
Write a C++ program to define a class \texttt{batsman} with the following specifications:
\begin{itemize}
    \item Private members:
    \begin{itemize}
        \item \texttt{batsman\_code}: 4 digits code number
        \item \texttt{batsman\_name} 20 characters (string)
        \item \texttt{total\_innings, notout\_innings, total\_runs}: integer type
        \item \texttt{calcavg()}: Function to compute batavg
    \end{itemize}
    \item Public members:
    \begin{itemize}
        \item \texttt{readdata()}: Function to accept value from batsman code, batsman name, to- tal innings, notout innings, total runs and invoke the function calcavg().
        \item \texttt{displaydata()}: Function to display the data members on the screen.
    \end{itemize}
\end{itemize}

\subsection*{Code}
\lstinputlisting[language=C++]{../Prac-2.cpp}

\subsection*{Output}
\begin{figure}[htpb]
    \centering
    \includegraphics[width=0.8\textwidth]{Prac-2.png}
    \caption{Output of Prac-2.cpp}
\end{figure}


\newpage
\section*{Discussion}
\begin{itemize}
    \item In this lab, basic concepts of class and objects in C++ were discussed.
    \item The visibility of data members and member functions were discussed.
    \item When a private member is accessed from outside the class, it generates an error. The errors can be fixed by making the member public, or by using public member functions.
\end{itemize}

\end{document}
