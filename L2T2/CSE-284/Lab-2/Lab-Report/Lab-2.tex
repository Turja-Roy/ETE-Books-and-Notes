\documentclass[12pt]{article}
\usepackage[a4paper, margin=0.75in]{geometry}
\usepackage[document]{ragged2e}
\usepackage{graphicx}
\usepackage{placeins}
\graphicspath{ {./images/} }
% \usepackage[tmargin=2cm,rmargin=1in,lmargin=1in,margin=0.85in,bmargin=2cm,footskip=.2in]{geometry}
\usepackage{enumerate}
\usepackage{framed}
\usepackage{amsmath,amsfonts,amsthm,thmtools,amssymb,mathtools,commath}
\usepackage{tikz}
\usepackage{xcolor}
\usepackage[most]{tcolorbox}


\tcbuselibrary{theorems}
\newtcbtheorem[number within=section]{example}{Example}%
{
    colback=green!5,
    frame hidden,
    detach title,
    before upper = \tcbtitle\par\smallskip,
    coltitle=green!35!black,
    % colframe=green!35!black,
    fonttitle=\bfseries\sffamily
    % description font=\mdseries
}{th}


\title{
    CSE-284: Object Oriented Programming \\
    Experiment 2: Constructor and Destruction in OOP
}

\author{
    Turja Roy \\ 
    ID: 2108052 \\ 
    Group: G-2
}
\date{}

\begin{document}
\maketitle

\section*{Objectives:}
\begin{itemize}
    \item Introduction with Constructor class in C++.
    \item To define different types of constructors.
    \item To learn Constructor and Destructor in C++.
\end{itemize}


\FloatBarrier
\section*{Example 1}
Write a C++ program to demonstrate the use of the default constructor.

\subsection*{Code}
\lstinputlisting[language=C++]{../Exp-1.cpp}

\subsection*{Output}
\vspace{-1em}
\begin{figure}[htpb]
    \centering
    \includegraphics[width=0.7\textwidth]{Exp-1.png}
    \caption{Output of Exp-1.cpp}
\end{figure}


\FloatBarrier
\section*{Example 2}
Write a C++ program to demonstrate the use of Parameterized Constructor.

\subsection*{Code}
\lstinputlisting[language=C++]{../Exp-2.cpp}

\subsection*{Output}
\begin{figure}[htpb]
    \centering
    \includegraphics[width=0.8\textwidth]{Exp-2.png}
    \caption{Output of Exp-2.cpp}
\end{figure}


\FloatBarrier
\section*{Example 3}
Write a C++ program to demonstrate the use of Copy Constructor.

\subsection*{Code}
\lstinputlisting[language=C++]{../Exp-3.cpp}

\subsection*{Output}
\begin{figure}[htpb]
    \centering
    \includegraphics[width=0.8\textwidth]{Exp-3.png}
    \caption{Output of Exp-3.cpp}
\end{figure}


\FloatBarrier
\section*{Example 4}
Write a C++ program to understand the Destructor Class in C++.

\subsection*{Code}
\lstinputlisting[language=C++]{../Exp-4.cpp}

\subsection*{Output}
\begin{figure}[htpb]
    \centering
    \includegraphics[width=0.7\textwidth]{Exp-4.png}
    \caption{Output of Exp-4.cpp}
\end{figure}


\FloatBarrier
\section*{Lab Task}
Write a C++ program with a class named \texttt{Student} that contains three variables \texttt{a, b, c}. If a constructor is not used, the variables will be initialized with values \texttt{1, 2, 3} respectively. If a constructor is used, the variables will be initialized with the values passed as arguments.

\subsection*{Code}
\lstinputlisting[language=C++]{../Lab-Test.cpp}


\FloatBarrier
\section*{Practice 1}
Suppose you have a Savings Account with an initial amount of 500 and you have to add some more amount to it. Create a class 'AddMoney' with a data member named 'amount' with an initial value of 500. Now make two constructors of this cclass as follows:
\begin{itemize}
    \item without any parameter -- no amount will be added to the Savings Account. 
    \item having a parameter which is the amount that will be added to the Savings Account.
\end{itemize}
Create an object of the 'AddMoney' class and display the final amount in the Savings Account.

\subsection*{Code}
\lstinputlisting[language=C++]{../Prac-1.cpp}

\subsection*{Output}
\begin{figure}[htpb]
    \centering
    \includegraphics[width=0.7\textwidth]{Prac-1.png}
    \caption{Output of Prac-1.cpp}
\end{figure}


\FloatBarrier
\section*{Practice 2}
Write a C++ program to define a class 'Car' with the following specifications:
\begin{itemize}
    \item Private members:
        \begin{itemize}
            \item \texttt{car\_name}, \texttt{model\_name}, \texttt{fuel\_type}: string type 
            \item \texttt{mileage}: float type 
            \item \texttt{price}: float type
        \end{itemize}
    \item Public members:
        \begin{itemize}
            \item \texttt{displaydata()}: Function to display the data members on the screen.
        \end{itemize}
\end{itemize}
Use Constructor (both Default and parameterized) and Destructor. When no parameter is passed, the default Constructor will be called with the message "Default Constructor has been called."

\subsection*{Code}
\lstinputlisting[language=C++]{../Prac-2.cpp}

\subsection*{Output}
\begin{figure}[htpb]
    \centering
    \includegraphics[width=0.8\textwidth]{Prac-2.png}
    \caption{Output of Prac-2.cpp}
\end{figure}


\newpage
\section*{Discussion}
\begin{itemize}
    \item In this lab, constructors and destructors were discussed.
    \item Default and parameterized constructors were implemented. 
    \item In experiment 3, the copy constructor was used to copy the values of one object to another object. 
    \item While using copy constructor, the addresses were checked. It can be noticed that the addresses of the two objects were different even though the values were the same. 
    \item The destructor is always called at the end of the program. It is used to free the memory allocated to the object. 
    \item When calling the default constructor, if no value is assigned for the variables, garbage values are assigned there.
\end{itemize}

\end{document}
