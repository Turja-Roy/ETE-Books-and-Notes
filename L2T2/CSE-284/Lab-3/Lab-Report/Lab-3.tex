\documentclass[12pt]{article}
\usepackage[a4paper, margin=0.75in]{geometry}
\usepackage[document]{ragged2e}
\usepackage{graphicx}
\usepackage{placeins}
\graphicspath{ {./images/} }
% \usepackage[tmargin=2cm,rmargin=1in,lmargin=1in,margin=0.85in,bmargin=2cm,footskip=.2in]{geometry}
\usepackage{enumerate}
\usepackage{framed}
\usepackage{amsmath,amsfonts,amsthm,thmtools,amssymb,mathtools,commath}
\usepackage{tikz}
\usepackage{xcolor}
\usepackage[most]{tcolorbox}


\tcbuselibrary{theorems}
\newtcbtheorem[number within=section]{example}{Example}%
{
    colback=green!5,
    frame hidden,
    detach title,
    before upper = \tcbtitle\par\smallskip,
    coltitle=green!35!black,
    % colframe=green!35!black,
    fonttitle=\bfseries\sffamily
    % description font=\mdseries
}{th}


\title{
    CSE-284: Object Oriented Programming \\
    Experiment 3: Static Data Member, and Function Overloading in C++
}

\author{
    Turja Roy \\ 
    ID: 2108052 \\ 
    Group: G-2
}
\date{}

\begin{document}
\maketitle

\section*{Objectives:}
\begin{itemize}
    \item Introduction with the Static Data Member and Member function.
    \item To understand the concept of function overloading in C++.
\end{itemize}


\FloatBarrier
\section*{Example 1}
A C++ program to demonstrate the use of static data member.

\subsection*{Code}
\lstinputlisting[language=C++]{../Exp-1.cpp}

\subsection*{Output}
\vspace{-1em}
\begin{figure}[htpb]
    \centering
    \includegraphics[width=0.7\textwidth]{Exp-1.png}
    \caption{Output of Exp-1.cpp}
\end{figure}


\FloatBarrier
\section*{Example 2}
A C++ program to demonstrate the use of static member function.

\subsection*{Code}
\lstinputlisting[language=C++]{../Exp-2.cpp}

\subsection*{Output}
\begin{figure}[htpb]
    \centering
    \includegraphics[width=0.8\textwidth]{Exp-2.png}
    \caption{Output of Exp-2.cpp}
\end{figure}


\FloatBarrier
\section*{Example 3}
A program to understand the Function Overloading in C++.

\subsection*{Code}
\lstinputlisting[language=C++]{../Exp-3.cpp}

\subsection*{Output}
\begin{figure}[htpb]
    \centering
    \includegraphics[width=0.8\textwidth]{Exp-3.png}
    \caption{Output of Exp-3.cpp}
\end{figure}


\FloatBarrier
\section*{Lab Task}
Write a C++ program with a class \texttt{Student} that contains two variables \texttt{a, b}, and a static variable \texttt{objCount} to keep track of the number of objects created starting from 100. Print the values of \texttt{a, b, objCount} for each object created.

\subsection*{Code}
\lstinputlisting[language=C++]{../Lab-Test.cpp}

\subsection*{Output}
\begin{figure}[htpb]
    \centering
    \includegraphics[width=0.8\textwidth]{Lab-Test.png}
    \caption{Output of Lab Task}
\end{figure}


\FloatBarrier
\section*{Practice 1}
Write a C++ program to define a class \texttt{Batsman} with the following specifications:
\begin{itemize}
    \item \texttt{batsman\_ID}: 6 digits roll number 
    \item \texttt{static member count}: To keep track on number of object 
    \item \texttt{static function getcount()}: return the value of count 
    \item \texttt{function getname()}: To take batsman name as input 
    \item \texttt{showname()}: To show batsman name
\end{itemize}
Access all the data members and member functions using the objects of class Batsman.

\subsection*{Code}
\lstinputlisting[language=C++]{../Prac-1.cpp}

\subsection*{Output}
\begin{figure}[htpb]
    \centering
    \includegraphics[width=0.7\textwidth]{Prac-1.png}
    \caption{Output of Prac-1.cpp}
\end{figure}


\newpage
\section*{Practice 2}
Write a C++ Program to calculate the area of different geometric shapes such as Circle, Triangle, and Rectangle. Use function overloading. \\ 
\underline{\textbf{Class Name: Shape}}

\subsection*{Code}
\lstinputlisting[language=C++]{../Prac-2.cpp}

\subsection*{Output}
\begin{figure}[htpb]
    \centering
    \includegraphics[width=0.8\textwidth]{Prac-2.png}
    \caption{Output of Prac-2.cpp}
\end{figure}


\FloatBarrier
\section*{Discussion}
\begin{itemize}
    \item In this lab, Static Data Member and Static Member Functions were discussed. 
    \item The concept of Function Overloading was discussed too. 
    \item Function overloading is used to define multiple functions with the same name but with different parameters. 
    \item In the Practice-2 task, the constructor function was overloaded to determine which shape it was. 
    \item The area of the shapes was calculated using the overloaded function \texttt{calculateArea}.
\end{itemize}

\end{document}
