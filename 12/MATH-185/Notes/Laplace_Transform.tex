\documentclass[12pt]{article}
\usepackage[a4paper, margin=0.75in]{geometry}
\usepackage[document]{ragged2e}
\usepackage{graphicx}
\graphicspath{ {./images/} }
% \usepackage[tmargin=2cm,rmargin=1in,lmargin=1in,margin=0.85in,bmargin=2cm,footskip=.2in]{geometry}
\usepackage{enumerate}
\usepackage{framed}
\usepackage{amsmath,amsfonts,amsthm,thmtools,amssymb,mathtools,commath}
\usepackage{tikz}
\usepackage{xcolor}
\usepackage[most]{tcolorbox}


\tcbuselibrary{theorems}
\newtcbtheorem[number within=section]{example}{Example}%
{
    colback=green!5,
    frame hidden,
    detach title,
    before upper = \tcbtitle\par\smallskip,
    coltitle=green!35!black,
    % colframe=green!35!black,
    fonttitle=\bfseries\sffamily
    % description font=\mdseries
}{th}

\usepackage[scr]{rsfso}
\usepackage{multicol}
\renewcommand{\arraystretch}{2.5}

\newcommand{\Lap}{\mathscr{L}}

\title{
    \textbf{Laplace Transform}
}

\author{
    Turja Roy\\
    ID: 2108052
}
\date{}

\begin{document}
\maketitle
\tableofcontents
\newpage

%%%%%%%%%%%%%%%%%%%%%%%%%%%%%%%%%%%%%%%%%%%%%%%%%
%  Definition, Existence, and Basic Properties  %
%%%%%%%%%%%%%%%%%%%%%%%%%%%%%%%%%%%%%%%%%%%%%%%%%

\section{Definition, Existence, and Basic Properties of the Laplace Transform}

%%%%%%%%%%%%%%%%%%%%%%%%%%%%%%
%  Definition and Existence  %
%%%%%%%%%%%%%%%%%%%%%%%%%%%%%%

\subsection{Definition and Existence}

\begin{definition}{Laplace Transform}{}
    Let $F$ be a real-valued function of the real variable $t$, defined for $t>0$. Let $s$ be a variable that we shall assume to be real, and consider the function $f$ defined by
    \begin{equation} \label{eq1}
        f(s) = \int_{0}^{\infty} {e^{-st} F(t)} \: d{t} 
    \end{equation}
    for all values of $s$ for which this integral exists. The function $f$ defined by the integral \eqref{eq1} is called the Laplace Transform of the function $F$. We shall denote the Laplace transform of $F$ by $\Lap\{F(t)\}$.\\
    Thus the Laplace transform of a function $f$ is given by
    \begin{equation} \label{eq2}
        \Lap\{ F(t) \} = f(s) = \int_{0}^{\infty} {e^{-st} F(t)} \: d{t}  = \lim_{R \to \infty} \int_{0}^{R} {e^{-st} F(t)} \: d{t} 
    \end{equation}
\end{definition}

Some ways to write Laplace transforms:\\
\[ \Lap{F(t)} = f(s) = \int_{0}^{\infty} { e^{-st} F(t) } \: d{t} \]
\[ \Lap{G(t)} = g(s) \]
\[ \Lap{u(t)} = \tilde{u}(s) \]

\vspace{20pt}
\begin{table}[htpb]
    \centering

    \begin{tabular}{c | c}
        \hline
        $F(t)$ &  $\Lap\{ F(t) \} = f(s)$ \\
        \hline\hline
        $1$ & $\dfrac{1}{s}$ \\\hline
        $t$ & $\dfrac{1}{s^2}$ \\\hline
        $e^{at}$ & $\dfrac{1}{s-a}$ \\\hline
        $\sin{at}$ & $\dfrac{a}{s^2+a^2}$ \\\hline
        $\cos{at}$ & $\dfrac{s}{s^2+a^2}$ \\
        \hline
    \end{tabular}
    \begin{tabular}{c | c}
        \hline
        $F(t)$ &  $\Lap\{ F(t) \} = f(s)$ \\
        \hline\hline
        $n$ & $\dfrac{n}{s}$ \\\hline
        $t^n$ &  $\dfrac{n!}{s^{n+1}}$ \\\hline
        $e^{-at}$ & $\dfrac{1}{s+a}$ \\\hline
        $\sinh{at}$ & $\dfrac{a}{s^2-a^2}$ \\\hline
        $\cosh{at}$ &  $\dfrac{s}{s^2-a^2}$ \\
        \hline
    \end{tabular}

    \caption{Functions and their Laplace Transform}
    \label{LapTable}
\end{table}

\newpage
\begin{multicols}{2}
    [
    \underline{\textbf{Proofs :}}\\
    ]
    Let $F(t) = n$, for $t>0$\\
    Then\\
    \begin{align*}
        \Lap \{ n \} &= \int_{0}^{\infty} { e^{-st} \cdot n } \: d{t} \\
                     &= \eval{ n\frac{-e^{st}}{s} }_{0}^{\infty} \\
                     &= \frac{n}{s}\qed
    \end{align*}\\
    \columnbreak

    Let $F(t) = t$, for $t>0$\\
    Then\\
    \begin{align*}
        \Lap \{ t \} &= \int_{0}^{\infty} { e^{-st} \cdot t } \: d{t} \\
                     &= -t \frac{e^{-st}}{s} + \int_{0}^{\infty} {\frac{e^{-st}}{s}} \: d{t} \\
                     &= -e^{-st}\frac{t}{s}\bigg|_{0}^{\infty} - e^{-st}\frac{1}{s^2}\bigg|_{0}^{\infty} \\
                     &= \frac{1}{s^2}\qed
    \end{align*}
\end{multicols}

\vspace{20pt}
Let $F(t) = t^{n}$, for $t>0$ \\
Then\\
\begin{align*}
    \Lap \{ t^{n} \} &= \int_{0}^{\infty} {e^{-s t}t^n} \: d{t} \\
                     &= -t^n \frac{e^{s t}}{s} + \int_{0}^{\infty} { nt^{n-1} \frac{e^{-s t}}{s} } \: d{t} \\
                     &= \eval{ -nt^{n-1} \frac{e^{-s t}}{s^2} }_{0}^{\infty} + \int_{0}^{\infty} { n(n-1)t^{n-2} \frac{e^{-s t}}{s^2} } \: d{t} \\
                     &= \eval{ -n(n-1)t^{n-2} \left( \frac{e^{-s t}}{s^3} \right) }_{0}^{\infty} + \int_{0}^{\infty} { n(n-1)(n-2)t^{n-3} \frac{e^{-s t}}{s^3} } \: d{t}  \\
                     &= \cdots \\
                     &= \eval{n!t^{n-n}\frac{e^{-s t}}{s^{n+1}}}_{0}^{\infty} + \int_{0}^{\infty} { n(n-1)\cdots(n-n) \frac{e^{-s t}}{s^{n+1}} } \: d{t} \\
                     &= \frac{n!}{s^{n+1}}\qed
\end{align*}\\~\\

\begin{multicols}{2}
    Let $F(t) = e^{at}$, for $t>0$\\
    Then\\
    \begin{align*}
        \Lap \{ e^{at} \} &= \int_{0}^{\infty} { e^{-s t} e^{at} } \: d{t} \\
                          &= \int_{0}^{\infty} { e^{(a-s)t} } \: d{t} \\
                          &= \eval{ \frac{e^{(a-s)t}}{a-s} }_{0}^{\infty} \\
                          &= \frac{1}{s-a}\qed
    \end{align*}
    \columnbreak
    
    Let $F(t)=e^{-at}$, for $t>0$ \\
    Then\\
    \begin{align*}
        \Lap \{ e^{-at} \} &= \int_{0}^{\infty} {e^{-s t} e^{-at}} \: d{t} \\
        &= \int_{0}^{\infty} {e^{-(a+s)t}} \: d{x} \\
        &= \eval{ \frac{e^{-(a+s)t}}{s+a} }_{0}^{\infty} \\
        &= \frac{1}{s+a}\qed
    \end{align*}
\end{multicols}

\newpage
\begin{multicols}{2}
    Let $F(t) = \sin{at}$, for $t>0$ \\
    Then\\
    \begin{align*}
        \Lap \{ \sin{at} \} &= \int_{0}^{\infty} { e^{-s t \sin{at}} } \: d{t} \\
        &= - \eval{ \frac{e^{-s t}}{s^2+a^2} \left( s\sin{at} + a\cos{at} \right) }_{0}^{\infty} \\
        &= \frac{a}{s^2+a^2}\qed
    \end{align*}
    \columnbreak
    
    Let $F(t) = \cos{at}$, for $t>0$ \\
    Then\\
    \begin{align*}
        \Lap \{ \cos{at} \} &= \int_{0}^{\infty} { e^{-s t}\cos{at} } \: d{t} \\
        &= \eval{ \frac{e^{-s t}}{s^2+a^2} \left( -s\cos{at} + a\sin{at} \right) }_{0}^{\infty} \\
        &= \frac{s}{s^2+a^2}\qed
    \end{align*}
\end{multicols}


\vspace{30pt}
\begin{definition}{Piecewise continuous or Sectionally continuous Function}{}
    A function $f$ is said to be piecewise continuous on a finite interval $a \le t \le b$ if this interval can be divided into a finite number of subintervals such that
    \begin{enumerate}
        \item $f$ is continuous in the interior of each of these subintervals, and
        \item $f(t)$ approaches finite limits as $t$ approaches either endpoint of each of the subintervals from its interior
    \end{enumerate}
\end{definition}

\begin{example}{Consider the function $f$ defined by\\
    \[ f(t) = \begin{cases}
        -1, &\text{ if }0<t<2\\
        1, & \text{ if }t>2
    \end{cases} \]}{}
    $f$ is piecewise continuous on every finite interval $0 \le t \le b$, for every positive number $b$. At $t=2$, we have
    \[ f(2-) = \lim_{t \to 2-} f(t) = -1 \]
    \[ f(2+) = \lim_{t \to 2+} f(t) = +1 \]
\end{example}


















\end{document}
