\documentclass[12pt]{article}
\usepackage[a4paper, margin=0.75in]{geometry}
\usepackage[document]{ragged2e}
\usepackage{graphicx}
\graphicspath{ {./images/} }
% \usepackage[tmargin=2cm,rmargin=1in,lmargin=1in,margin=0.85in,bmargin=2cm,footskip=.2in]{geometry}
\usepackage{enumerate}
\usepackage{framed}
\usepackage{amsmath,amsfonts,amsthm,thmtools,amssymb,mathtools,commath}
\usepackage{tikz}
\usepackage{xcolor}
\usepackage[most]{tcolorbox}


\tcbuselibrary{theorems}
\newtcbtheorem[number within=section]{example}{Example}%
{
    colback=green!5,
    frame hidden,
    detach title,
    before upper = \tcbtitle\par\smallskip,
    coltitle=green!35!black,
    % colframe=green!35!black,
    fonttitle=\bfseries\sffamily
    % description font=\mdseries
}{th}

\usepackage[scr]{rsfso}
\renewcommand{\arraystretch}{2.5}

\newcommand{\Lap}{\mathscr{L}}

\title{
    \textbf{Laplace Transform}
}

\author{
    Turja Roy\\
    ID: 2108052
}
\date{}

\begin{document}
\maketitle

%%%%%%%%%%%%%%%%%%%%%%%%%%%%%%%%%%%%%%%%%%%%%%%%%
%  Definition, Existence, and Basic Properties  %
%%%%%%%%%%%%%%%%%%%%%%%%%%%%%%%%%%%%%%%%%%%%%%%%%

\section{Definition, Existence, and Basic Properties of the Laplace Transform}

%%%%%%%%%%%%%%%%%%%%%%%%%%%%%%
%  Definition and Existence  %
%%%%%%%%%%%%%%%%%%%%%%%%%%%%%%

\subsection{Definition and Existence}

\begin{definition}{Laplace Transform}{}
    Let $F$ be a real-valued function of the real variable $t$, defined for $t>0$. Let $s$ be a variable that we shall assume to be real, and consider the function $f$ defined by
    \begin{equation} \label{eq1}
        f(s) = \int_{0}^{\infty} {e^{-st} F(t)} \: d{t} 
    \end{equation}
    for all values of $s$ for which this integral exists. The function $f$ defined by the integral \eqref{eq1} is called the Laplace Transform of the function $F$. We shall denote the Laplace transform of $F$ by $\Lap\{F(t)\}$.\\
    Thus the Laplace transform of a function $f$ is given by
    \begin{equation} \label{eq2}
        \Lap\{ F(t) \} = f(s) = \int_{0}^{\infty} {e^{-st} F(t)} \: d{t}  = \lim_{R \to \infty} \int_{0}^{R} {e^{-st} F(t)} \: d{t} 
    \end{equation}
\end{definition}

Some ways to write Laplace transforms:\\
\[ \Lap{F(t)} = f(s) = \int_{0}^{\infty} { e^{-st} F(t) } \: d{t} \]
\[ \Lap{G(t)} = g(s) \]
\[ \Lap{u(t)} = \tilde{u}(s) \]

\begin{table}[htpb]
    \centering
    \caption{Functions and their Laplace Transform}
    \label{LapTable}

    \begin{tabular}{c | c}
        $F(t)$ &  $\Lap\{ F(t) \} = f(s)$ \\
        \hline\hline
        $1$ & $\dfrac{1}{s}$ \\\hline
        $t$ & $\dfrac{1}{s^2}$ \\\hline
        $e^{at}$ & $\dfrac{1}{s-a}$ \\\hline
        $\sin{at}$ & $\dfrac{a}{s^2+a^2}$ \\\hline
        $\cos{at}$ & $\dfrac{s}{s^2+a^2}$ \\
        \hline
    \end{tabular} \mid
    \begin{tabular}{c | c}
        $F(t)$ &  $\Lap\{ F(t) \} = f(s)$ \\
        \hline\hline
        $n$ & $\dfrac{n}{s}$ \\\hline
        $t^n$ &  $\dfrac{n!}{s^{n+1}}$ \\\hline
        $e^{-at}$ & $\dfrac{1}{s+a}$ \\\hline
        $\sinh{at}$ & $\dfrac{a}{s^2-a^2}$ \\\hline
        $\cosh{at}$ &  $\dfrac{s}{s^2-a^2}$ \\
        \hline
    \end{tabular}
\end{table}














\end{document}
