%%%%%%%%%%%%%%%%%%%%%%%%%
%  Complex Integration  %
%%%%%%%%%%%%%%%%%%%%%%%%%
\section{Complex Integration}
\subsection{Definitions}
\begin{definition}{Simply Connected Region}{}
    A connected region is said to be a \textbf{Simply Connected} region if all the interior points of a closed curve $C$ drawn in the region $D$ are the points of the region $D$.
\end{definition}

\begin{definition}{Multi-Connected Region}{}
    A \textbf{Multi-connected} region is bounded by more than one curve. A multi-connected region can be divided into simply connected regions.
\end{definition}

\vspace{20pt}\rule{3in}{1pt}


\subsection{Complex Line Integrals}
\begin{definition}{Complex Line Integral}{}
    The \textbf{Complex Line Integral} of a function $f(z)$ along a curve $C$ is defined as
    \begin{equation}
        \oint_C f(z) \, dz = \lim_{n \to \infty} \sum_{k=1}^n f(z_k^*) \Delta z_k
    \end{equation}
    where $z_k^*$ is a point on the curve $C$ and $\Delta z_k$ is the length of the curve $C$.
\end{definition}
\vspace{20pt}

If $z = x + iy$ and $f(z) = u(x,y) + iv(x,y)$, then \[
    d{z} = dx + i \, dy
\] and \[
    f(z) \, dz = (u \, dx - v \, dy) + i(u \, dy + v \, dx)
\] Hence, the complex line integral can be written as
\begin{equation}
    \boxed{ \oint_C f(z) \, dz = \int_C (u \, dx - v \, dy) + i \int_C (u \, dy + v \, dx) }
\end{equation}

\vspace{20pt}\rule{3in}{1pt}


\subsection{Cauchy's Integral Theorem}
\begin{theorem}{Cauchy's Integral Theorem}{}
    \\If $f(z)$ is analytic and its derivative $f'(z)$ is continuous at all points inside and on a simple closed curve $C$, then
    \begin{equation}
        \oint_C f(z) \, dz = 0
    \end{equation}
\end{theorem}

\underline{\textbf{Proof:}} \\
Let the region enclosed by the curve $C$ be $D$, and let
\begin{align*}
    f(z) &= u(x,y) + iv(x,y) \\
    z &= x + iy \\
    dz &= dx + i \, dy
\end{align*}

Now,
\begin{align*}
    \oint_C {f(z) \, dz} &= \oint_C {\left( u+iv \right)\left( dx + idy \right)}  \\
    &= \oint_C {(u \, dx - v \, dy) + i \oint_C (u \, dy + v \, dx)} \\
    &= \iint_R \left( -\pdv{v}{x} - \pdv{u}{y} \right) \, dx \, dy + i \iint_R \left( \pdv{u}{x} - \pdv{v}{y} \right) \, dx \, dy
\end{align*}
\hfill[By Green's Theorem]

Since $f(z)$ is analytic, the Cauchy-Riemann equations hold, i.e.
\begin{equation*}
    \pdv{u}{x} = \pdv{v}{y} 
    \quad \text{and} \quad
    \pdv{u}{y} = -\pdv{v}{x}
\end{equation*}
Thus, we get \[
    \oint_C f(z) \, dz = 0 \qed
\] 

\vspace{20pt}\rule{3in}{1pt}


\subsection{Cauchy's Integral Formula}
\begin{theorem}{Cauchy's Integral Formula}{}
    \\If $f(z)$ is analytic inside and on a simple closed curve $C$, and if $a$ is a point inside the curve $C$, then
    \begin{equation}
        f(a) = \frac{1}{2\pi i} \oint_C \frac{f(z)}{z-a} \, dz
    \end{equation}
\end{theorem}

\underline{\textbf{Proof:}} \\
Let $f(z)$ be analytic inside and on a simple closed curve $C$, and let $a$ be a point inside the curve $C$. Then, by Cauchy's Integral Theorem, we have
\begin{equation*}
    \oint_C f(z) \, dz = 0
\end{equation*}
Now, consider the function
\begin{equation*}
    g(z) = \frac{f(z)}{z-a}
\end{equation*}
This function is analytic inside and on the curve $C$, except at the point $z=a$. Thus, by Cauchy's Integral Theorem, we have
\begin{equation*}
    \oint_C g(z) \, dz = 0
\end{equation*}
Now, we have
\begin{align*}
    \oint_C g(z) \, dz &= \oint_C \frac{f(z)}{z-a} \, dz \\
    &= \oint_C \frac{f(z)}{z-a} \, dz - \oint_C \frac{f(a)}{z-a} \, dz + \oint_C \frac{f(a)}{z-a} \, dz \\
    &= \oint_C \frac{f(z) - f(a)}{z-a} \, dz + f(a) \oint_C \frac{1}{z-a} \, dz
\end{align*}
For any point on the curve $C_1$, we have
\[
    z - a = re^{i\theta}
    \quad \text{and} \quad
    dz = ire^{i\theta} \, d\theta
\] 
\begin{align*}
    \oint_{C_1} \frac{f(z) - f(a)}{z-a} \, dz &= \int_{C_1}^{} {\frac{f(z) - f(a)}{z-a}} \; dz \\
    &= \int_{0}^{2\pi} {\frac{f(re^{i\theta}) - f(a)}{re^{i\theta}}} \; ire^{i\theta} \, d{\theta} \\
    &= \int_{0}^{2\pi} i \left[ f(re^{i\theta}) - f(a) \right] \, d{\theta} \\
    &= 0 \\
    \int_{C_1}^{} {\frac{1}{z-a}} \; d{z} &= \int_{0}^{2\pi} {\frac{ire^{i\theta}}{re^{i\theta}}} \; d{\theta} \\
    &= \int_{0}^{2\pi} i \, d{\theta} \\
    &= 2\pi i
\end{align*}

Thus, we have
\begin{align*}
    \oint_C g(z) \, dz &= \oint_C \frac{f(z) - f(a)}{z-a} \, dz + f(a) \oint_C \frac{1}{z-a} \, dz \\
    &= 0 + f(a) \cdot 2\pi i \\
    &= 2\pi i f(a)
\end{align*}

Hence, we have
\begin{align*}
    \oint_C g(z) \, dz &= 2\pi i f(a) \\
    \oint_C \frac{f(z)}{z-a} \, dz &= 2\pi i f(a) \\
    f(a) &= \frac{1}{2\pi i} \oint_C \frac{f(z)}{z-a} \, dz \qed
\end{align*}

\begin{theorem}{Cauchy Integral Formula for the Derivative of an Analytic Function}{}
    \\If $f(z)$ is analytic inside and on a simple closed curve $C$, and if $a$ is a point inside the curve $C$, then
    \begin{equation}
        f'(a) = \frac{1}{2\pi i} \oint_C \frac{f(z)}{(z-a)^2} \, dz
    \end{equation}
    \begin{equation}
        f''(a) = \frac{2!}{2\pi i} \oint_C \frac{f(z)}{(z-a)^3} \, dz
    \end{equation}
    \begin{equation}
        f^{(n)}(a) = \frac{n!}{2\pi i} \oint_C \frac{f(z)}{(z-a)^{n+1}} \, dz
    \end{equation}
\end{theorem}

\underline{\textbf{Proof:}} \\
The proof of these formulas can be obtained by differentiating the Cauchy Integral Formula and using the Cauchy Integral Formula for $f(a)$. \\
We know
\begin{equation}
    f(a) = \frac{1}{2\pi i} \oint_C \frac{f(z)}{z-a} \, dz
\end{equation} 
Differentiating both sides with respect to $a$, we get
\begin{equation}
    f'(a) = \frac{1}{2\pi i} \oint_C \frac{f(z)}{(z-a)^2} \, dz
\end{equation}
Differentiating again, we get
\begin{equation}
    f''(a) = \frac{2!}{2\pi i} \oint_C \frac{f(z)}{(z-a)^3} \, dz
\end{equation}
Continuing this process, we get
\begin{equation}
    f^{(n)}(a) = \frac{n!}{2\pi i} \oint_C \frac{f(z)}{(z-a)^{n+1}} \, dz \qed
\end{equation}

\vspace{20pt}\rule{3in}{1pt}


\subsection{Cauchy's Extended Theorem}
\begin{theorem}{Cauchy's Extended Theorem}{}
    \\If $f(x)$ is analytic within and on the boundary of a region bounded by two closed curves $C_1$ and $C_2$, then
    \begin{equation}
        \oint_{C_1} f(z) \, dz = \oint_{C_2} f(z) \, dz
    \end{equation}
\end{theorem}
