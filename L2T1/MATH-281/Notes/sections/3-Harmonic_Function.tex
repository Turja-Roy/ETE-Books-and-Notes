%%%%%%%%%%%%%%%%%%%%%%%
%  Harmonic Function  %
%%%%%%%%%%%%%%%%%%%%%%%

\section{Harmonic Function}

\subsection{Laplace's Equation}
\begin{definition}{Laplace's Equation}{}
    An equation of the form
    \begin{equation}
        \pdv[2]{\phi}{x} + \pdv[2]{\phi}{y} = 0
        \quad \text{or} \quad
        \nabla^2 \phi = 0
    \end{equation}
    is called \textbf{Laplace's equation} (in two dimentions). \\
    Here, $\nabla^2$ is the Laplacian operator.
\end{definition}

\vspace{20pt}\rule{3in}{1pt}

\subsection{Harmonic Function}
\begin{definition}{Harmonic Function}{}
    A function $\phi(x,y)$ is called \textbf{harmonic} if it satisfies Laplace's equation
    \begin{equation}
        \nabla^2 \phi = 0
    \end{equation}
    where $\nabla^2$ is the Laplacian operator.
\end{definition}

\begin{theorem}{}{}
    If $f(z) = u + iv$ is an analytic function, then $u$ and $v$ are both harmonic functions.
\end{theorem}

\underline{\textbf{Proof:}} \\
Since $f(z)$ is analytic, it satisfies the Cauchy-Riemann equations
\begin{align}
    \pdv{u}{x} &= \pdv{v}{y} \\
    \pdv{u}{y} &= -\pdv{v}{x}
\end{align}

Differentiating (3.2.2) w.r.t. $x$ and (3.2.3) w.r.t. $y$, we get
\begin{align}
    \pdv[2]{u}{x} &= \frac{\partial v}{\partial x \partial y} \\
    \pdv[2]{u}{y} &= -\frac{\partial v}{\partial y \partial x}
\end{align}

Adding (3.2.4) and (3.2.5), we get
\begin{equation}
    \pdv[2]{u}{x} + \pdv[2]{u}{y} = 0
\end{equation}

Similarly,
\begin{equation}
    \pdv[2]{v}{x} + \pdv[2]{v}{y} = 0
\end{equation}

Hence, both $u$ and $v$ are harmonic functions. \hfill $\qed$ \\~\\

\begin{definition}{Conjugate Harmonic Function}{}
    Any two functions $\phi$ and $\psi$ such that $f(z) = \phi + i\psi$ is analytic, are called \textbf{Conjugate Harmonic Functions}.
\end{definition}

\vspace{20pt}\rule{3in}{1pt}


\subsection{Velocity Potential}
Consider a two-dimensional flow of an incompressible fluid. The velocity of the fluid at a point $(x,y)$ is given by the vector
\begin{equation}
    \vb v = v_x \vu{i} + v_y \vu{j}
\end{equation}
Here, $v$ is called the stream function. \\~\\

The \textbf{velocity potential} $\phi(x,y)$ is defined as the scalar function such that
\begin{equation}
    \boxed{ \vb V = \nabla \phi = \left( \vu{i} \pdv{}{x} + \vu{j} \pdv{}{y} \right) \phi = \vu{i} \pdv{\phi}{x} + \vu{j} \pdv{\phi}{y} }
\end{equation}

Comparing (3.3.1) and (3.3.2), we get
\begin{equation}
    v_x = \pdv{\phi}{x} \quad \text{and} \quad v_y = \pdv{\phi}{y}
\end{equation}

The scalar function $\phi(x,y)$ gives the velocity components. \\
Since the fluid is incompressible,
\begin{align*}
    \nabla v &= 0 \\
    \left( \vu{i} \pdv{}{x} + \vu{j} \pdv{}{y} \right) \left( \vu{i} v_x + \vu{j} v_y \right) &= 0 \\
    \pdv{v_x}{x} + \pdv{v_y}{y} &= 0
\end{align*}

Putting the values of $v_x$ and $v_y$ from (3.3.3),
\begin{align*}
    \pdv{}{x} \left( \pdv{\phi}{x} \right) + \pdv{}{y} \left( \pdv{\phi}{y} \right) &= 0 \\
    \pdv[2]{\phi}{x} + \pdv[2]{\phi}{y} &= 0
\end{align*}

This is Laplace's equation. Hence, the velocity potential $\phi(x,y)$ is a harmonic function and is a real part of the analytic function
\[
    f(z) = \phi + i\psi
\]

\vspace{20pt}\rule{3in}{1pt}


\subsection{Method for Finding Conjugate Harmonic Function}
\subsubsection{Method 1: Real or Imaginary Part of an Analytic Function is Given}
If $f(z) = u + iv$ and $u$ is known \\

We know that \[
    d{v} = \pdv{v}{x} d{x} + \pdv{v}{y} d{y}
\] 
Using C-R equations,
\begin{align*}
    d{v} &= -\pdv{u}{y} d{x} + \pdv{u}{x} d{y} \\
    v &= -\int \pdv{u}{y} d{x} + \int \pdv{u}{x} d{y} + C
\end{align*}
Since $u$ is known, $v$ can be found using the above method. \\~\\

If $v$ is known, then $u$ can be found using the same method.

\subsubsection{Method 2: Milne's Method/ Milne Thomson Method}
By this method, $f(z)$ is directly constructed without finding $v$. \\
Since \[
    z = x + iy \quad \text{and} \quad \bar{z} = x - iy
\] \[
    x = \frac{z + \bar{z}}{2} \quad \text{and} \quad y = \frac{z - \bar{z}}{2i}
\] 

Thus,
\begin{equation}
    \boxed{ f(z) = u\left( \frac{z + \bar{z}}{2}, \frac{z - \bar{z}}{2i} \right) + iv\left( \frac{z + \bar{z}}{2}, \frac{z - \bar{z}}{2i} \right) }
\end{equation} \\~\\

\underline{\textbf{Case 1: $u$ is given}} \\
Let $f(z) = u + iv$ be an analytic function and $u$ is given. \\
Then, \[
    \pdv{u}{x} = u_1(x,y) \quad \text{and} \quad \pdv{u}{y} = u_2(x,y)
\] 

By Milne's method, we get
\begin{equation}
    f'(z) = u_1(z,0) - iu_2(z,0)
\end{equation} 

Integrating (3.4.1) w.r.t. $z$, we get
\begin{equation}
    f(z) = \int \left[ u_1(z,0) - iu_2(z,0) \right] dz + C_1
\end{equation}

\underline{\textbf{Case 2: $v$ is given}} \\
If $v$ is given, then \[
    \pdv{v}{y} = v_1(x,y) \quad \text{and} \quad \pdv{v}{x} = v_2(x,y)
\] 

By Milne's method, we get
\begin{equation}
    f'(z) = v_1(z,0) + iv_2(z,0)
\end{equation}

Integrating (3.4.3) w.r.t. $z$, we get
\begin{equation}
    f(z) = \int \left[ v_1(z,0) + iv_2(z,0) \right] dz + C_2
\end{equation}

\vspace{20pt}\rule{3in}{1pt}


\subsection{Complex Potential Function}
\begin{definition}{Complex Potential Function}{}
    The analytic function  \[
        W = \phi(x,y) + i\psi(x,y)
    \] is called the \textbf{Complex Potential Function}. \\
    The real part $\phi(x,y)$ represents the velocity potential function, and the imaginary part $\psi(x,y)$ represents the stream function.
\end{definition}

\begin{example}{
        If $W = \phi + i\psi$ represents the complex potential for an electric field, and $\psi = 3x^2y-y^3$, then find $\phi$.
    }{}
    Given, \[
        \psi = 3x^2y - y^3
    \] Hence,
    \begin{align*}
        \pdv{\psi}{y} &= 3x^2 - 3y^2 \\
        \pdv{\phi}{x} &= 6xy
    \end{align*}
    
    By Milne's method, we have
    \begin{align*}
        W'(z) &= \psi_1(z,0) + i\psi_2(z,0) \\
        &= 3z^2 + i\cdot 0 \\
        &= 3z^2
    \end{align*}
    Integrating $W'(z)$ w.r.t. $z$, we get
    \begin{align*}
        W(z) &= \int 3z^2 dz + C \\
        \phi + i\psi &= z^3 + c_1 + ic_2 \\
        \phi + i\psi = x^3 - 3xy^2 + c_1 + i\left( 3x^2y - y^3 + c_2 \right)
    \end{align*}
    Comparing real and imaginary parts, we get the required potential function \[
        \boxed{ \phi = x^3 - 3xy^2 + c_1 }
    \] 
\end{example}

\vspace{20pt}\rule{3in}{1pt}
