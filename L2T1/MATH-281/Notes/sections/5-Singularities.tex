%%%%%%%%%%%%%%%%%%%
%  Singularities  %
%%%%%%%%%%%%%%%%%%%

\section{Singularities and Residues}
\subsection{Definitions}
\begin{definition}{Singular Points}{}
    All the points of the z-plane at which an analytic function does not have a unique derivative are called singular points.  \\~\\
    
    For example, the function $f(z) = \frac{1}{z}$ has a singular point at $z = 0$ because the derivative of $f(z)$ at $z = 0$ is not unique.
\end{definition}

\begin{definition}{Poles}{}
    A singular point $z_0$ of a function $f(z)$ is called a pole of order $m$ if the function $f(z)$ can be written as
    \[ f(z) = \frac{g(z)}{(z - z_0)^m} \]
    where $g(z)$ is analytic at $z_0$ and $g(z_0) \neq 0$. \\~\\
    
    The smallest positive integer $m$ for which the above equation holds is called the order of the pole. \\
    Poles of order 1 are called simple poles, poles of order 2 are called double poles, and so on.
\end{definition}

\begin{definition}{Laurent Series}{}
    If $f(z)$ is analytic inside and on a circle $C$ with center at $z=a$ and radius $R$, then $f(z)$ can be expanded in a \textbf{Laurent series} about $z=a$ as
    \begin{equation}
        f(z) = \sum_{n=-\infty}^{\infty} a_n (z-a)^n
    \end{equation}
    where the coefficients $a_n$ are given by
    \[ a_n = \frac{1}{2\pi i} \oint_C \frac{f(z)}{(z-a)^{n+1}} \, dz \]
\end{definition}

\begin{definition}{Residues}{}
    If $f(z)$ has a pole of order $n$ at $z=a$ but is analytic at every other point inside and on a circle $C$ with center at $a$, then the \textbf{Laurent series} about $z=a$ is given by
    \begin{equation}
        f(z) = \sum_{n=-\infty}^{\infty} a_n (z-a)^n
    \end{equation} Or,
    \begin{align*}
        f(z) &= \sum_{n=0}^{\infty} a_n (z-a)^n + \sum_{n=1}^{\infty} a_{-n} (z-a)^{-n} \\
        f(z) &= a_0 + a_1(z-a) + a_2(z-a)^2 + \cdots + \frac{a_{-1}}{z-a} + \frac{a_{-2}}{(z-a)^2} + \cdots
    \end{align*}

    The part of the Laurent series containing the positive powers of $(z-a)$ is called the \textbf{analytic part} of $f(z)$ at $z=a$ and is denoted by $P(f;a)$, and the part containing the negative powers of $(z-a)$ is called the \textbf{principal part} of $f(z)$ at $z=a$ and is denoted by $Q(f;a)$.  \\~\\

    The coefficient $a_{-1}$ is called the \textbf{residue} of $f(z)$ at $z=a$ and is denoted by $\Res(f;a)$.
\end{definition}

\vspace{20pt}\rule{3in}{1pt}

\subsection{Methods of Finding Residues}
\begin{theorem}{Residue at a Simple Pole}{}
    \\If $f(z)$ has a simple pole at $z=a$, then the residue of $f(z)$ at $z=a$ is given by
    \[ \Res(f;a) = \lim_{z \to a} (z-a)f(z) \]
\end{theorem}

\underline{\textbf{Proof:}} \\
Since $f(z)$ has a simple pole at $z=a$, we can write $f(z)$ as
\[ f(z) = \frac{g(z)}{z-a} \]
where $g(z)$ is analytic at $z=a$ and $g(a) \neq 0$. \\
Multiplying both sides by $(z-a)$, we get
\[ (z-a)f(z) = g(z) \]
Taking the limit as $z \to a$ on both sides, we get
\[ \lim_{z \to a} (z-a)f(z) = \lim_{z \to a} g(z) = g(a) \]
Therefore, the residue of $f(z)$ at $z=a$ is given by
\[ \Res(f;a) = \lim_{z \to a} (z-a)f(z) = g(a) \qed\]

\begin{theorem}{Residue at a Pole of Order $m$}{}
    \\If $f(z)$ has a pole of order $m$ at $z=a$, then the residue of $f(z)$ at $z=a$ is given by
    \[ \Res(f;a) = \frac{1}{(m-1)!} \lim_{z \to a} \frac{d^{m-1}}{dz^{m-1}} \left[ (z-a)^m f(z) \right] \]
\end{theorem}

\underline{\textbf{Proof:}} \\
Since $f(z)$ has a pole of order $m$ at $z=a$, we can write $f(z)$ as
\[ f(z) = \frac{g(z)}{(z-a)^m} \]
where $g(z)$ is analytic at $z=a$ and $g(a) \neq 0$. \\
Multiplying both sides by $(z-a)^m$, we get
\[ (z-a)^m f(z) = g(z) \]
Differentiating both sides $m-1$ times, we get
\[ \frac{d^{m-1}}{dz^{m-1}} \left[ (z-a)^m f(z) \right] = \frac{d^{m-1}}{dz^{m-1}} g(z) \]
Taking the limit as $z \to a$ on both sides, we get
\[ \frac{1}{(m-1)!} \lim_{z \to a} \frac{d^{m-1}}{dz^{m-1}} \left[ (z-a)^m f(z) \right] = \frac{1}{(m-1)!} \frac{d^{m-1}}{dz^{m-1}} g(a) \]
Therefore, the residue of $f(z)$ at $z=a$ is given by
\[ \Res(f;a) = \frac{1}{(m-1)!} \lim_{z \to a} \frac{d^{m-1}}{dz^{m-1}} \left[ (z-a)^m f(z) \right] = \frac{1}{(m-1)!} \frac{d^{m-1}}{dz^{m-1}} g(a) \qed \]

\vspace{20pt}\rule{3in}{1pt}

\subsection{Cauchy's Residue Theorem}
\begin{theorem}{Cauchy's Residue Theorem}{}
    If $f(z)$ is analytic inside and on a simple closed curve $C$ except at a finite number of poles $z_1, z_2, \ldots, z_n$ inside $C$, then
    \[ \oint_C f(z) \, dz = 2\pi i \left( \Res(f;z_1) + \Res(f;z_2) + \cdots + \Res(f;z_n) \right) \]
\end{theorem}

\underline{\textbf{Proof:}} \\
Let $C$ be a simple closed curve and $f(z)$ be analytic inside and on $C$ except at a finite number of poles $z_1, z_2, \ldots, z_n$ inside $C$. \\
Let $C_i$ be a small circle centered at $z_i$ and $C$ be the union of $C_i$ and $C$. \\
By Cauchy's Integral Theorem, we have
\[ \oint_C f(z) \, dz = \sum_{i=1}^{n} \oint_{C_i} f(z) \, dz \]
By Cauchy's Integral Formula, we have
\[ \oint_{C_i} f(z) \, dz = 2\pi i \Res(f;z_i) \]
Therefore,
\[ \oint_C f(z) \, dz = 2\pi i \left( \Res(f;z_1) + \Res(f;z_2) + \cdots + \Res(f;z_n) \right) \qed \]

\vspace{20pt}\rule{3in}{1pt}
