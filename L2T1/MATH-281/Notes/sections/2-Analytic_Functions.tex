%%%%%%%%%%%%%%%%%%%%%%%%
%  Analytic Functions  %
%%%%%%%%%%%%%%%%%%%%%%%%

\section{Analytic Functions}
\subsection{Definitions}

\begin{definition}{Complex variable}{}
    A \textbf{complex variable} is a variable that can take on complex values. A complex variable is usually denoted by $z$. \\~\\
    
    If $x$ and $y$ are real variables, then $z = x + iy$ is a complex variable, where $i$ is the imaginary unit.
\end{definition}

\begin{definition}{Complex Function}{}
    A \textbf{complex function} is a function that takes complex variables as input and returns complex values. A complex function is usually denoted by $f(z)$. \\~\\

    If $z = x + iy$ and $w = u + iv$ are complex variables, then $f(z) = u(x, y) + iv(x, y)$ is a complex function, where $u(x, y)$ and $v(x, y)$ are real functions.
\end{definition}

\begin{definition}{Single-valued Function}{}
    A \textbf{single-valued function} is a function that returns a unique value for each input. \\~\\
    
    A complex function $f(z)$ is single-valued if and only if $f(z_1) = f(z_2)$ implies $z_1 = z_2$. In other words, if $z_1 \neq z_2$, then $f(z_1) \neq f(z_2)$.
    \[
        \forall z_1, z_2 \in \mathbb{C} \quad \text{s.t.} \quad z_1 \neq z_2 \quad \text{implies} \quad f(z_1) \neq f(z_2)
    \] 
\end{definition}

\begin{definition}{Multiple-valued Function}{}
    A \textbf{multiple-valued function} is a function that returns multiple values for each input. \\~\\
    
    A complex function $f(z)$ is multiple-valued if and only if $f(z_1) = f(z_2)$ for some $z_1 \neq z_2$.
    \[
        \exists z_1, z_2 \in \mathbb{C} \quad \text{s.t.} \quad z_1 \neq z_2 \quad \text{and} \quad f(z_1) = f(z_2)
    \] 
\end{definition}

\begin{definition}{Derivative}{}
    The \textbf{derivative} of a complex function $f(z)$ is defined as
    \[
        f'(z) = \lim_{\Delta z \to 0} \frac{f(z + \Delta z) - f(z)}{\Delta z}
    \]
    where $\Delta z$ is a complex number. \\~\\
    
    If the limit exists, then $f(z)$ is said to be \textbf{differentiable} at $z$. If $f(z)$ is differentiable at every point in a region $R$, then $f(z)$ is said to be \textbf{analytic} in $R$.
\end{definition}

\begin{definition}{Analytic Function}{}
    A complex function $f(z)$ is \textbf{analytic} in a region $R$ if it is differentiable at every point in $R$. \\~\\
    
    If $f(z)$ is analytic in a region $R$, then $f(z)$ is also said to be \textbf{regular} or \textbf{holomorphic} in $R$.
\end{definition}


\subsection{Necessary Conditions for Analyticity}
Let $f(z) = u(x,y) + iv(x,y)$ be an analytic function in a region $R$. \\~\\

That means, $f(z)$ is differentiable at every point in $R$. \\~\\

$\displaystyle\implies f'(z) = \lim_{\Delta z \to 0} \frac{f(z + \Delta z) - f(z)}{\Delta z}$ exists at every point in $R$. \\~\\

Now, let $z = x+iy$ and $\Delta z = \Delta x + i\Delta y$.
\[ z + \Delta z = (x + \Delta x) + i(y + \Delta y) \]

Then,
\begin{align*}
    \displaystyle f'(z) &= \lim \limits_{%
        \begin{subarray}{c}
            \Delta x \to 0 \\
            \Delta y \to 0
    \end{subarray}}
    \frac{f(x + \Delta x + i(y + \Delta y)) - f(x + iy)}{\Delta x + i\Delta y} \\
    &= \lim \limits_{%
        \begin{subarray}{c}
            \Delta x \to 0 \\
            \Delta y \to 0
    \end{subarray}}
    \frac{u(x + \Delta x, y + \Delta y) + iv(x + \Delta x, y + \Delta y) - u(x, y) - iv(x, y)}{\Delta x + i\Delta y} \\
    &= \lim \limits_{%
        \begin{subarray}{c}
            \Delta x \to 0 \\
            \Delta y \to 0
    \end{subarray}}
    \frac{u(x + \Delta x, y + \Delta y) - u(x, y)}{\Delta x + i\Delta y} + i\lim \limits_{%
        \begin{subarray}{c}
            \Delta x \to 0 \\
            \Delta y \to 0
    \end{subarray}}
    \frac{v(x + \Delta x, y + \Delta y) - v(x, y)}{\Delta x + i\Delta y}
\end{align*}

Along the real axis, $\Delta y = 0$. Hence, the limit is
\begin{equation*}
    f'(z) = \lim \limits_{\Delta x \to 0} \frac{u(x + \Delta x, y) - u(x, y)}{\Delta x} + i\lim \limits_{\Delta x \to 0} \frac{v(x + \Delta x, y) - v(x, y)}{\Delta x}
\end{equation*}
\begin{equation}
    f'(z) = \frac{\partial u}{\partial x} + i\frac{\partial v}{\partial x}
\end{equation}

Along the imaginary axis, $\Delta x = 0$. Hence, the limit is
\begin{equation*}
    f'(z) = \lim \limits_{\Delta y \to 0} \frac{u(x, y + \Delta y) - u(x, y)}{i\Delta y} + i\lim \limits_{\Delta y \to 0} \frac{v(x, y + \Delta y) - v(x, y)}{i\Delta y}
\end{equation*}
\begin{equation}
    f'(z) = -i\frac{\partial u}{\partial y} + \frac{\partial v}{\partial y}
\end{equation}


\subsection{Cauchy-Riemann Equations}
Since $f'(z)$ exists, (2.2.4) and (2.2.5) must be equal.
\begin{equation}
    \frac{\partial u}{\partial x} + i\frac{\partial v}{\partial x} = \frac{\partial v}{\partial y} - i\frac{\partial u}{\partial y}
\end{equation}
Comparing real and imaginary parts,
\begin{equation}
    \boxed{ \frac{\partial u}{\partial x} = \frac{\partial v}{\partial y} }
    \quad \text{and} \quad 
    \boxed{ \frac{\partial v}{\partial x} = -\frac{\partial u}{\partial y} }
\end{equation}

These are called the \textbf{Cauchy-Riemann equations}. \\~\\


\subsection{Cauchy-Riemann Equations in Polar Form}
Let \[
    z = re^{i\theta}
\] \[
    f(z) = u(r,\theta) + iv(r,\theta)
\] 
Then
\begin{equation}
    f(re^{i\theta}) = u(r,\theta) + iv(r,\theta)
\end{equation}

Differentiating (2.4.1) with respect to $r$, we get
\begin{equation}
    e^{i\theta}f'(re^{i\theta}) = \frac{\partial u}{\partial r} + i\frac{\partial v}{\partial r}
\end{equation}

Differentiating (2.4.1) with respect to $\theta$, we get
\begin{equation}
    ire^{i\theta}f'(re^{i\theta}) = \frac{\partial {u}}{\partial {\theta}} + i\frac{\partial {v}}{\partial {\theta}}
\end{equation}

Now, from (2.4.2) and (2.4.3),
\begin{align*}
    \pdv{u}{r} + i \pdv{v}{r} &= \frac{1}{ir} \left( \pdv{u}{\theta} + i \pdv{v}{\theta} \right)  \\
    \pdv{u}{r} + i \pdv{v}{r} &= \frac{1}{ir} \pdv{u}{\theta} + \frac{1}{r} \pdv{v}{\theta} \\
    \pdv{u}{r} + i \pdv{v}{r} &= -\frac{i}{r} \pdv{u}{\theta} + \frac{1}{r} \pdv{v}{\theta}
\end{align*}

Equating the real and imaginary parts, we get
\begin{equation}
    \boxed{ \pdv{u}{r} = \frac{1}{r} \pdv{v}{\theta} }
    \quad \text{and} \quad 
    \boxed{ \pdv{v}{r} = -\frac{1}{r} \pdv{u}{\theta} }
\end{equation}

These are the \textbf{Cauchy-Riemann equations in polar form}.
