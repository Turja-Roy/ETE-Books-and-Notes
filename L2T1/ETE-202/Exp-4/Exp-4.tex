\documentclass[12pt]{article}
\usepackage[a4paper, margin=0.75in]{geometry}
\usepackage[document]{ragged2e}
\usepackage{graphicx}
\usepackage{subcaption}
\usepackage{placeins}
\graphicspath{ {./images/} }
% \usepackage[tmargin=2cm,rmargin=1in,lmargin=1in,margin=0.85in,bmargin=2cm,footskip=.2in]{geometry}
\usepackage{enumerate}
\usepackage{framed}
\usepackage{amsmath,amsfonts,amsthm,thmtools,amssymb,mathtools,commath}
\usepackage{tikz}
\usepackage{xcolor}
\usepackage[most]{tcolorbox}


\tcbuselibrary{theorems}
\newtcbtheorem[number within=section]{example}{Example}%
{
    colback=green!5,
    frame hidden,
    detach title,
    before upper = \tcbtitle\par\smallskip,
    coltitle=green!35!black,
    % colframe=green!35!black,
    fonttitle=\bfseries\sffamily
    % description font=\mdseries
}{th}


\title{
    \textbf{\Large Experiment 4}\\
    \textbf{\Large Experimentation on Class A, Class B, and Class B push-pull Multistage Amplifier}
}

\author{
    Turja Roy\\
    2108052
}
\date{February 06, 2024}

\begin{document}
\maketitle

\section{Objective}
\begin{enumerate}
    \item To study the working principle of a Class A Multistage Amplifier.
    \item To study the working principle of a Class B and Class B push pull Multistage Amplifier.
    \item To measure the voltage gain of the amplifier.
\end{enumerate}

\section{Apparatus}
\begin{enumerate}
    \item Transistors (2N3904 npn) - 2
    \item Resistors
    \item Capacitors
    \item DC Power Supply (0-30V)
    \item Function Generator
    \item Oscilloscope
    \item Multimeter
    \item Breadboard, Connecting wires, etc.
\end{enumerate}

\section{Circuit Diagram}
In the following page is given the circuit diagrams for the Class A, Class B, and Class B push-pull Multistage Amplifiers. Figure 1 shows the circuit diagram for the Class A Amplifier. Figure 2 shows the circuit diagram for the Class B Amplifier. Figure 3 shows the circuit diagram for the Class B push-pull Amplifier.

\newpage
\begin{figure}[!htpb]
    \centering
    \includegraphics[width=0.5\textwidth]{Class_A_Diagram.png}
    \caption{Class A Multistage Amplifier Circuit Diagram}
\end{figure}

\begin{minipage}{0.5\textwidth}
    \centering
    \includegraphics[width=\textwidth]{Class_B_Diagram.jpg}
    \captionof{figure}{Class B Amplifier Circuit Diagram}
\end{minipage}
\begin{minipage}{0.5\textwidth}
    \centering
    \includegraphics[width=\textwidth]{Class_B_push-pull_Diagram.jpg}
    \captionof{figure}{Class B Push-Pull Amplifier Circuit Diagram}
\end{minipage}
\vspace{1cm}

In Figure 1, The Class A multistage amplifier was built by connecting two amplifier circuits through the capacitor $C_5$. The output of the first amplifier ($Q_1$) acts as the input of the second amplifier ($Q_3$). The 10k $\Omega$ is the load resistor in the circuit. Oscilloscope has been connected in the shown way to get the input and output curves. \\~\\

In Figure 2, The Class B amplifier was built using two npn transistor. The input signal was given to the base and $R_1$ and $R_2$ were base resistors. The output was taken from the emitter and the $R_3 = 100 \Omega$ was the load resistor. \\~\\

In Figure 3, The Class B push-pull amplifier was built using two npn transistors. The input signal was given parallelly to the bases of $Q_1$ and $Q_2$. The base resistance is $R_{th} = R_1 \| (R_2 + R_3) \| R_4$. The emitter of $Q_1$ and the collector of $Q_2$ were connected to the output terminal. The load resistor was $R_5 = 100 \Omega$.

\newpage
\section{Result Analysis}

\subsection{Class A Multistage Amplifier}
A multistage amplifier amplifies a signal in multiple stages. The input of the first stage is amplified and collected at at the output terminal and that output becomes the input for the second stage. This is how a signal gets amplified in multiple stages. The overall voltage gain of the signal is going to be the multiplication of the voltage gain of each of these circuits. Table 1 shows the data for the Class A multistage amplifier.

\subsubsection{Data Table for Class A Multistage Amplifier}
\bgroup
\def\arraystretch{1.5}
\begin{table}[h!]
    \centering
    \caption{Data table for Class A multistage amplifier}
    \begin{tabular}{|c|c|c|c|c|c|c|}
        \hline
        \multicolumn{7}{|c|}{\textbf{Simulation Data}} \\
        \hline
        $V_{in}$ (mV) & $V_{out}$ (mV) & $I_{in}$ ($\mu$A) & $I_{out}$ ($\mu$A) & $A_V$ & $A_I$ & $A_P$ \\ \hline
        50 & 617 & 19.2 & 61.7 & 12.34 & 3.21 & 39.61 \\ \hline\hline
        \multicolumn{7}{|c|}{\textbf{Practical Data}} \\
        \hline
        $V_{in}$ (mV) & $V_{out}$ (mV) & $I_{in}$ ($\mu$A) & $I_{out}$ ($\mu$A) & $A_V$ & $A_I$ & $A_P$ \\ \hline
        50 & 572 & 15.2 & 65.7 & 11.44 & 4.32 & 49.42 \\ \hline\hline
        \multicolumn{7}{|c|}{\textbf{Error Analysis (Simulation vs Practical Data)}} \\
        \hline
        $V_{in}$ \% & $V_{out}$ \% & $I_{in}$ \% & $I_{out}$ \% & $A_V$ \% & $A_I$ \% & $A_P$ \% \\ \hline
        0 & 10.86 & 20.83 & 6.48 & 0.38 & 34.58 & 24.77 \\ \hline
    \end{tabular}
\end{table}
\egroup

\subsubsection{Class A Multistage Amplifier Input and Output Graphs}
\begin{figure}[h!]
    \centering
    \begin{subfigure}{0.45\textwidth}
        \includegraphics[width=0.8\textwidth]{Class_A_Graph.png}
        \caption{Simulated Graph}
    \end{subfigure}
    \begin{subfigure}{0.45\textwidth}
        \includegraphics[width=0.8\textwidth]{Class_A_Practical.jpg}
        \caption{Experimental Graph}
    \end{subfigure}
    \caption{Simulated and Practical Input and Output Graphs for Class A Multistage Amplifier}
\end{figure}

In the simulated graph (Figure 4-a), the yellow and the green dashed lines show the input and output curves respectively. In the practical graph (Figure 4-b), the shorter amplitide curve is input graph and the other one is output curve.

\FloatBarrier
\subsection{Class B Amplifier}
The Class B amplifier operates in such a way that the positive half cycle of the input signal is amplified by one transistor and the negative half cycle is amplified by another transistor. The output of these two transistors are then combined to get the amplified output. Table 2 shows the data for the Class B amplifier.

\subsubsection{Data Table for Class B Amplifier}
\bgroup
\def\arraystretch{1.5}
\begin{table}[h!]
    \centering
    \caption{Data table for Class B Amplifier}
    \begin{tabular}{|c|c|c|c|c|c|c|}
        \hline
        \multicolumn{7}{|c|}{\textbf{Simulation Data}} \\
        \hline
        $V_{in}$ (mV) & $V_{out}$ (mV) & $I_{in}$ (mA) & $I_{out}$ (mA) & $A_V$ & $A_I$ & $A_P$ \\ \hline
        707 & 456 & 1.17 & 4.56 & 0.64 & 3.90 & 2.50 \\ \hline\hline
        \multicolumn{7}{|c|}{\textbf{Practical Data}} \\
        \hline
        $V_{in}$ (mV) & $V_{out}$ (mV) & $I_{in}$ (mA) & $I_{out}$ (mA) & $A_V$ & $A_I$ & $A_P$ \\ \hline
        707 & 432 & 1.12 & 4.32 & 0.61 & 3.57 & 2.96 \\ \hline\hline
        \multicolumn{7}{|c|}{\textbf{Error Analysis (Simulation vs Practical Data)}} \\
        \hline
        $V_{in}$ \% & $V_{out}$ \% & $I_{in}$ \% & $I_{out}$ \% & $A_V$ \% & $A_I$ \% & $A_P$ \% \\ \hline
        0 & 5.31 & 4.27 & 5.26 & 4.69 & 8.33 & 18.40 \\ \hline
    \end{tabular}
\end{table}
\egroup

\subsubsection{Class B Amplifier Input and Output Graphs}
\begin{figure}[h!]
    \centering
    \begin{subfigure}{0.45\textwidth}
        \includegraphics[width=\textwidth]{Class_B_Graph.jpg}
        \caption{Simulated Graph}
    \end{subfigure}
    \begin{subfigure}{0.45\textwidth}
        \includegraphics[width=\textwidth]{Class_B_Practical.jpg}
        \caption{Experimental Graph}
    \end{subfigure}
    \caption{Simulated and Practical Input and Output Graphs for Class B Amplifier}
\end{figure}

In the simulated graph (Figure 5-a), the red full waves and the clipped half waves show the input and output curves respectively. In the practical graph (Figure 5-b), the shorter amplitide curve is output graph and the other one is input curve.


\FloatBarrier
\subsection{Class B Push-Pull Amplifier}
The Class B push-pull amplifier is a combination of two Class B amplifiers. The input signal is divided into two halves and each half is amplified by one transistor. The output of these two transistors are then combined to get the amplified output. Table 3 shows the data for the Class B push-pull amplifier.

\subsubsection{Data Table for Class B Push-Pull Amplifier}
\bgroup
\def\arraystretch{1.5}
\begin{table}[h!]
    \centering
    \caption{Data table for Class B Push Pull Amplifier}
    \begin{tabular}{|c|c|c|c|c|c|c|}
        \hline
        \multicolumn{7}{|c|}{\textbf{Simulation Data}} \\
        \hline
        $V_{in}$ (mV) & $V_{out}$ (mV) & $I_{in}$ ($\mu$A) & $I_{out}$ (mA) & $A_V$ & $A_I$ & $A_P$ \\ \hline
        707 & 535 & 339 & 5.35 & 0.76 & 15.78 & 11.99 \\ \hline\hline
        \multicolumn{7}{|c|}{\textbf{Practical Data}} \\
        \hline
        $V_{in}$ (mV) & $V_{out}$ (mV) & $I_{in}$ ($\mu$A) & $I_{out}$ (mA) & $A_V$ & $A_I$ & $A_P$ \\ \hline
        707 & 512 & 345 & 5.12 & 0.72 & 15.63 & 13.28 \\ \hline\hline
        \multicolumn{7}{|c|}{\textbf{Error Analysis (Simulation vs Practical Data)}} \\
        \hline
        $V_{in}$ \% & $V_{out}$ \% & $I_{in}$ \% & $I_{out}$ \% & $A_V$ \% & $A_I$ \% & $A_P$ \% \\ \hline
        0 & 4.67 & 1.76 & 4.67 & 5.56 & 0.96 & 11.07 \\ \hline
    \end{tabular}
\end{table}
\egroup

\subsubsection{Class B Push-Pull Amplifier Input and Output Graphs}
\begin{figure}[h!]
    \centering
    \begin{subfigure}{0.45\textwidth}
        \includegraphics[width=\textwidth]{Class_B_push-pull_Graph.jpg}
        \caption{Simulated Graph}
    \end{subfigure}
    \begin{subfigure}{0.45\textwidth}
        \includegraphics[width=\textwidth]{Class_B_push-pull_Practical.jpg}
        \caption{Experimental Graph}
    \end{subfigure}
    \caption{Simulated and Practical Input and Output Graphs for Class B Push-Pull Amplifier}
\end{figure}

In the simulated graph (Figure 6-a), the red full waves and the black full waves show the input and output curves respectively. The output curve has crossover distortion. In the practical graph (Figure 6-b), the output curve has crossover distortion.

\section{Discussion}
The experiment was conducted to observe the amplification of a signal in different types of amplifiers (Class A, B, and B push pull). Error had been measured between the simulated data and the practical data. The error analysis shows that the accuracy of the practical data is roughly within the range of 75\% to 90\%. The Class B push-pull amplifier had crossover distortion in the output curve. \\~\\

One of the possible reason for this discrepancy was that in the simulated data, the internal resistance of wires and other equipments were considered negligible. But in practical, these resistances are not negligible. Another reason could be the temperature of the transistors. The temperature of the transistors could have been different in the practical data and the simulated data.

\end{document}
