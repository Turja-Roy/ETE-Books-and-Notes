\documentclass[12pt]{article}
\usepackage[a4paper, margin=0.75in]{geometry}
\usepackage[document]{ragged2e}
\usepackage{graphicx}
\graphicspath{ {./images/} }
% \usepackage[tmargin=2cm,rmargin=1in,lmargin=1in,margin=0.85in,bmargin=2cm,footskip=.2in]{geometry}
\usepackage{enumerate}
\usepackage{framed}
\usepackage{amsmath,amsfonts,amsthm,thmtools,amssymb,mathtools,commath}
\usepackage{tikz}
\usepackage{xcolor}
\usepackage[most]{tcolorbox}


\tcbuselibrary{theorems}
\newtcbtheorem[number within=section]{example}{Example}%
{
    colback=green!5,
    frame hidden,
    detach title,
    before upper = \tcbtitle\par\smallskip,
    coltitle=green!35!black,
    % colframe=green!35!black,
    fonttitle=\bfseries\sffamily
    % description font=\mdseries
}{th}


\title{
    \textbf{Experiment 7} \\
    \textbf{\Large Analyzing the frequency response of a single tuned amplifier to investigate the low pass frequency and high pass frequency.}
}

\author{
    Turja Roy \\
    ID: 2108052
}
\date{}

\begin{document}
\maketitle

\section{Objective}

\begin{enumerate}
    \item To draw and analyze the frequency response of a single tuned amplifier.
    \item To investigate the low pass frequency and high pass frequency by analyzing the frequency response graphs. 
    \item Establish a comprehensive understanding of the correlation between simulation and experimental data for the frequency response of a single tuned amplifier.
\end{enumerate}

\section{Circuit Diagram}

\begin{figure}[h!]
    \centering
    \includegraphics[width=.7\textwidth]{Circuit.png}
    \caption{Circuit Diagram for Determining Frequency Response.png}
\end{figure}

A voltage divider bias circuit was constructed using an NPN transistor and four resistors. The circuit included a bypass capacitor, C2. A DC voltage source was connected to the collector of the transistor. The input voltage,V1, was applied, and the output voltage was obtained from the load Resistor, R7. Figure 1 represents the circuit diagram.

\section{Result Analysis}

\subsection{Data Table}

\begin{table}[h!]
    \centering
    \caption{Experimental Data}
    \begin{tabular}{rrrrrr}
        \hline
        Vin (mV) &  Frequency &  Logarithmic Frequency &  Vout (mV) &  Av &  Gain (dB) \\
        \hline
        100 &         50 &                1.69897 &       2600 &  26 &  28.299467 \\
        100 &        100 &                2.00000 &       5000 &  50 &  33.979400 \\
        100 &        200 &                2.30103 &       9100 &  91 &  39.180828 \\
        100 &        500 &                2.69897 &      15000 & 150 &  43.521825 \\
        100 &       1000 &                3.00000 &      17000 & 170 &  44.608978 \\
        100 &       2000 &                3.30103 &      17800 & 178 &  45.008400 \\
        100 &       5000 &                3.69897 &      18000 & 180 &  45.105450 \\
        100 &      10000 &                4.00000 &      18000 & 180 &  45.105450 \\
        100 &      20000 &                4.30103 &      18000 & 180 &  45.105450 \\
        100 &      50000 &                4.69897 &      18000 & 180 &  45.105450 \\
        100 &     100000 &                5.00000 &      18000 & 180 &  45.105450 \\
        100 &     200000 &                5.30103 &      18000 & 180 &  45.105450 \\
        100 &     500000 &                5.69897 &      18000 & 180 &  45.105450 \\
        100 &    1000000 &                6.00000 &      18000 & 180 &  45.105450 \\
        100 &    2000000 &                6.30103 &      18000 & 180 &  45.105450 \\
        100 &    5000000 &                6.69897 &      16000 & 160 &  44.082400 \\
        100 &   10000000 &                7.00000 &      12000 & 120 &  41.583625 \\
        100 &   20000000 &                7.30103 &       7500 &  75 &  37.501225 \\
        100 &   40000000 &                7.60206 &       4000 &  40 &  32.041200 \\
        \hline
    \end{tabular}
\end{table}

\begin{table}[h!]
    \centering
    \caption{Simulation Data}
    \begin{tabular}{rrrrrr}
        \hline
        Vin (mV) &  Frequency &  Logarithmic Frequency &  Vout (mV) &    Av &  Gain (dB) \\
        \hline
        100 &         50 &                1.69897 &        380 &   3.8 &  11.595672 \\
        100 &        100 &                2.00000 &        600 &   6.0 &  15.563025 \\
        100 &        200 &                2.30103 &       1100 &  11.0 &  20.827854 \\
        100 &        500 &                2.69897 &       2600 &  26.0 &  28.299467 \\
        100 &       1000 &                3.00000 &       5000 &  50.0 &  33.979400 \\
        100 &       2000 &                3.30103 &       9100 &  91.0 &  39.180828 \\
        100 &       5000 &                3.69897 &      15200 & 152.0 &  43.636872 \\
        100 &      10000 &                4.00000 &      17400 & 174.0 &  44.810985 \\
        100 &      20000 &                4.30103 &      18200 & 182.0 &  45.201428 \\
        100 &      50000 &                4.69897 &      18200 & 182.0 &  45.201428 \\
        100 &     100000 &                5.00000 &      18200 & 182.0 &  45.201428 \\
        100 &     200000 &                5.30103 &      18200 & 182.0 &  45.201428 \\
        100 &     500000 &                5.69897 &      18200 & 182.0 &  45.201428 \\
        100 &    1000000 &                6.00000 &      18200 & 182.0 &  45.201428 \\
        100 &    2000000 &                6.30103 &      18200 & 182.0 &  45.201428 \\
        100 &    5000000 &                6.69897 &      15800 & 158.0 &  43.973142 \\
        100 &   10000000 &                7.00000 &      11600 & 116.0 &  41.289160 \\
        100 &   20000000 &                7.30103 &       7500 &  75.0 &  37.501225 \\
        100 &   40000000 &                7.60206 &       4000 &  40.0 &  32.041200 \\
        \hline
    \end{tabular}
\end{table}

\subsection{Graph}
The comparison between two graphs of the frequency of the single tuned ampilifier which had been obtained from experimental and simulation data are given below.

\begin{figure}[htpb]
    \centering
    \includegraphics[width=.8\textwidth]{Graph.png}
    \caption{Plot of Frequency Response}
\end{figure}

\section{Discussion}
The experiment focused on analyzing the frequency response of a single-tuned amplifier. The output voltage varied significantly across different frequencies despite a constant input voltage, reflecting typical amplifier behavior. Notably, the experimental and simulated gain values showed discrepancies, likely due to real-world factors like parasitic elements and component tolerances. However, both sets of data exhibited similar overall frequency response curves, indicating a strong correlation between experimental and simulated results. This correlation validates the experimental approach and confirms the theoretical predictions of frequency response for single-tuned amplifiers. The experiment thus successfully demonstrated the amplifier's performance characteristics across a wide frequency range.

\end{document}
