\documentclass[12pt]{article}
\usepackage[a4paper, margin=0.75in]{geometry}
\usepackage[document]{ragged2e}
\usepackage{graphicx}
\usepackage{circuitikz}
\graphicspath{ {./images/} }
% \usepackage[tmargin=2cm,rmargin=1in,lmargin=1in,margin=0.85in,bmargin=2cm,footskip=.2in]{geometry}
\usepackage{enumerate}
\usepackage{framed}
\usepackage{amsmath,amsfonts,amsthm,thmtools,amssymb,mathtools,commath}
\usepackage{tikz}
\usepackage{xcolor}
\usepackage[most]{tcolorbox}


\tcbuselibrary{theorems}
\newtcbtheorem[number within=section]{example}{Example}%
{
    colback=green!5,
    frame hidden,
    detach title,
    before upper = \tcbtitle\par\smallskip,
    coltitle=green!35!black,
    % colframe=green!35!black,
    fonttitle=\bfseries\sffamily
    % description font=\mdseries
}{th}


\title{
    \textbf{ETE-201} \\
    \textsc{Operational Amplifiers}
}

\author{
    Note taken by: Turja Roy \\ 
    ID: 2108052
}
\date{}

\begin{document}
\maketitle

\tableofcontents

\newpage
\section{Introduction}
\begin{definition}{Operational Amplifier (Op-Amp)}{}
    An operational amplifier (op-amp) is a high-gain electronic voltage amplifier with a differential input and, usually, a single-ended output.
\end{definition}

\begin{definition}{The Ideal Op-Amp}{}
    An ideal op-amp has the following properties:
    \begin{itemize}[itemsep=0pt]
        \item Infinite input impedance
        \item Zero output impedance
        \item Infinite open-loop gain
        \item Zero input offset voltage
        \item Infinite bandwidth
    \end{itemize}
\end{definition}

\begin{definition}{The Practical Op-Amp}{}
    A practical op-amp has the following properties:
    \begin{itemize}[itemsep=0pt]
        \item High input impedance
        \item Low output impedance
        \item High open-loop gain
        \item Low input offset voltage
        \item Limited bandwidth
    \end{itemize}
\end{definition}\vspace{10pt}

\begin{figure}[htpb]
    \centering
    \begin{tikzpicture}
        % Manually draw a big triangle
        
    \end{tikzpicture}
    \caption{}
\end{figure}


\end{document}
