\documentclass[12pt]{article}
\usepackage[legalpaper, margin=0.75in]{geometry}
\usepackage[document]{ragged2e}
\usepackage{graphicx}
\graphicspath{ {./images/} }
\usepackage{enumerate}
\usepackage{framed}
\usepackage{amsmath,amsfonts,amsthm,thmtools,amssymb,mathtools,commath}
\usepackage{physics}
\usepackage{tikz}
\usetikzlibrary{mindmap}
\usepackage{caption}
\usepackage{xcolor}
\usepackage[most]{tcolorbox}
\usepackage{cleveref}


%%%%%%%%%%%%%%%%
%  Definition  %
%%%%%%%%%%%%%%%%
\tcbuselibrary{theorems,skins,hooks}
\newtcbtheorem[number within=subsection]{definition}{Definition}%
{
    % theorem style=definition,
    enhanced,
	before skip=2mm,after skip=2mm, colback=cyan!5,colframe=cyan!80!black,boxrule=0.5mm,
	attach boxed title to top left={xshift=1cm,yshift*=1mm-\tcboxedtitleheight},
	boxed title style={frame code={
					\path[fill=cyan]
					([yshift=-1mm,xshift=-1mm]frame.north west)
					arc[start angle=0,end angle=180,radius=1mm]
					([yshift=-1mm,xshift=1mm]frame.north east)
					arc[start angle=180,end angle=0,radius=1mm];
					\path[left color=cyan!30!black,right color=cyan!30!black,
						middle color=cyan!50!black]
					([xshift=-2mm]frame.north west) -- ([xshift=2mm]frame.north east)
					[rounded corners=1mm]-- ([xshift=1mm,yshift=-1mm]frame.north east)
					-- (frame.south east) -- (frame.south west)
					-- ([xshift=-1mm,yshift=-1mm]frame.north west)
					[sharp corners]-- cycle;
				},interior engine=empty,
		},
	fonttitle=\bfseries,
	title={#2},#1
}{def}


%%%%%%%%%%%%%
%  Theorem  %
%%%%%%%%%%%%%
\tcbuselibrary{theorems,skins,hooks}
\newtcbtheorem[use counter from=definition]{theorem}{Theorem}%
{
    theorem style=plain,
    enhanced,
    colframe=green,
    boxrule=1pt,
    titlerule=0mm,
    toptitle=1mm,
    bottomtitle=1mm,
    fonttitle=\bfseries,
    fontupper=\mdseries\itshape,
    coltitle=green!30!black,
    colbacktitle=cyan!15!white,
    colback=green!10,
    description font=\bfseries\sffamily
}{thrm}


%%%%%%%%%%%%%%
% Corollary  %
%%%%%%%%%%%%%%
 \tcbuselibrary{theorems,skins}
 \newtcbtheorem[use counter from=theorem]{corollary}{Corollary}%
 {
    theorem style=plain,
    enhanced,
    colframe=green,
    frame hidden,
    titlerule=0mm,
    toptitle=1mm,
    bottomtitle=1mm,
    fonttitle=\bfseries,
    fontupper=\mdseries\itshape,
    coltitle=green!30!black,
    colbacktitle=cyan!15!white,
    colback=green!10,
    description font=\bfseries\sffamily
 }{corl}


%%%%%%%%%%%%%
%  Example  %
%%%%%%%%%%%%%
\tcbuselibrary{theorems,skins,hooks}
\newtcbtheorem[number within=section]{example}{Example}%
{
	enhanced,
	breakable,
	colback = gray!5,
	frame hidden,
	boxrule = 0sp,
	borderline west = {2pt}{0pt}{gray},
	sharp corners,
	detach title,
	before upper = \tcbtitle\par\smallskip,
    coltitle=gray!70!black,
	fonttitle = \bfseries\sffamily,
	description font = \mdseries\bfseries
}
{xmp}


%%%%%%%%%%%%%%
%  Exercise  %
%%%%%%%%%%%%%%
\tcbuselibrary{theorems,skins,hooks}
\newtcbtheorem[number within=section]{exercise}{Exercise}%
{
    enhanced,
    breakable,
    colback=black!5,
    colframe=black!30,
    left=0.5em,
    before skip=10pt,
    after skip=10pt,
    boxrule=0pt,
    boxsep=0pt,
    arc=0pt,
    outer arc=0pt,
    borderline west={3pt}{0pt}{black!30},
}{exc}

%%%%%%%%%%
%  Note  %
%%%%%%%%%%
\usetikzlibrary{arrows,calc,shadows.blur}
\tcbuselibrary{skins}
\newtcolorbox{note}[1][]{%
	enhanced jigsaw,
	colback=gray!20!white,%
	colframe=gray!80!black,
	size=small,
	boxrule=1pt,
	title=\textbf{Note:-},
	halign title=flush center,
	coltitle=black,
	breakable,
	drop shadow=black!50!white,
	attach boxed title to top left={xshift=1cm,yshift=-\tcboxedtitleheight/2,yshifttext=-\tcboxedtitleheight/2},
	minipage boxed title=1.5cm,
	boxed title style={%
			colback=white,
			size=fbox,
			boxrule=1pt,
			boxsep=2pt,
			underlay={%
					\coordinate (dotA) at ($(interior.west) + (-0.5pt,0)$);
					\coordinate (dotB) at ($(interior.east) + (0.5pt,0)$);
					\begin{scope}
						\clip (interior.north west) rectangle ([xshift=3ex]interior.east);
						\filldraw [white, blur shadow={shadow opacity=60, shadow yshift=-.75ex}, rounded corners=2pt] (interior.north west) rectangle (interior.south east);
					\end{scope}
					\begin{scope}[gray!80!black]
						\fill (dotA) circle (2pt);
						\fill (dotB) circle (2pt);
					\end{scope}
				},
		},
	#1,
}

\usepackage[scr]{rsfso}
\usepackage{physics}
\usepackage{multicol}
\renewcommand{\arraystretch}{2.5}

\newcommand{\Lap}{\mathscr{L}}
\newcommand{\Lapinv}{\mathscr{L}^{-1}}

\title{
    \textbf{Laplace Transform}
}

\author{
    Turja Roy\\
    ID: 2108052
}
\date{}

\begin{document}
\maketitle
\tableofcontents
\newpage

%%%%%%%%%%%%%%%%%%%%%%%%%%%%%%
%  Definition and Existence  %
%%%%%%%%%%%%%%%%%%%%%%%%%%%%%%

\section{Laplace Transform}

%%%%%%%%%%%%%%%%%%%%%%%%%%%%%%
%  Definition and Existence  %
%%%%%%%%%%%%%%%%%%%%%%%%%%%%%%

\subsection{Definition and Existence}

\begin{definition}{Laplace Transform}{}
    Let $F$ be a real-valued function of the real variable $t$, defined for $t>0$. Let $s$ be a variable that we shall assume to be real, and consider the function $f$ defined by
    \begin{equation} \label{eq1}
        f(s) = \int_{0}^{\infty} {e^{-st} F(t)} \: d{t} 
    \end{equation}
    for all values of $s$ for which this integral exists. The function $f$ defined by the integral \eqref{eq1} is called the Laplace Transform of the function $F$. We shall denote the Laplace transform of $F$ by $\Lap\{F(t)\}$.\\
    Thus the Laplace transform of a function $f$ is given by
    \begin{equation} \label{eq2}
        \Lap\{ F(t) \} = f(s) = \int_{0}^{\infty} {e^{-st} F(t)} \: d{t}  = \lim_{R \to \infty} \int_{0}^{R} {e^{-st} F(t)} \: d{t} 
    \end{equation}
\end{definition}

Some ways to write Laplace transforms:\\
\[ \Lap{F(t)} = f(s) = \int_{0}^{\infty} { e^{-st} F(t) } \: d{t} \]
\[ \Lap{G(t)} = g(s) \]
\[ \Lap{u(t)} = \tilde{u}(s) \]

\vspace{30pt}
\begin{theorem}{}{}
    \textbf{Hypothesis: } Let $F$ be a real function that has the following properties:
    \begin{enumerate}
        \item $F$ is a piecewise continuous in every finite closed interval $0 \le t \le a \: (b>0)$.
        \item $F$ is of exponential order, i.e, there exists $\alpha$, $M>0$, and $t_0>0$ such that
    \end{enumerate}
    \[ e^{-\alpha t}|F(t)| < M \text{ for } t>t_0 \]
    
    \textbf{Conclusion: } The Laplace transform of $F$ exists for $s>\alpha$.
    \[
        \Lap \{ F(t) \} = \int_{0}^{\infty} {e^{-s t}F(t)} \: d{t}
    \] 
\end{theorem}

% \newpage
%%%%%%%%%%%%%%%%%%%%%%%%%%%%%%%%%%%%%%%%%%%%%%%%%
%  Some Functions and Their Laplace Transforms  %
%%%%%%%%%%%%%%%%%%%%%%%%%%%%%%%%%%%%%%%%%%%%%%%%%
\subsection{Some Functions and Their Laplace Transforms}

\begin{table}[htpb]
    \centering

    \begin{tabular}{c | c}
        \hline
        $F(t)$ &  $\Lap\{ F(t) \} = f(s)$ \\
        \hline\hline
        $1$ & $\dfrac{1}{s}$ \\\hline
        $t$ & $\dfrac{1}{s^2}$ \\\hline
        $e^{at}$ & $\dfrac{1}{s-a}$ \\\hline
        $\sin{at}$ & $\dfrac{a}{s^2+a^2}$ \\\hline
        $\cos{at}$ & $\dfrac{s}{s^2+a^2}$ \\
        \hline
    \end{tabular}
    \begin{tabular}{c | c}
        \hline
        $F(t)$ &  $\Lap\{ F(t) \} = f(s)$ \\
        \hline\hline
        $n$ & $\dfrac{n}{s}$ \\\hline
        $t^n$ &  $\dfrac{n!}{s^{n+1}}$ \\\hline
        $e^{-at}$ & $\dfrac{1}{s+a}$ \\\hline
        $\sinh{at}$ & $\dfrac{a}{s^2-a^2}$ \\\hline
        $\cosh{at}$ &  $\dfrac{s}{s^2-a^2}$ \\
        \hline
    \end{tabular}

    \caption{Functions and their Laplace Transform}
    \label{LapTable}
\end{table}

\begin{multicols}{2}
    [
    \underline{\textbf{Proofs :}}\\
    ]
    \[ \Lap \{ n \} = \frac{n}{s} \]
    Let $F(t) = n$, for $t>0$\\
    Then\\
    \begin{align*}
        \Lap \{ n \} &= \int_{0}^{\infty} { e^{-st} \cdot n } \: d{t} \\
                     &= \eval{ n\frac{-e^{st}}{s} }_{0}^{\infty} \\
                     &= \frac{n}{s}\qed
    \end{align*}\\
    \columnbreak

    \[ \Lap \{ t \} = \frac{1}{s^2} \]
    Let $F(t) = t$, for $t>0$\\
    Then\\
    \begin{align*}
        \Lap \{ t \} &= \int_{0}^{\infty} { e^{-st} \cdot t } \: d{t} \\
                     &= -t \frac{e^{-st}}{s} + \int_{0}^{\infty} {\frac{e^{-st}}{s}} \: d{t} \\
                     &= -e^{-st}\frac{t}{s}\bigg|_{0}^{\infty} - e^{-st}\frac{1}{s^2}\bigg|_{0}^{\infty} \\
                     &= \frac{1}{s^2}\qed
    \end{align*}
\end{multicols}

\[ \Lap \{ t^n \} = \frac{n!}{s^{n+1}} \]
Let $F(t) = t^{n}$, for $t>0$ \\
Then\\
\begin{align*}
    \Lap \{ t^{n} \} &= \int_{0}^{\infty} {e^{-s t}t^n} \: d{t} \\
                     &= -t^n \frac{e^{s t}}{s} + \int_{0}^{\infty} { nt^{n-1} \frac{e^{-s t}}{s} } \: d{t} \\
                     &= \eval{ -nt^{n-1} \frac{e^{-s t}}{s^2} }_{0}^{\infty} + \int_{0}^{\infty} { n(n-1)t^{n-2} \frac{e^{-s t}}{s^2} } \: d{t} \\
                     &= \eval{ -n(n-1)t^{n-2} \left( \frac{e^{-s t}}{s^3} \right) }_{0}^{\infty} + \int_{0}^{\infty} { n(n-1)(n-2)t^{n-3} \frac{e^{-s t}}{s^3} } \: d{t}  \\
                     &= \cdots \\
                     &= \eval{n!t^{n-n}\frac{e^{-s t}}{s^{n+1}}}_{0}^{\infty} + \int_{0}^{\infty} { n(n-1)\cdots(n-n) \frac{e^{-s t}}{s^{n+1}} } \: d{t} \\
                     &= \frac{n!}{s^{n+1}}\qed
\end{align*}\\~\\

\begin{multicols}{2}
    Let $F(t) = e^{at}$, for $t>0$\\
    Then\\
    \begin{align*}
        \Lap \{ e^{at} \} &= \int_{0}^{\infty} { e^{-s t} e^{at} } \: d{t} \\
                          &= \int_{0}^{\infty} { e^{(a-s)t} } \: d{t} \\
                          &= \eval{ \frac{e^{(a-s)t}}{a-s} }_{0}^{\infty} \\
                          &= \frac{1}{s-a}\qed
    \end{align*}
    \columnbreak
    
    Let $F(t)=e^{-at}$, for $t>0$ \\
    Then\\
    \begin{align*}
        \Lap \{ e^{-at} \} &= \int_{0}^{\infty} {e^{-s t} e^{-at}} \: d{t} \\
        &= \int_{0}^{\infty} {e^{-(a+s)t}} \: d{x} \\
        &= \eval{ \frac{e^{-(a+s)t}}{s+a} }_{0}^{\infty} \\
        &= \frac{1}{s+a}\qed
    \end{align*}
\end{multicols}

\newpage
\begin{multicols}{2}
    Let $F(t) = \sin{at}$, for $t>0$ \\
    Then\\
    \begin{align*}
        \Lap \{ \sin{at} \} &= \int_{0}^{\infty} { e^{-s t \sin{at}} } \: d{t} \\
        &= - \eval{ \frac{e^{-s t}}{s^2+a^2} \left( s\sin{at} + a\cos{at} \right) }_{0}^{\infty} \\
        &= \frac{a}{s^2+a^2}\qed
    \end{align*}
    \columnbreak
    
    Let $F(t) = \cos{at}$, for $t>0$ \\
    Then\\
    \begin{align*}
        \Lap \{ \cos{at} \} &= \int_{0}^{\infty} { e^{-s t}\cos{at} } \: d{t} \\
        &= \eval{ \frac{e^{-s t}}{s^2+a^2} \left( -s\cos{at} + a\sin{at} \right) }_{0}^{\infty} \\
        &= \frac{s}{s^2+a^2}\qed
    \end{align*}
\end{multicols}



%%%%%%%%%%%%%%%%%%%%%%%%%%%%%%%%%%%%%%%%%%%%%%%
%  Basic Properties of the Laplace Transform  %
%%%%%%%%%%%%%%%%%%%%%%%%%%%%%%%%%%%%%%%%%%%%%%%
\vspace{30pt}
\section{Basic Properties of the Laplace Transform}

%%%%% Linearity Property %%%%%
\subsection{Linearity Property}
\begin{theorem*}{The Linearity Property}{}
    \\Let $F_1$ and $F_2$ be functions whose Laplace transform exist, and let $c_1$ and $c_2$ be constants. Then
    \[
        \Lap \{ c_1F_1(t) + c_2F_2(t) \} = c_1 \Lap \{ F_1(t) \} + c_2 \Lap \{ F_2(t) \}
    \]
\end{theorem*}

\underline{\textbf{Proof :}}\\
Let $F(t) = c_1F_1(t) + c_2F_2(t)$, for $t>0$\\
Then\\
\begin{align*}
    \Lap \{ c_1F_1(t) + c_2F_2(t) \} &= \int_{0}^{\infty} { e^{-st} \left[ c_1F_1(t) + c_2F_2(t) \right] } \: d{t} \\
    &= c_1 \int_{0}^{\infty} { e^{-st} F_1(t) } \: d{t} + c_2 \int_{0}^{\infty} { e^{-st} F_2(t) } \: d{t} \\
    &= c_1 \Lap \{ F_1(t) \} + c_2 \Lap \{ F_2(t) \}\qed
\end{align*}

\begin{example}{
        $$ \Lap \{ 4t^2 - 3\cos{2t} + 5e^{-t} \} $$
    }{}
    \begin{align*}
        \Lap \{ 4t^2 - 3\cos{2t} + 5e^{-t} \} &= 4 \Lap \{ t^2 \} - 3 \Lap \{ \cos{2t} \} + 5 \Lap \{ e^{-t} \} \\
        &= 4 \cdot \frac{2!}{s^3} - 3 \cdot \frac{s}{s^2+4} + 5 \cdot \frac{1}{s+1} \\
        &= \frac{8}{s^3} - \frac{3s}{s^2+4} + \frac{5}{s+1}
    \end{align*}
\end{example}

\begin{example}{Find $\Lap \{ F(w) \}$, when \(F(t) =
    \begin{cases}
        5 & \text{for } 0 < t < 3 \\
        0 & \text{for } t > 3
    \end{cases}\)}{}
    \begin{align*}
        \Lap \{ F(t) \} &= \int_{0}^{\infty} {e^{-st}F(t)} \: d{t}
        = \int_{0}^{3} {e^{-st} \cdot 5} \: d{t} + \int_{3}^{\infty} {0} \: d{t} \\
        &= \int_{0}^{3} {e^{-st} \cdot 5} \: d{t} \\
        &= \eval{ \frac{5e^{-st}}{s} }_{0}^{3} \\
        &= \frac{5}{s} \left( 1 - e^{-3s} \right)
    \end{align*}
\end{example}

\begin{example}{Find $\Lap \{ F(t) \}$, when \(F(t =
    \begin{cases}
        (t-1)^2 & \text{for } t > 1 \\
        0 & \text{for } 0 < t < 1
    \end{cases})\)}{}
    \begin{align*}
        \Lap \{ F(t) \} &= \int_{0}^{\infty} {e^{-s t}F(t)} \: d{t}
        = \int_{0}^{1} {0} \: d{t} + \int_{1}^{\infty} {e^{-s t}(t-1)^2} \: d{t} \\
        &= \int_{1}^{\infty} {e^{-s t}(t^2 -2t + 1)} \: d{t}  \\
        &= { - t^2 \frac{e^{-s t}}{s}\bigg|_1^\infty } - 2 \int_{1}^{\infty} { t \cdot \frac{e^{-s t}}{s} } \: d{t} - 2 \int_{1}^{\infty} { t \cdot e^{-s t} } \: d{t} - \frac{e^{-s t}}{s} \\
        &= \frac{e^{-s}}{s} + 2 \left[ -\frac{t}{s} \left( \frac{e^{-s t}}{s} \right) \bigg|_1^\infty + \int_{1}^{\infty} { \frac{e^{-s t}}{s^2} } \: d{t} + 2 t \frac{e^{-s t}}{s}\bigg|_1^\infty - \int_{1}^{\infty} {\frac{e^{-s t}}{s}} \: d{t}\right] + \frac{e^{-s t}}{s} \\
        &= \frac{e^{-s}}{s} - 2 \frac{e^{-s}}{s^2} + 2 \frac{e^{-s}}{s^2} - 2 \frac{e^{-s t}}{s} + 2 \frac{e^{-s}}{s^2} + \frac{e^{-s t}}{s} \\
        &= 2 \frac{e^{-s}}{s^3}
    \end{align*}
\end{example}

%%%%% First Translation %%%%%
\subsection{First Translation Property}
\begin{theorem}{First Translation of Shifting Property}{}
    \\If \[
        \Lap \{ F(t) \} = f(s)
    \] then
    \[ \Lap \{ e^{at}F(t) \} = f(s-a) \]
\end{theorem}

\underline{\textbf{Proof: }} \\
Let $G(t) = e^{at}F(t)$, for $t>0$\\
Then\\
\begin{align*}
    \Lap \{ e^{at}F(t) \} &= \int_{0}^{\infty} {e^{-st}e^{at}F(t)} \: d{t} \\
    &= \int_{0}^{\infty} {e^{-(s-a)t}F(t)} \: d{t} \\
    &= \Lap \{ F(t) \} \bigg|_{s-a} \\
    &= f(s-a)\qed
\end{align*}

\begin{example}{\[
    \Lap \{ e^{-t}\cos{2t} \}
\]}{}
    Since $\Lap \{ \cos{2t} \} = \dfrac{s}{s^2+4}$, we have
    \[
    \Lap \{ e^{-t}\cos{2t} \} = \frac{s+1}{(s+1)^2 + 4} = \frac{s+1}{s^2+2s+5} \}
    \]
\end{example}

\begin{example}{Evaluate $\Lap \{ e^{-2t} (3\cos{6t} - 5\sin{6t}) \}$}{}
    Now,
    \begin{align*}
        f(s) &= \Lap \{ 3\cos{6t} - 5\sin{6t} \} \\
        &= \frac{3s}{s^2+36} - \frac{30}{s^2+36} \\
        &= \frac{3s-30}{s^2+36}
    \end{align*}
    \begin{align*}
        \therefore \Lap \{ e^{-2t} (3\cos{6t} - 5\sin{6t}) \} &= f(s+2) \\
        &= \frac{3(s+2)-30}{(s+2)^2+36} \\
        &= \frac{3s-24}{s^2 + 4s + 40}
    \end{align*}
\end{example}

%%%%% Second Translation %%%%%
\subsection{Second Translation Property}
\begin{theorem}{Second Translation or Shifting Property}{}
    \\If \[
        \Lap \{ F(t) \} = f(s)
    \] and \[
        G(t) = 
        \begin{cases}
            F(t-a) & \text{for } t>a \\
            0 & \text{for } t < a
        \end{cases}
    \] then
    \[
        \Lap \{ G(t) \} = e^{-as}f(s)
    \]

\end{theorem}

\underline{\textbf{Proof: }} \\
Let \( G(t) =
\begin{cases}
    F(t-a) & \text{for } t>a \\
    0 & \text{for } t < a
\end{cases}\)\\
Then\\
\begin{align*}
    \Lap \{ G(t) \} &= \int_{0}^{\infty} {e^{-st}F(t-a)} \: d{t} \\
    &= \int_{a}^{\infty} {e^{-st}F(t-a)} \: d{t} \\
    &= \int_{0}^{\infty} {e^{-s(u+a)}F(u)} \: d{u} \quad \text{where } u=t-a \\
    &= e^{-as} \int_{0}^{\infty} {e^{-su}F(u)} \: d{u} \\
    &= e^{-as} \Lap \{ F(t) \} \\
    &= e^{-as} f(s)\qed
\end{align*}

\begin{example}{
        Find $\Lap \{ G(t) \}$ where $
        G(t) = 
        \begin{cases}
            \cos{\left( t - \frac{2\pi}{3} \right)} & \text{for } t > \frac{2\pi}{3} \\
            0 & \text{for } t < \frac{2\pi}{3}
        \end{cases}
        $
    }{}
    \begin{align*}
        \Lap \{ G(t) \} &= e^{-\frac{2\pi}{3}s} \cdot \Lap \{ \cos{t} \} \\
        &= e^{-\frac{2\pi}{3}s} \cdot \frac{s}{s^2+1} \\
        &= \frac{s e^{-\frac{2\pi}{3}s}}{s^2+1} \\
    \end{align*}
\end{example}

\begin{example}{Find $\Lap \{ F(t) \}$, if \( F(t) =
    \begin{cases}
        (t-1)^2 & \text{for } t > 1 \\
        0 & \text{for } t < 0
    \end{cases}\)}{}
    Let \[
        G(t) = t^2
    \] \[ \therefore \Lap \{ G(t) \} = \frac{2!}{s^3} \]
    Now, \[
        F(t) =
        \begin{cases}
            G(t-1) & \text{for } t > 0 \\
            0 & \text{for } t < 0
        \end{cases}
    \]
    \[ \therefore \Lap \{ F(t) \} = \frac{e^{-s} \cdot 2!}{s^3} = \frac{2e^{-s}}{s^3} \]
\end{example}

%%%%% Change of Scale %%%%%
\subsection{Change of Scale Property}
\begin{theorem}{Change of Scale Property}{}
    \\If \[
        \Lap \{ F(t) \} = f(s)
    \] then \[
        \Lap \{ F(at) \} = \frac{1}{a} f \left( \frac{s}{a} \right)
    \]
\end{theorem}

\underline{\textbf{Proof: }} \\
Let $G(t) = F(at)$, for $t>0$\\
Then\\
\begin{align*}
    \Lap \{ G(t) \} &= \int_{0}^{\infty} {e^{-st}F(at)} \: d{t} \\
    &= \int_{0}^{\infty} {e^{-\frac{s}{a}u}F(u)} \: d{(u/a)} \quad \text{where } u=at \\
    &= \frac{1}{a} \int_{0}^{\infty} {e^{-\frac{s}{a}u}F(u)} \: d{u} \\
    &= \frac{1}{a} \Lap \{ F(t) \} \\
    &= \frac{1}{a} f \left( \frac{s}{a} \right)\qed
\end{align*}

\begin{example}{Evaluate $\Lap \{ \sin{3t} \}$}{}
    \begin{align*}
        \Lap \{ \sin{3t} \} &= \frac{1}{3} \cdot \frac{1}{\left( \frac{s}{3} \right) + 1} \\
        &= \frac{1}{3} \cdot \frac{3^2}{s^2+3^2}\\
        &= \frac{3}{s^2+9}
    \end{align*}
\end{example}

\begin{example}{If $\Lap \{ \frac{\sin{t}}{t} = \tan^{-1}{\frac{1}{s}} = f(s) \}$, then evaluate $\Lap \{ \frac{\sin{at}}{t} \}$}{}
    \begin{align*}
        \Lap \{ \frac{\sin{at}}{t} \} &= \Lap \left\{ a \cdot \frac{\sin{at}}{t} \right\} \\
        &= a \cdot \Lap \left\{ \frac{\sin{at}}{at} \right\} \\
        &= a \cdot \frac{1}{a} f\left(\frac{s}{a}\right) \\
        &= \tan^{-1}{\frac{a}{s}}
    \end{align*}
\end{example}


%%%%% Multiplication by t %%%%%
\subsection{Multiplication by $t$}
\begin{theorem}{Multiplication by $t^n$}{}
    \\If \[
        \Lap \{ F(t) \} = f(s)
    \] then \[
        \Lap \{ t^nF(t) \} = (-1)^n \frac{d^n}{ds^n} f(s)
    \]
\end{theorem}

\underline{\textbf{Proof: }} \\
Let $G(t) = t^nF(t)$, for $t>0$\\
Then\\
\begin{align*}
    \Lap \{ G(t) \} &= \int_{0}^{\infty} {e^{-st}t^nF(t)} \: d{t} \\
    &= (-1)^n \int_{0}^{\infty} {e^{-st} \frac{d^n}{ds^n} F(t)} \: d{t} \\
    &= (-1)^n \frac{d^n}{ds^n} \int_{0}^{\infty} {e^{-st} F(t)} \: d{t} \\
    &= (-1)^n \frac{d^n}{ds^n} f(s)\qed
\end{align*}

\underline{\textbf{Alternative Proof: }} \\
We have \[
    f(s) = \int_{0}^{\infty} {e^{-st}F(t)} \: d{t}
\]
Then by Leibnitz's rule for differentiating under the integral sign, we have
\begin{align*}
    \dv{f}{s} = f'(s) &= \dv{s} \int_{0}^{\infty} {e^{-st}F(t)} \: d{t} \\
    &= \int_{0}^{\infty} {\dv{s} \left( e^{-st}F(t) \right)} \: d{t} \\
    &= \int_{0}^{\infty} {e^{-st} \left( -te^{-st}F(t) + F'(t) \right)} \: d{t} \\
    &= - \int_{0}^{\infty} {e^{-st} \{ tF(t) \}} \: d{t} \\
    &= - \Lap \{ tF(t) \}
\end{align*}
\[
    \therefore \Lap \{ tF(t) \} = -\dv{s}f(s) = f'(s)
\]
This proves the theorem for $n=1$.\\
To establish the theorem in general, we use mathematical induction. Suppose that the theorem is true for $n=k$, i.e 
\begin{equation} \tag{2.5.1}
    \int_{0}^{\infty} {e^{-st} \{ t^k F(t) \}} \: d{t} = (-1)^k f^{(k)}(s)
\end{equation}
Then \[
    \dv{s} \left[ \int_{0}^{\infty} {e^{-st} \{ t^k F(t) \}} \: d{t} \right] = (-1)^k f^{(k+1)}(s)
\]
Or by Leibnitz's rule, \[
    - \int_{0}^{\infty} {\dv{s} \left( e^{-st} \{ t^{k+1} F(t) \} \right)} \: d{t} = (-1)^{(k)} f^{(k+1)}(s)
\]
i.e 
\begin{equation} \tag{2.5.2}
    \int_{0}^{\infty} {e^{-st} \{ t^{k+1} F(t) \}} \: d{t} = (-1)^{(k+1)} f^{(k+1)}(s)
\end{equation}
If follows that if (2.5.1) is true for $n=k$, then (2.5.2) is true for $n=k+1$. Since (2.5.1) is true for $n=1$, it follows that (2.5.1) is true for all positive integers $n$.\qed\\~\\

\begin{example}{Find $\Lap \{ t^2 \cos{at} \}$}{}
    \begin{align*}
        \Lap \{ t^2 \cos{at} \} &= (-1)^2 \cdot \frac{d^2}{d{x^2}} \: \left( \frac{s}{s^2 + a^2} \right) \\
        &= \dv{}{s} { \left[ \frac{s^2 + a^2 - 2s^2}{ (s^2 + a^2)^2 } \right] } \\
        &= \dv{}{s} { \left[ \frac{a^2 - s^2}{(s^2 + a^2)^2} \right] } \\
        &= \frac{(s^2+a^2)^2(-2s) - (-s^2+a^2)2(s^2+a^2)\cdot2s}{(s^2+a^2)^4} \\
        &= \frac{2s^3 - 6a^2s}{(s^2+a^2)^3}
    \end{align*}
\end{example}


%%%%% Division by t %%%%%
\subsection{Division by $t$}
\begin{theorem}{Division by $t$}{}
    \\If \[
        \Lap \{ F(t) \} = f(s)
    \] then \[
        \Lap \left\{ \frac{F(t)}{t} \right\} = \int_{s}^{\infty} {f(x)} \: d{x}
    \]
\end{theorem}

\underline{\textbf{Proof: }} \\
Let $G(t) = \dfrac{F(t)}{t}$, for $t>0$. Then $F(t) = t G(t)$. Taking the Laplace Transform of both sides, we get
\begin{align*}
    \Lap \{ F(t) \} &= \Lap \{ t G(t) \} \\
    f(s) &= -\dv{s} \Lap \{ G(t) \} = - \dv{s} g(s)
\end{align*}
Then integrating, we have
\begin{align*}
    g(s) &= - \int_{\infty}^{s} {f(u)} \: d{u} \\
    \Lap \{ G(t) \} &= \int_{s}^{\infty} {f(u)} \: d{u} \\
    \Lap \left\{ \frac{F(t)}{t} \right\} &= \int_{s}^{\infty} {f(u)} \: d{u} \qed
\end{align*}

\begin{example}{Find $\Lap \{ \frac{\sin{t}}{t} \}$}{}
    \begin{align*}
        \Lap \left\{ \frac{\sin{t}}{t} \right\} &= \int_{s}^{\infty} {\frac{1}{s^2+1}} \: d{u} \\
        &= \tan^{-1}{u} \bigg|_s^\infty \\
        &= \frac{\pi}{2} - \tan^{-1}{s} = \cot^{-1}{s} \\
        &= \tan^{-1}{\frac{1}{s}}
    \end{align*}
\end{example}


%%%%% Laplace Transform of Integral %%%%%
\subsection{Laplace Transform of Integral}
\begin{theorem}{Laplace transform of Integral}{}
    \\If \[
        \Lap \{ F(t) \} = f(s)
    \] then \[
        \Lap \left\{ \int_{0}^{t} {F(x)} \: d{x} \right\} = \frac{1}{s} f(s)
    \]
\end{theorem}

\underline{\textbf{Proof: }} \\
Let $G(t) = \displaystyle \int_{0}^{t} {F(x)} \: d{x}$, for $t>0$. Then $G'(t) = F(t)$ and $G(0) = 0$. Taking the Laplace Transform of both sides, we have
\begin{align*}
    \Lap \{ G'(t) \} &= \Lap \{ F(t) \} \\
    s \Lap \{ G(t) \} - G(0) &= f(s) \\
    s \Lap \{ G(t) \} &= f(s) \\
    \Lap \{ G(t) \} &= \frac{f(s)}{s} \\
    \Lap \left\{ \int_{0}^{t} {F(u)} \: d{u} \right\} &= \frac{f(s)}{s} \qed
\end{align*}


\begin{example}{Evaluate $\Lap \{ \int_{0}^{t} {\frac{\sin{u}}{u}} \: d{u} \}$}{}
\[ \Lap \left\{ \frac{\sin{u}}{u} \right\} = \tan^{-1}{\frac{1}{s}} \]
    \[
        \Lap \left\{ \int_{0}^{t} {\frac{\sin{u}}{u}} \: d{u} \right\} = \frac{f(s)}{s} = \frac{1}{s} \tan^{-1}{\frac{1}{s}}
    \]
\end{example}

\begin{example}{Evaluate $\Lap \{ \int_{0}^{t} {\frac{\sin{u}}{u}} \: d{u} \}$}{}
    Let $F(t) = \frac{\sin{t}}{t}$
    \begin{align*}
        \Lap \left\{ \frac{\sin{t}}{t} \right\} &= \int_{s}^{\infty} {\frac{1}{u^2+1}} \: d{u} \\
                                                &= \tan^{-1}{u} \bigg|_s^\infty \\
                                                &= \tan^{-1}{\frac{1}{s}}
    \end{align*}
    \[
        \therefore \Lap \left\{ \int_{0}^{t} {\frac{\sin{u}}{u}} \: d{u} \right\} = \frac{1}{s} \cdot \tan^{-1}{\frac{1}{s}}
    \]

\end{example}

\begin{example}{Evaluate $\Lap \{ \int_{0}^{t} {\sin{2u}} \: d{u} \}$}{}
    \[ \Lap \{ \sin{2t} \} = \frac{2}{s^2+4} \]
    \[ \therefore \Lap \left\{ \int_{0}^{t} {\sin{2u}} \: d{u} \right\} = \frac{2}{s^3 + 4s} \]
\end{example}

%%%%% Laplace Transform of Periodic Functions %%%%%
\subsection{Laplace Transform of Periodic Functions}
\begin{theorem}{Periodic functions}{}
    \\If \[
        \Lap \{ F(t) \} = f(s)
    \] then \[
        \Lap \{ F(t) \} = \frac{1}{1-e^{-Ts}} \int_{0}^{T} {e^{-st}F(t)} \: d{t}
    \] where $T$ is the period of $F(t)$.
\end{theorem}

\underline{\textbf{Proof: }} \\
Let $F(t)$ has period $T$. Then $F(t) = F(t+T)$ for all $t$. Then
\begin{align*}
    \Lap \{ F(t) \} &= \int_{0}^{\infty} {e^{-st}F(t)} \: d{t} \\
    &= \int_{0}^{T} {e^{-st}F(t)} \: d{t} + \int_{T}^{2T} {e^{-st}F(t)} \: d{t} + \int_{2T}^{3T} {e^{-st}F(t)} \: d{t} + \cdots \\
    &= \int_{0}^{T} {e^{-st}F(t)} \: d{t} + \int_{0}^{T} {e^{-s(t+T)}F(t+T)} \: d{t} + \int_{0}^{T} {e^{-s(t+2T)}F(t+2T)} \: d{t} + \cdots \\
    &= \int_{0}^{T} {e^{-st}F(t)} \: d{t} + e^{-sT} \int_{0}^{T} {e^{-st}F(t)} \: d{t} + e^{-2sT} \int_{0}^{T} {e^{-st}F(t)} \: d{t} + \cdots \\
    &= \left[ 1 + e^{-sT} + e^{-2sT} + \cdots \right] \int_{0}^{T} {e^{-st}F(t)} \: d{t} \\
    &= \frac{1}{1-e^{-sT}} \int_{0}^{T} {e^{-st}F(t)} \: d{t} \qed
\end{align*}
\begin{note}{}
    Sum of an infinite series $1 + r + r^2 + \cdots = \dfrac{1}{1-r}$ for $|r|<1$.
\end{note}

\begin{example}{
        Find $\Lap \{ F(t) \}$ for \[
            F(t) = 
            \begin{cases}
                \sin{t} & \text{for } 0 < t < \pi \\
                0 & \text{for } \pi < t < 2\pi
            \end{cases}
        \]
    }{}
    \begin{align*}
        \Lap \{ F(t) \} &= \frac{1}{1 - e^{-2\pi s}} \int_{0}^{2\pi} { e^{st} \sin{t} } \: d{t} \\
        &= \frac{1}{1-e^{-2\pi s}} \int_{0}^{\pi} {e^{-st} \sin{t}} \: d{t}
    \end{align*}
\end{example}

\begin{example}{Find $\Lap \{ F(t) \}$ for \[
    F(t) =
    \begin{cases}
        t & \text{for } 0 < t < 1 \\
        0 & \text{for } 1 < t < 2
    \end{cases}
\]}{}
    \begin{align*}
        \Lap \{ F(t) \} &= \frac{1}{1-e^{-2s}} \int_{0}^{1} {t} \: d{t} \\
        &= \frac{t^2 \bigg|_0^1}{2 - 2e^{-2s}} \\
        &= \frac{1}{2 - 2e^{-2s}}
    \end{align*}
\end{example}


%%%%% Laplace Transform of Derivatives %%%%%
\subsection{Laplace Transform of Derivatives}
\begin{theorem}{Laplace transform of derivatives}{}
    \\If \[
        \Lap \{ F(t) \} = f(s)
    \] then \[
        \Lap \{ F'(t) \} = sf(s) - F(0)
    \] \[
        \Lap \{ F''(t) \} = s^2F(s) - sF(0) - F'(0)
    \] \[
        \Lap \{ F'''(t) = s^3f(s) - s^2F(0) - sF'(0) - F''(0) \}
    \] \[
        \boxed{ \Lap \{ F^{(n)}(t) \} = s^nf(s) - s^{n-1}F(0) - s^{n-2}F'(0) - \cdots - F^{(n-1)}(0) }
    \] \[
        \boxed{ \Lap \{ F^{(n)}(t) \} = s^nf(s) - \sum_{i=n-1}^{0}\sum_{j=0}^{n-1} s^i F^{(j)}(0) }
    \]
\end{theorem}

\underline{\textbf{Proof: }} \\
Using integration by parts, we have
\begin{align*}
    \Lap \{ F'(t) \} &= \int_{0}^{\infty} {e^{-st}F'(t)} \: d{t} \\
    &= e^{-st}F(t)\bigg|_{0}^{\infty} + s \int_{0}^{\infty} {e^{-st}F(t)} \: d{t}  \\
    &= - F(0) + s \int_{0}^{\infty} {e^{-st}F(t)} \: d{t} \\
    &= sf(s) - F(0) \qed
\end{align*}
Similarly,
\begin{align*}
    \Lap \{ F''(t) \} &= s \Lap \{ F'(t) \} - F'(0) \\
    &= s \left[ sf(s) - F(0) \right] - F'(0) \\
    &= s^2f(s) - sF(0) - F'(0)
\end{align*}
Thus using mathematical induction, we get
\begin{align*}
    \Lap \{ F^{(n)}(t) \} &= s \Lap \{ F^{(n-1)}(t) \} - F^{(n-1)}(0) \\
    &= s \left[ s \Lap \{ F^{(n-2)}(t) \} - F^{(n-2)}(0) \right] - F^{(n-1)}(0) \\
    &= s^2 \Lap \{ F^{(n-2)}(t) \} - sF^{(n-2)}(0) - F^{(n-1)}(0) \\
    &= s^2 \left[ s \Lap \{ F^{(n-3)}(t) \} - F^{(n-3)}(0) \right] - sF^{(n-2)}(0) - F^{(n-1)}(0) \\
    &= s^3 \Lap \{ F^{(n-3)}(t) \} - s^2F^{(n-3)}(0) - sF^{(n-2)}(0) - F^{(n-1)}(0) \\
    &= \cdots \\
    &= s^n \Lap \{ F(t) \} - s^{n-1}F(0) - s^{n-2}F'(0) - \cdots - F^{(n-1)}(0) \\
    &= s^nf(s) - \sum_{i=n-1}^{0}\sum_{j=0}^{n-1} s^i F^{(j)}(0) \qed
\end{align*}

%%%%%%%%%%%%%%%%%%%%%%%%%%%%%%%
%  Inverse Laplace Transform  %
%%%%%%%%%%%%%%%%%%%%%%%%%%%%%%%
\newpage
\section{Inverse Laplace Transform}

\subsection{Definition and Existence}

\begin{definition}{Inverse Laplace Transform}{}
    If the Laplace Transform of a function $F(t)$ is $f(s)$, i.e
    \[
        \Lap \{ F(t) \} = f(s)
    \] 
    then the Inverse Laplace Transform is defined as
    \[
        \Lapinv \{ f(s) \} = F(t)
    \]
\end{definition}

\subsection{Some Functions and their Inverse Laplace Transforms}

\begin{table}[htpb]

    \centering
    \begin{tabular}{c|c}
        \hline
        $f(s)$ & $\Lap^{-1} \{ f(s) \} = F(t)$ \\
        \hline\hline
        $\dfrac{1}{s}$ & $1$ \\
        \hline
        $\dfrac{1}{s^2}$ & $t$ \\
        \hline
        $\dfrac{1}{s-a}$ & $e^{at}$ \\
        \hline
        $\dfrac{1}{s^2+a^2}$ & $\dfrac{\sin{at}}{a}$ \\
        \hline
        $\dfrac{s}{s^2+a^2}$ & $\cos{at}$ \\
        \hline
    \end{tabular}
    \begin{tabular}{c | c}
        \hline
        $f(s)$ & $\Lap^{-1} \{ f(s) \} = F(t)$ \\
        \hline\hline
        $\dfrac{n}{s}$ & $n$ \\
        \hline
        $\dfrac{1}{s^n}$ & $\dfrac{t^{n-1}}{(n-1)!}$ \\
        \hline
        $\dfrac{1}{s+a}$ & $e^{-at}$ \\
        \hline
        $\dfrac{1}{s^2-a^2}$ & $\dfrac{\sinh{at}}{a}$ \\
        \hline
        $\dfrac{s}{s^2-a^2}$ & $\cosh{at}$ \\
        \hline
    \end{tabular}

    \caption{Functions and their Inverse Laplace Transform}
    \label{InvLapTable}
\end{table}

%%%%%%%%%%%%%%%%%%%%%%%%%%%%%%%%%%%%%%%%%%%%%%%%%%%
%  Basic Properties of Inverse Laplace Transform  %
%%%%%%%%%%%%%%%%%%%%%%%%%%%%%%%%%%%%%%%%%%%%%%%%%%%

\section{Basic Properties of Inverse Laplace Transform}

\subsection{Linearity Property}

\begin{theorem}{Linearity Property}{}
    \\If \[
        \Lapinv \{ f(s) \} = F(t) \quad \text{ and } \quad
        \Lapinv \{ g(s) \} = G(t)
    \] Then, \[
        \Lapinv \{ c_1f(s) + c_2g(s) \} = c_1 \Lapinv \{ f(s) \} + c_2 \Lapinv \{ g(s) \}
    \]
\end{theorem}

\begin{example}{Evaluate $\Lapinv \left\{ \frac{4}{s-2} - \frac{3s}{s^2+16} + \frac{s}{s^2+4} \right\}$}{}
    \begin{align*}
        \Lapinv \left\{ \frac{4}{s-2} - \frac{3s}{s^2+16} + \frac{s}{s^2+4} \right\} &= 4\Lapinv \left\{ \frac{1}{s-2} \right\} - 3\Lapinv \left\{ \frac{s}{s^2+4^2} \right\} + 5\Lapinv \left\{ \frac{1}{s^2+2^2} \right\} \\
        &= 4e^{2t} - 3\cos{4t} + \frac{5}{2}\sin{2t}
    \end{align*}
\end{example}


\subsection{First Translation or Shifting Property}

\begin{theorem}{First Translation or Shifting Property}{}
    \\If \[
        \Lapinv \{ f(s) \} = F(t)
    \] then \[
        \Lapinv \{ f(s-a) \} = e^{at}F(t)
    \]
\end{theorem}

\begin{example}{Evaluate $\displaystyle \Lapinv \left\{ \frac{6s-4}{s^2-4s+20} \right\}$}{}
    \begin{align*}
        \Lapinv \left\{ \frac{6s-4}{s^2-4s+20} \right\} &= \Lapinv \left\{ \frac{6(s-2)+8}{(s-2)^2+16} \right\} \\
        &= 6\Lapinv \left\{ \frac{(s-2)}{(s-2)^2+16} \right\} + 2\Lapinv \left\{ \frac{4}{(s-2)^2+16} \right\} \\
        &= 6e^{2t}\cos{4t} + 2e^{2t}\sin{4t}
    \end{align*}
\end{example}


\subsection{Second Translation or Shifting Property}

\begin{theorem}{Second Translation or Shifting Property}{}
    \\If \[
        \Lapinv \{ f(s) \} = F(t)
    \] then \[
        \Lapinv \{ e^{-as}f(s) \} = 
        \begin{cases}
            F(t-a) & \text{for } t > a \\
            0 & \text{for } t < a
        \end{cases}
    \]
\end{theorem}

\begin{example}{Evaluate $\displaystyle \frac{e^{-5t}}{(s-2)^4}$}{}
    Here \[
        \Lapinv \left\{ \frac{1}{(s-2)^4} \right\} = e^{2t}\frac{t^3}{3!} = \frac{e^{2t}}{6} t^3
    \] \[
        \therefore \Lapinv \left\{ \frac{e^{-5s}}{(s-2)^4} \right\} = \frac{e^{2(t-5)}}{6} (t-5)^3 \text{, when } t > 5
    \] \[
        \Lapinv \left\{ \frac{e^{-5s}}{(s-2)^4} \right\} = 0 \text{, when } t < 5
    \]
\end{example}


\subsection{Inverse Laplace Transform of Derivatives}

\begin{theorem}{Inverse Laplace Transform of Derivatives}{}
    \\If \[
        \Lapinv \{ f(s) \} = F(t)
    \] then \[
        \Lapinv \{ f^(n)(s) \} = \Lapinv \left\{ \dv[n]{s} f(s) \right\} = (-1)^n t^n F(t)
    \]
\end{theorem}

\begin{example}{Evaluate $\displaystyle \Lapinv \left\{ \ln{\left( 1+\frac{1}{s^2} \right) } \right\}$}{}
    Here
    \begin{align*}
        f(s) &= \ln{\left( 1 + \frac{1}{s^2} \right) } = \Lap \{ F(t) \} \\
        f'(s) &= \frac{-\frac{2}{s^3}}{1+\frac{1}{s^2}} \\
        &= -2 \left\{ \frac{1}{s(s^2+1)} \right\} \\
        &= -2 \left( \frac{1}{s - \frac{s}{s^2+1}} \right)
    \end{align*}
    \begin{align*}
        \Lapinv \{ f'(s) \} &= -2 (1-\cos{t}) \\
        t F(t) &= -2 (1-\cos{t}) \\
        F(t) &= \frac{2(1-\cos{t})}{t} \\
        \therefore \Lapinv \left\{ \ln{\left( 1 + \frac{1}{s^2} \right)} \right\} &= \frac{2(1-\cos{t})}{t}
    \end{align*}
\end{example}


\subsection{Inverse Laplace Transform of Integrals}

\begin{theorem}{Inverse Laplace Transform of Integrals}{}
    \\If \[
        \Lapinv \{ f(s) \} = F(t)
    \] then \[
        \Lapinv \left\{ \int_{s}^{\infty} {f(u)} \: d{u} \right\} = \frac{F(t)}{t}
    \]
\end{theorem}

\begin{example}{Find $\displaystyle \Lapinv \left\{ \int_{s}^{\infty} {\left( \frac{1}{u} - \frac{1}{u+1} \right)} \: d{u} \right\}$}{}
    Let \[
        f(s) = \frac{1}{s} - \frac{1}{s+1}
    \]
    \begin{align*}
        \Lapinv \{ f(s) \} &= \Lapinv \left\{ \frac{1}{s - \frac{1}{s+1}} \right\} \\
        &= 1 - e^{-t} = F(t) \\
        \therefore \Lapinv \left\{ \int_{s}^{\infty} {f(u)} \: d{u} \right\} &= \frac{F(t)}{t} \\
        &= \frac{1-e^{-t}}{t}
    \end{align*}
\end{example}


\subsection{Division by s}

\begin{theorem}{Division by $s$}{}
    \\If \[
        \Lapinv \{ f(s) \} = F(t)
    \] then \[
        \Lapinv \left\{ \frac{f(s)}{s} \right\} = \int_{0}^{t} {F(u)} \: d{u}
    \]
\end{theorem}

\begin{example}{Evaluate $\displaystyle \Lapinv \left\{ \frac{1}{s^3(s^2+1)} \right\}$}{}
    Let \[
        f(s) = \frac{1}{s^2+1}
    \]
    \begin{align*}
        \therefore \Lapinv \{ f(s) \} &= \sin{t} \\
        \therefore \Lapinv \left\{ \frac{1}{s(s^2+1)} \right\} &= \int_{0}^{t} {\sin{u}} \: d{u} &&= 1-\cos{t}\\
        \therefore \Lapinv \left\{ \frac{1}{s^2(s^2+1)} \right\} &= \int_{0}^{t} {(1-\cos{u})} \: d{u} &&= t - \sin{t} \\
        \therefore \Lapinv \left\{ \frac{1}{s^3(s^2+1)} \right\} &= \int_{0}^{t} {(u-\sin{u})} \: d{u} &&= \frac{t^2}{2} + \cos{t} - 1
    \end{align*}\\~\\
    
    \underline{\textbf{Alternative approach: }}
    \\Let \[
        f(s) = \frac{1}{s^4 + s^2} = \frac{1}{s^2} - \frac{1}{s^2+1}
    \]
    \begin{align*}
        \Lapinv \left\{ \frac{1}{s^2(s^2+1)} \right\} &= \Lapinv \left\{ \frac{1}{s^2} - \frac{1}{s^2+1} \right\} \\
        &= t - \sin{t} \\
        &= F(t) \\
        \therefore \Lapinv \left\{ \frac{1}{s^3(s^2+1)} \right\} &= \int_{0}^{t} {F(u)} \: d{u} \\
        &= \int_{0}^{t} {(u-\sin{u})} \: d{u} \\
        &= \frac{t^2}{2} + \cos{t} - 1
    \end{align*}
\end{example}


\subsection{Inverse Laplace Transform by Partial Fraction}

\begin{example}{Find $\displaystyle \Lapinv \left\{ \frac{5^2 - 15s - 11}{(s+1)(s-2)^2} \right\}$}{}
    \[
        \frac{5^2 - 15s - 11}{(s+1)(s-2)^2} = \frac{-\frac{1}{3}}{ss+1} - \frac{7}{(s-2)^3} + \frac{4}{(s-2)^2} + \frac{\frac{1}{3}}{s-2}
    \] \[
        \therefore \Lapinv \left\{ \frac{5^2 - 15s - 11}{(s+1)(s-2)^2} \right\} = -\frac{1}{e}e^{-t} - 7 \frac{t^2}{2} e^{2t} + 4te^{2t} + \frac{1}{3}e^{2t}
    \]
\end{example}


\subsection{The Convolution Theorem}

\begin{theorem}{The Convolution Theorem}{}
    \\If \[
        \Lapinv \{ f(s) \} = F(t) \quad \text{ and } \quad \Lapinv \{ g(s) \} = G(t)
    \] then \[
        \Lapinv \{ f(s)g(s) \} = \int_{0}^{t} {F(u)G(t-u)} \: d{u} = F * G
    \] \[
        G*F = \int_{0}^{t} {G(u)F(t-u)} \: d{u} = F*G
    \]
\end{theorem}

\begin{example}{Find $\displaystyle \Lapinv \left\{ \frac{1}{s^2(s^2+1)^2} \right\}$}{}
    Here, \[
        \Lapinv \left\{ \frac{1}{s^2} \right\} = t = F(t)
    \] \[
        \Lapinv \left\{ \frac{1}{(s+1)^2} \right\} = e^{-t}t = G(t)
    \]
    \begin{align*}
        \therefore \Lapinv \left\{ \frac{1}{s^2(s+1)^2} \right\} &= G*F \\
        &= \int_{0}^{t} {ue^{-u} (t-u)} \: d{u} \\
        &= \int_{0}^{t} {(ut - u^2)e^{-u}} \: d{u}  \\
        &= -(ut-u^2) \cdot e^{-u} \bigg|_{0}^{t} + \int_{0}^{t} {(t - 2u)e^{-u}} \: d{u}
    \end{align*}
\end{example}








\end{document}
