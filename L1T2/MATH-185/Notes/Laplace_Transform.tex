\documentclass[12pt]{article}
\usepackage[a4paper, margin=0.75in]{geometry}
\usepackage[document]{ragged2e}
\usepackage{graphicx}
\graphicspath{ {./images/} }
\usepackage{enumerate}
\usepackage{framed}
\usepackage{amsmath,amsfonts,amsthm,thmtools,amssymb,mathtools,commath}
\usepackage{physics}
\usepackage{tikz}
\usetikzlibrary{mindmap}
\usepackage{caption}
\usepackage{xcolor}
\usepackage[most]{tcolorbox}
\usepackage{cleveref}


%%%%%%%%%%%%%%%%
%  Definition  %
%%%%%%%%%%%%%%%%
\tcbuselibrary{theorems,skins,hooks}
\newtcbtheorem[number within=subsection]{definition}{Definition}%
{
    % theorem style=definition,
    enhanced,
	before skip=2mm,after skip=2mm, colback=cyan!5,colframe=cyan!80!black,boxrule=0.5mm,
	attach boxed title to top left={xshift=1cm,yshift*=1mm-\tcboxedtitleheight},
	boxed title style={frame code={
					\path[fill=cyan]
					([yshift=-1mm,xshift=-1mm]frame.north west)
					arc[start angle=0,end angle=180,radius=1mm]
					([yshift=-1mm,xshift=1mm]frame.north east)
					arc[start angle=180,end angle=0,radius=1mm];
					\path[left color=cyan!30!black,right color=cyan!30!black,
						middle color=cyan!50!black]
					([xshift=-2mm]frame.north west) -- ([xshift=2mm]frame.north east)
					[rounded corners=1mm]-- ([xshift=1mm,yshift=-1mm]frame.north east)
					-- (frame.south east) -- (frame.south west)
					-- ([xshift=-1mm,yshift=-1mm]frame.north west)
					[sharp corners]-- cycle;
				},interior engine=empty,
		},
	fonttitle=\bfseries,
	title={#2},#1
}{def}


%%%%%%%%%%%%%
%  Theorem  %
%%%%%%%%%%%%%
\tcbuselibrary{theorems,skins,hooks}
\newtcbtheorem[use counter from=definition]{theorem}{Theorem}%
{
    theorem style=plain,
    enhanced,
    colframe=green,
    boxrule=1pt,
    titlerule=0mm,
    toptitle=1mm,
    bottomtitle=1mm,
    fonttitle=\bfseries,
    fontupper=\mdseries\itshape,
    coltitle=green!30!black,
    colbacktitle=cyan!15!white,
    colback=green!10,
    description font=\bfseries\sffamily
}{thrm}


%%%%%%%%%%%%%%
% Corollary  %
%%%%%%%%%%%%%%
 \tcbuselibrary{theorems,skins}
 \newtcbtheorem[use counter from=theorem]{corollary}{Corollary}%
 {
    theorem style=plain,
    enhanced,
    colframe=green,
    frame hidden,
    titlerule=0mm,
    toptitle=1mm,
    bottomtitle=1mm,
    fonttitle=\bfseries,
    fontupper=\mdseries\itshape,
    coltitle=green!30!black,
    colbacktitle=cyan!15!white,
    colback=green!10,
    description font=\bfseries\sffamily
 }{corl}


%%%%%%%%%%%%%
%  Example  %
%%%%%%%%%%%%%
\tcbuselibrary{theorems,skins,hooks}
\newtcbtheorem[number within=section]{example}{Example}%
{
	enhanced,
	breakable,
	colback = gray!5,
	frame hidden,
	boxrule = 0sp,
	borderline west = {2pt}{0pt}{gray},
	sharp corners,
	detach title,
	before upper = \tcbtitle\par\smallskip,
    coltitle=gray!70!black,
	fonttitle = \bfseries\sffamily,
	description font = \mdseries\bfseries
}
{xmp}


%%%%%%%%%%%%%%
%  Exercise  %
%%%%%%%%%%%%%%
\tcbuselibrary{theorems,skins,hooks}
\newtcbtheorem[number within=section]{exercise}{Exercise}%
{
    enhanced,
    breakable,
    colback=black!5,
    colframe=black!30,
    left=0.5em,
    before skip=10pt,
    after skip=10pt,
    boxrule=0pt,
    boxsep=0pt,
    arc=0pt,
    outer arc=0pt,
    borderline west={3pt}{0pt}{black!30},
}{exc}

%%%%%%%%%%
%  Note  %
%%%%%%%%%%
\usetikzlibrary{arrows,calc,shadows.blur}
\tcbuselibrary{skins}
\newtcolorbox{note}[1][]{%
	enhanced jigsaw,
	colback=gray!20!white,%
	colframe=gray!80!black,
	size=small,
	boxrule=1pt,
	title=\textbf{Note:-},
	halign title=flush center,
	coltitle=black,
	breakable,
	drop shadow=black!50!white,
	attach boxed title to top left={xshift=1cm,yshift=-\tcboxedtitleheight/2,yshifttext=-\tcboxedtitleheight/2},
	minipage boxed title=1.5cm,
	boxed title style={%
			colback=white,
			size=fbox,
			boxrule=1pt,
			boxsep=2pt,
			underlay={%
					\coordinate (dotA) at ($(interior.west) + (-0.5pt,0)$);
					\coordinate (dotB) at ($(interior.east) + (0.5pt,0)$);
					\begin{scope}
						\clip (interior.north west) rectangle ([xshift=3ex]interior.east);
						\filldraw [white, blur shadow={shadow opacity=60, shadow yshift=-.75ex}, rounded corners=2pt] (interior.north west) rectangle (interior.south east);
					\end{scope}
					\begin{scope}[gray!80!black]
						\fill (dotA) circle (2pt);
						\fill (dotB) circle (2pt);
					\end{scope}
				},
		},
	#1,
}

\usepackage[scr]{rsfso}
\usepackage{multicol}
\renewcommand{\arraystretch}{2.5}

\newcommand{\Lap}{\mathscr{L}}

\title{
    \textbf{Laplace Transform}
}

\author{
    Turja Roy\\
    ID: 2108052
}
\date{}

\begin{document}
\maketitle
\tableofcontents
\newpage

%%%%%%%%%%%%%%%%%%%%%%%%%%%%%%%%%%%%%%%%%%%%%%%%%
%  Definition, Existence, and Basic Properties  %
%%%%%%%%%%%%%%%%%%%%%%%%%%%%%%%%%%%%%%%%%%%%%%%%%

\section{Definition, Existence, and Basic Properties of the Laplace Transform}

%%%%%%%%%%%%%%%%%%%%%%%%%%%%%%
%  Definition and Existence  %
%%%%%%%%%%%%%%%%%%%%%%%%%%%%%%

\subsection{Definition and Existence}

\begin{definition}{Laplace Transform}{}
    Let $F$ be a real-valued function of the real variable $t$, defined for $t>0$. Let $s$ be a variable that we shall assume to be real, and consider the function $f$ defined by
    \begin{equation} \label{eq1}
        f(s) = \int_{0}^{\infty} {e^{-st} F(t)} \: d{t} 
    \end{equation}
    for all values of $s$ for which this integral exists. The function $f$ defined by the integral \eqref{eq1} is called the Laplace Transform of the function $F$. We shall denote the Laplace transform of $F$ by $\Lap\{F(t)\}$.\\
    Thus the Laplace transform of a function $f$ is given by
    \begin{equation} \label{eq2}
        \Lap\{ F(t) \} = f(s) = \int_{0}^{\infty} {e^{-st} F(t)} \: d{t}  = \lim_{R \to \infty} \int_{0}^{R} {e^{-st} F(t)} \: d{t} 
    \end{equation}
\end{definition}

Some ways to write Laplace transforms:\\
\[ \Lap{F(t)} = f(s) = \int_{0}^{\infty} { e^{-st} F(t) } \: d{t} \]
\[ \Lap{G(t)} = g(s) \]
\[ \Lap{u(t)} = \tilde{u}(s) \]

\vspace{20pt}
\begin{table}[htpb]
    \centering

    \begin{tabular}{c | c}
        \hline
        $F(t)$ &  $\Lap\{ F(t) \} = f(s)$ \\
        \hline\hline
        $1$ & $\dfrac{1}{s}$ \\\hline
        $t$ & $\dfrac{1}{s^2}$ \\\hline
        $e^{at}$ & $\dfrac{1}{s-a}$ \\\hline
        $\sin{at}$ & $\dfrac{a}{s^2+a^2}$ \\\hline
        $\cos{at}$ & $\dfrac{s}{s^2+a^2}$ \\
        \hline
    \end{tabular}
    \begin{tabular}{c | c}
        \hline
        $F(t)$ &  $\Lap\{ F(t) \} = f(s)$ \\
        \hline\hline
        $n$ & $\dfrac{n}{s}$ \\\hline
        $t^n$ &  $\dfrac{n!}{s^{n+1}}$ \\\hline
        $e^{-at}$ & $\dfrac{1}{s+a}$ \\\hline
        $\sinh{at}$ & $\dfrac{a}{s^2-a^2}$ \\\hline
        $\cosh{at}$ &  $\dfrac{s}{s^2-a^2}$ \\
        \hline
    \end{tabular}

    \caption{Functions and their Laplace Transform}
    \label{LapTable}
\end{table}

\newpage
\begin{multicols}{2}
    [
    \underline{\textbf{Proofs :}}\\
    ]
    Let $F(t) = n$, for $t>0$\\
    Then\\
    \begin{align*}
        \Lap \{ n \} &= \int_{0}^{\infty} { e^{-st} \cdot n } \: d{t} \\
                     &= \eval{ n\frac{-e^{st}}{s} }_{0}^{\infty} \\
                     &= \frac{n}{s}\qed
    \end{align*}\\
    \columnbreak

    Let $F(t) = t$, for $t>0$\\
    Then\\
    \begin{align*}
        \Lap \{ t \} &= \int_{0}^{\infty} { e^{-st} \cdot t } \: d{t} \\
                     &= -t \frac{e^{-st}}{s} + \int_{0}^{\infty} {\frac{e^{-st}}{s}} \: d{t} \\
                     &= -e^{-st}\frac{t}{s}\bigg|_{0}^{\infty} - e^{-st}\frac{1}{s^2}\bigg|_{0}^{\infty} \\
                     &= \frac{1}{s^2}\qed
    \end{align*}
\end{multicols}

\vspace{20pt}
Let $F(t) = t^{n}$, for $t>0$ \\
Then\\
\begin{align*}
    \Lap \{ t^{n} \} &= \int_{0}^{\infty} {e^{-s t}t^n} \: d{t} \\
                     &= -t^n \frac{e^{s t}}{s} + \int_{0}^{\infty} { nt^{n-1} \frac{e^{-s t}}{s} } \: d{t} \\
                     &= \eval{ -nt^{n-1} \frac{e^{-s t}}{s^2} }_{0}^{\infty} + \int_{0}^{\infty} { n(n-1)t^{n-2} \frac{e^{-s t}}{s^2} } \: d{t} \\
                     &= \eval{ -n(n-1)t^{n-2} \left( \frac{e^{-s t}}{s^3} \right) }_{0}^{\infty} + \int_{0}^{\infty} { n(n-1)(n-2)t^{n-3} \frac{e^{-s t}}{s^3} } \: d{t}  \\
                     &= \cdots \\
                     &= \eval{n!t^{n-n}\frac{e^{-s t}}{s^{n+1}}}_{0}^{\infty} + \int_{0}^{\infty} { n(n-1)\cdots(n-n) \frac{e^{-s t}}{s^{n+1}} } \: d{t} \\
                     &= \frac{n!}{s^{n+1}}\qed
\end{align*}\\~\\

\begin{multicols}{2}
    Let $F(t) = e^{at}$, for $t>0$\\
    Then\\
    \begin{align*}
        \Lap \{ e^{at} \} &= \int_{0}^{\infty} { e^{-s t} e^{at} } \: d{t} \\
                          &= \int_{0}^{\infty} { e^{(a-s)t} } \: d{t} \\
                          &= \eval{ \frac{e^{(a-s)t}}{a-s} }_{0}^{\infty} \\
                          &= \frac{1}{s-a}\qed
    \end{align*}
    \columnbreak
    
    Let $F(t)=e^{-at}$, for $t>0$ \\
    Then\\
    \begin{align*}
        \Lap \{ e^{-at} \} &= \int_{0}^{\infty} {e^{-s t} e^{-at}} \: d{t} \\
        &= \int_{0}^{\infty} {e^{-(a+s)t}} \: d{x} \\
        &= \eval{ \frac{e^{-(a+s)t}}{s+a} }_{0}^{\infty} \\
        &= \frac{1}{s+a}\qed
    \end{align*}
\end{multicols}

\newpage
\begin{multicols}{2}
    Let $F(t) = \sin{at}$, for $t>0$ \\
    Then\\
    \begin{align*}
        \Lap \{ \sin{at} \} &= \int_{0}^{\infty} { e^{-s t \sin{at}} } \: d{t} \\
        &= - \eval{ \frac{e^{-s t}}{s^2+a^2} \left( s\sin{at} + a\cos{at} \right) }_{0}^{\infty} \\
        &= \frac{a}{s^2+a^2}\qed
    \end{align*}
    \columnbreak
    
    Let $F(t) = \cos{at}$, for $t>0$ \\
    Then\\
    \begin{align*}
        \Lap \{ \cos{at} \} &= \int_{0}^{\infty} { e^{-s t}\cos{at} } \: d{t} \\
        &= \eval{ \frac{e^{-s t}}{s^2+a^2} \left( -s\cos{at} + a\sin{at} \right) }_{0}^{\infty} \\
        &= \frac{s}{s^2+a^2}\qed
    \end{align*}
\end{multicols}


\vspace{30pt}
\begin{theorem}{}{}
    \textbf{Hypothesis: } Let $F$ be a real function that has the following properties:
    \begin{enumerate}
        \item $F$ is a piecewise continuous in every finite closed interval $0 \le t \le a \: (b>0)$.
        \item $F$ is of exponential order, i.e, there exists $\alpha$, $M>0$, and $t_0>0$ such that
    \end{enumerate}
    \[ e^{-\alpha t}|F(t)| < M \text{ for } t>t_0 \]
    
    \textbf{Conclusion: } The Laplace transform of $F$ exists for $s>\alpha$.
    \[
        \Lap \{ F(t) \} = \int_{0}^{\infty} {e^{-s t}F(t)} \: d{t}
    \] 
\end{theorem}


%%%%%%%%%%%%%%%%%%%%%%%%%%%%%%%%%%%%%%%%%%%%%%%
%  Basic Properties of the Laplace Transform  %
%%%%%%%%%%%%%%%%%%%%%%%%%%%%%%%%%%%%%%%%%%%%%%%
\vspace{30pt}
\subsection{Basic Properties of the Laplace Transform}

\begin{theorem}{The Linear Property}{}
    \\Let $F_1$ and $F_2$ be functions whose Laplace transform exist, and let $c_1$ and $c_2$ be constants. Then
    \[
        \Lap \{ c_1F_1(t) + c_2F_2(t) \} = c_1 \Lap \{ F_1(t) \} + c_2 \Lap \{ F_2(t) \}
    \]
\end{theorem}

\begin{theorem}{}{}
    \textbf{Hypothesis: }
    \begin{enumerate}
        \item Let $F$ be a real function that is continuous for $t \ge 0$ and of exponential order $e^{\alpha t}$.
        \item Let $F'$ be piecewise continuous in ever finite closed interval $0 \le t \le a$.
    \end{enumerate}

    \textbf{Conclusion: } Then $\Lap \{ F' \}$ exists for $s>\alpha$ ; and
    \[
        \Lap \{ F'(t) \} = s \Lap \{ F(t) \} - F(o)
    \]
\end{theorem}
















\end{document}
