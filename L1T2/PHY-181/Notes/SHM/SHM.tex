\documentclass[12pt]{article}
\usepackage[a4paper, margin=0.75in]{geometry}
\usepackage[document]{ragged2e}
\usepackage{graphicx}
\graphicspath{ {./images/} }
% \usepackage[tmargin=2cm,rmargin=1in,lmargin=1in,margin=0.85in,bmargin=2cm,footskip=.2in]{geometry}
\usepackage{enumerate}
\usepackage{framed}
\usepackage{amsmath,amsfonts,amsthm,thmtools,amssymb,mathtools,commath}
\usepackage{tikz}
\usepackage{xcolor}
\usepackage[most]{tcolorbox}


\tcbuselibrary{theorems}
\newtcbtheorem[number within=section]{example}{Example}%
{
    colback=green!5,
    frame hidden,
    detach title,
    before upper = \tcbtitle\par\smallskip,
    coltitle=green!35!black,
    % colframe=green!35!black,
    fonttitle=\bfseries\sffamily
    % description font=\mdseries
}{th}

\usepackage{makecell}
\allowdisplaybreaks

\title{
    \textbf{Simple Harmonic Motion}
}

\author{
    Turja Roy\\
    ID: 2108052
}
\date{}

\begin{document}
\maketitle
\tableofcontents
\newpage

\section{Oscillation and Vibration}
\subsection{Oscillation}
\begin{itemize}
    \item Oscillation is the repetitive variation, typically in time, of some measure about a central value (often a point of equilibrium) or between two or more different states.
    \item The term vibration is precisely used to describe mechanical oscillation.
    \item Familiar examples of oscillation include a swinging pendulum and alternating current.
\end{itemize}

\subsection{Vibration}
\begin{itemize}
    \item Vibration is a mechanical phenomenon whereby oscillations occur about an equilibrium point.
    \item The word comes from Latin vibrationem ("shaking, brandishing").
    \item The oscillations may be periodic, such as the motion of a pendulum—or random, such as the movement of a tire on a gravel road.
\end{itemize}

\subsection{Differences between Oscillation and Vibration}
\begin{itemize}
    \item Oscillation is the definite displacement of a body in terms of distance or time, whereas vibration is the movement brought about in a body due to oscillation.
    \item Oscillation takes place in physical, biological systems, and often in our society, but vibrations is associated with mechanical systems only.
    \item Oscillation is about a single body, whereas vibration is the result of collective oscillation of atoms in the body.
    \item All vibrations are oscillations, but not all oscillations are vibrations.
\end{itemize}

\section{Simple Harmonic Motion}
\subsection{Definition}

\begin{definition}{Simple Harmonic Motion}{}
    Simple harmonic motion is a type of periodic motion where the restoring force is directly proportional to the displacement and acts in the direction opposite to that of displacement.
\end{definition}

A particle is said to execute SHM when it will
\begin{enumerate}[(a)]
    \item Trace and retrace the same path over and over again.
    \item Change direction at a regular interval of time.
    \item Move along a straight line.
    \item Have acceleration proportional to its displacement from the mean position.
\end{enumerate}

A particle which satisfies the condition (a) only is said to execute \textbf{periodic motion}. A particle which satisfies condition (a) and (b) is said to execute \textbf{vibratory motion}. \\~\\

Let $P$ be a particle moving on the circumference of a circle of radius $r$ with a uniform velocity $v$. Let angular velocity be $\omega = v/r$.

\begin{figure}[htpb]
    \centering
    \includegraphics[width=0.5\textwidth]{1.png}
\end{figure}

\vspace{20pt}
Displacement of the particle from the mean position is given by $\displaystyle y = r \sin{\omega t}$\\
So, velocity of the particle is given by $\displaystyle v = \frac{dy}{dt} = \omega r \cos{\omega t}$\\
And acceleration of the particle is given by $\displaystyle a = \frac{dv}{dt} = -\omega^2 r \sin{\omega t} = -\omega^2 y$

\vspace{20pt}
\begin{center}
    \begin{tabular}{c | c | c | c | c}
        Angle & \makecell{Position of \\ vibrating particle} & \makecell{Displacement \\ $y = r \sin{\omega t}$} & \makecell{Velocity \\ $\displaystyle \dv{y}{t} = \omega r \cos{\omega t}$} & \makecell{Acceleration \\ $-\omega^2 r \sin{\omega t} = -\omega^2 y$} \\\hline\hline
        0 & O & 0 & $\omega r$ & 0 \\\hline
        $\pi/2$ & X & $r$ & 0 & $-\omega^2 a$ \\\hline
        $\pi$ & O & 0 & $-\omega r$ & 0 \\\hline
        $3\pi/2$ & Y & $-r$ & 0 & $\omega^2 r$ \\\hline
        $2\pi$ & O & 0 & $\omega r$ & 0
    \end{tabular}
\end{center}

\subsection{Differential Equation of SHM}
Let $y$ be the displacement of the particle from the mean position at time $t$, $r$ be the amplitude, and $\alpha$ be the epoch of the vibrating particle \\
\begin{equation}
    y = r \sin{(\omega t + \alpha)}
\end{equation}
\begin{equation}
    \dv{y}{t} = r \omega \cos{(\omega t + \alpha)}
\end{equation}
\begin{equation}
    \dv[2]{y}{t} = -r \omega^2 \sin{(\omega t + \alpha)}
\end{equation}

Hence the differential equation of SHM is
\begin{equation}
    \dv[2]{y}{t} + \omega^2 y = 0
\end{equation}

\subsection{Solution of the Differential Equation of SHM}
\begin{equation}
    \dv[2]{y}{t} + \omega^2 y = 0
\end{equation}

Here, \[
    \dv[2]{y}{t} = \dv{v}{t} = \dv{v}{y} \dv{y}{t} = v \dv{v}{y}
\]
\begin{align*}
    v \: d{v} + \omega^2y \: d{y} &= 0 \\
    \int{v} \: d{v} + \omega^2 \int{y} \: d{y} &= 0 \\
    \frac{v^2}{2} + \frac{\omega^2 y^2}{2} &= C' \\
    \left( \dv{y}{t} \right) ^2 + \omega^2 y^2 &= C^2
\end{align*}

At maximum displacement, $y = r$ and $\displaystyle \dv{y}{t} = 0$ \\
So, $C^2 = \omega^2 r^2$ \\
\begin{align*}
    \left( \dv{y}{t} \right)^2 &= \omega^2(r^2 - y^2) \\
    \dv{y}{t} &= \omega \sqrt{r^2 - y^2} \\
    \int{\frac{1}{\sqrt{r^2-y^2}}} \: d{y} &= \int{\omega} \: d{t} \\
    \sin^{-1}{\frac{y}{r}} &= \omega t + \alpha
\end{align*}
\begin{equation}
    \boxed{ y = r \sin{(\omega t + \alpha)} }
\end{equation}

By expanding equation (6), we get
\begin{equation}
    y = r \sin{\omega t} \cos{\alpha} + r \cos{\omega t} \sin{\alpha}
\end{equation}
If $y = 0$ at $t = 0$, then $\alpha = 0$ \\
\begin{equation}
    y = r \sin{\omega t}
\end{equation}
If $y = r$ at $t = 0$, then $\alpha = \pi/2$ \\
\begin{equation}
    y = r \cos{\omega t}
\end{equation}

Hence, the general solution of the differential equation of SHM is
\begin{equation}
    \boxed{ y = A \sin{\omega t} + B \cos{\omega t} }
\end{equation}

\section{Energy in SHM}
\subsection{Total Energy of a Vibrating Particle}
\begin{align*}
    \text{Kinetic Energy} &= \frac{1}{2} m \left( \dv{y}{t} \right)^2 \\
    &= \frac{1}{2} m \omega^2 r^2 \cos^2{(\omega t + \alpha)} \\
    \text{Potential Energy} &= \frac{1}{2} k y^2 \\
    &= \frac{1}{2} m \omega^2 r^2 \\
    &= \frac{1}{2} m \omega^2 r^2 \sin^2{(\omega t + \alpha)}
\end{align*}

Thus, the total energy of the vibrating particle is
\begin{equation}
    \boxed{ E = \frac{1}{2} k r^2 = \frac{1}{2} m \omega^2 r^2 }
\end{equation}

\begin{figure}[htpb]
    \centering
    \includegraphics[width=0.5\textwidth]{2.png}
\end{figure}

\subsection{Average Kinetic Energy}
Kinetic energy of the particle is given by
\begin{equation}
    K = \frac{1}{2} m \left( \dv{y}{t} \right)^2 = \frac{1}{2} m \omega^2 r^2 \cos^2{(\omega t + \alpha)}
\end{equation}
Hence, average kinetic energy is
\begin{align*}
    \overline{K} &= \frac{1}{T} \int_{0}^{T} K \: dt \\
    &= \frac{1}{T} \int_{0}^{T} \frac{1}{2} m \omega^2 r^2 \cos^2{(\omega t + \alpha)} \: dt \\
    &= \frac{1}{2} m \omega^2 r^2 \frac{1}{T} \int_{0}^{T} \cos^2{(\omega t + \alpha)} \: dt \\
    &= \frac{1}{2} m \omega^2 r^2 \frac{1}{T} \int_{0}^{T} \frac{1 + \cos{2(\omega t + \alpha)}}{2} \: dt \\
    &= \frac{1}{2} m \omega^2 r^2 \frac{1}{T} \left[ \frac{t}{2} + \frac{\sin{2(\omega t + \alpha)}}{4\omega} \right]_{0}^{T} \\
    &= \frac{1}{2} m \omega^2 r^2 \frac{1}{T} \left[ \frac{T}{2} + \frac{\sin{2(\omega T + \alpha)}}{4\omega} - \frac{\sin{2\alpha}}{4\omega} \right] \\
    &= \frac{1}{2} m \omega^2 r^2 \frac{1}{T} \left[ \frac{T}{2} + \frac{\sin{2\alpha}}{4\omega} - \frac{\sin{2\alpha}}{4\omega} \right] \\
    &= \frac{1}{4} m \omega^2 r^2
\end{align*}
\begin{equation}
    \boxed{ \overline{K} = \frac{1}{4} m \omega^2 r^2 = \frac{1}{4}k r^2 = \frac{1}{2} E }
\end{equation}

\subsection{Average Potential Energy}
Potential energy of the particle is given by
\begin{equation}
    U = \frac{1}{2} k y^2 = \frac{1}{2} m \omega^2 r^2 \sin^2{(\omega t + \alpha)}
\end{equation}
Hence, average potential energy is
\begin{align*}
    \overline{U} &= \frac{1}{T} \int_{0}^{T} U \: dt \\
    &= \frac{1}{T} \int_{0}^{T} \frac{1}{2} m \omega^2 r^2 \sin^2{(\omega t + \alpha)} \: dt \\
    &= \frac{1}{2} m \omega^2 r^2 \frac{1}{T} \int_{0}^{T} \sin^2{(\omega t + \alpha)} \: dt \\
    &= \frac{1}{2} m \omega^2 r^2 \frac{1}{T} \int_{0}^{T} \frac{1 - \cos{2(\omega t + \alpha)}}{2} \: dt \\
    &= \frac{1}{2} m \omega^2 r^2 \frac{1}{T} \left[ \frac{t}{2} - \frac{\sin{2(\omega t + \alpha)}}{4\omega} \right]_{0}^{T} \\
    &= \frac{1}{2} m \omega^2 r^2 \frac{1}{T} \left[ \frac{T}{2} - \frac{\sin{2(\omega T + \alpha)}}{4\omega} + \frac{\sin{2\alpha}}{4\omega} \right] \\
    &= \frac{1}{2} m \omega^2 r^2 \frac{1}{T} \left[ \frac{T}{2} + \frac{\sin{2\alpha}}{4\omega} - \frac{\sin{2\alpha}}{4\omega} \right] \\
    &= \frac{1}{4} m \omega^2 r^2
\end{align*}
\begin{equation}
    \boxed{ \overline{U} = \frac{1}{4} m \omega^2 r^2 = \frac{1}{4}k r^2 = \frac{1}{2} E }
\end{equation}







\end{document}
