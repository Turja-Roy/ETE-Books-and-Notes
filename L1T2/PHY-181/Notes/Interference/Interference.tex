\documentclass[12pt]{article}
\usepackage[a4paper, margin=0.75in]{geometry}
\usepackage[document]{ragged2e}
\usepackage{graphicx}
\graphicspath{ {./images/} }
% \usepackage[tmargin=2cm,rmargin=1in,lmargin=1in,margin=0.85in,bmargin=2cm,footskip=.2in]{geometry}
\usepackage{enumerate}
\usepackage{framed}
\usepackage{amsmath,amsfonts,amsthm,thmtools,amssymb,mathtools,commath}
\usepackage{tikz}
\usepackage{xcolor}
\usepackage[most]{tcolorbox}


\tcbuselibrary{theorems}
\newtcbtheorem[number within=section]{example}{Example}%
{
    colback=green!5,
    frame hidden,
    detach title,
    before upper = \tcbtitle\par\smallskip,
    coltitle=green!35!black,
    % colframe=green!35!black,
    fonttitle=\bfseries\sffamily
    % description font=\mdseries
}{th}


\title{
    \textbf{Interference of Light}
}

\author{
    Turja Roy \\
    ID: 2108052
}
\date{}

\begin{document}
\maketitle
\tableofcontents
\newpage

\section{Introduction}
\begin{definition}{Interference of light}{}
    The phenomenon of redistribution of energy when two or more waves of same frequency and amplitude superimpose on each other is called interference of light.
\end{definition}

\begin{definition}{Coherent Source}{}
    Two sources of light are said to be coherent if they emit light waves of same frequency and amplitude, and constant phase difference.
\end{definition}

\subsection{Conditions for Interference}
\begin{itemize}
    \item The two beams of light must be coherent.
    \item The two beams of light must have same frequency.
    \item The two beams of light must have constant phase difference.
    \item The two beams of light must have same amplitude.
    \item The original source must be monochromatic.
    \item The fringe width should reasonably be as large as possible and the separation between them must be as small as possible, while the distance of the screen from the source should be as large as possible
    \item The two interfering waves must be travelling in the same direction.
\end{itemize}





















\end{document}
