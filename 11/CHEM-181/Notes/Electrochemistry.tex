\documentclass[12pt]{article}
\usepackage[a4paper, margin=0.75in]{geometry}
\usepackage[document]{ragged2e}
\usepackage{graphicx}
\usepackage{amsmath}
\usepackage{amssymb}
\usepackage{commath}
\usepackage{wrapfig}
\usepackage{enumerate}
\usepackage{framed}
\usepackage{xcolor}

\renewenvironment{leftbar}[1][\hsize]
{%
    \def\FrameCommand
    {%
        {\color{black}\vrule width 1pt}%
        \hspace{0pt}%must no space.
        \fboxsep=\FrameSep\colorbox{white}%
    }%
    \MakeFramed{\hsize#1\advance\hsize-\width\FrameRestore}%
}
{\endMakeFramed}

\title{
    \textbf{Physical Chemistry} \\
    \textbf{Electrochemistry}
}
\author{
    Note taken by: Turja Roy
}
\date{}

\begin{document}
\maketitle

\section*{Electrolysis and Electrolytic conductance}
\underline{\textbf{Electrolysis:}} The process of decomposition of chemical substances in presence of electricity. \\


\underline{\textbf{Electrolytes:}} Electrovalent substances that form ions in solutions which conduct an electric current. \\
\textit{Example: all ionic substances.} \\

\subsection*{Electricity and Electron: Historical reference}
\begin{enumerate}
    \item Egyptians from eel fish
    \item 500 BC -- Greek (Thelis) : Static current from Amber (a type of fossil)
    \item 1600 -- William Gilbert : de magnato -- Electricus (latin)
    \item Benzamin Franklin : Kite experiment -- Lighning = Electricity
    \item Michael Faraday : Faraday's laws of electrolysis
\end{enumerate}

\section*{Faraday's Laws of Electrolysis}
\subsection*{1st Law}
The amount of substance deposited at the electrode during electrolysis is directly proportional to the current passing through the solution. \\
\begin{minipage}{0.4\textwidth}
    \begin{align*}
        m &\propto Q \\
        m &\propto it \; \; \; \; \; \; [Q = it] \\
        m &= Zit
    \end{align*}
\end{minipage}
\hfill\vline\hfill
\begin{minipage}{0.5\textwidth}
    m: deposited substance \\
    Q: current (in Coulomb) \\
    Z: Electrochemical equivalent \\
    Z = 1 for i = 1 A and t = 1 s (Q = 1 C)
\end{minipage} \\
\underline{\textbf{Electrochemical equivalent (Z):}} The amount of substance deposited at electrode for passing 1 C or 1 A current for 1 s time. \\

\subsection*{2nd Law}
If the same quantity of electricity is passed through different electrolyte solutions for the same time, the amount of substances deposited at the electrodes is directly proportional to their equivalent weight/chemical equivalent.

\begin{align*}
    \frac{\text{Amount of Ag deposited}}{\text{amount of Cu deposited}} &= \frac{\text{Chemical equivalent of Ag}}{\text{Chemical equivalent of Cu}} \\
    \frac{\text{Ag deposited}}{\text{Al deposited}} &= \frac{\text{Chemical equivalent of Ag}}{\text{Chemical equivalent of Al}} 
\end{align*}

\begin{minipage}{0.45\textwidth}
    \begin{align*}
        Ag^+ + e^- &= 1F \\
        Cu^+ + 2e^- &= 2F \\
        Al^+ + 3e^- &= 3F \\
    \end{align*}
\end{minipage}
\hfill\vline\hfill
\begin{minipage}{0.45\textwidth}
    For 1 mol Ag $\rightarrow$ 1 F \\
    For 1 mol Cu $\rightarrow$ 2 F \\
    For 1 mol Al $\rightarrow$ 3 F \\
\end{minipage}

\begin{minipage}{0.45\textwidth}
    \quad\quad\quad\quad Quantity of electricity = $n \times F$
\end{minipage}
\hfill\vline\hfill
\begin{minipage}{0.5\textwidth}
    n: Valency of the ion \\
    F: Faraday number (96500 C) \\
    (Charge of 1 mole electron) \\
\end{minipage}
\vspace{1cm}

\section*{Electrolytic Conductance}
The power of electrolytes to conduct electric current is termed conductivity or conductance. Like metallic conductors, electrolytes obey Ohm's law. \\
The resistance of a conductor is directly proportional to its length $l$ and inversely proportional to its cross-section $A$. \\
\begin{minipage}{0.7\textwidth}
    \begin{align*}
        \text{Resistance, }
        R &\propto \frac{l}{A} \\
        R &= \rho \times \frac{l}{A}
    \end{align*}
\end{minipage} \\
\begin{minipage}{0.7\textwidth}
    \begin{align*}
        \text{Conductance, }
        C &= \frac{1}{R} \\
        &= \frac{A}{\rho l} \\
    \end{align*}
\end{minipage}

\subsection*{Specific Conductance}
The conductacne of 1 cc electrolytic solution. \\
\begin{minipage}{0.5\textwidth}
    $$ \kappa = \frac{1}{R} \cdot \frac{l}{A} $$
\end{minipage}
\hfill\vline\hfill
\begin{minipage}{0.4\textwidth}
    l = 1 cm \\
    A = 1 $cm^2$
\end{minipage}
$$ \text{Unit = } \frac{1}{ohm} \cdot \frac{cm}{cm^2} = ohm^{-1}cm^{-1} = S \ cm^{-1} $$

\subsection*{Equivalent Conductance}
The conductance of 1 g-equivalent of electrolyte dissolved in $V$ cc of water. \\
\begin{minipage}{0.5\textwidth}
    $$ \Lambda = \kappa \times V $$
\end{minipage}
\hfill\vline\hfill
\begin{minipage}{0.4\textwidth}
    $\Lambda$ = Equivalent conductance \\
    $V$ = Volume in cc
\end{minipage} \vspace{.5cm}

In general, if N gram-equivalent electrolyte is dissolved in $1000$ cc of solution, the volume of hte solution containing 1 gram-equivalent will be $1000/N$. Thus,
$$ \Lambda = \kappa \times \frac{1000}{N} $$

In the same way, if N gram-equivalent electrolyte is dissolved in $V$ cc of solution, the volume of hte solution containing 1 gram-equivalent will be $V/N$. Thus,
$$ \Lambda = \kappa \times \frac{V}{N} $$
\begin{align*}
    \text{Unit: }
    \Lambda &= \kappa \times V \\
    &= \frac{1}{R} \times \frac{l}{A} \times V \\
    &= \frac{1}{ohm} \times \frac{cm}{cm^2} \times \frac{cm^3}{eqvt} \\
    &= \text{ohm}^{-1} cm^2 \text{eqvt}^{-1} \\
\end{align*}

\subsection*{Molar Conductance}
The conductance of 1 mole of electrolyte dissolved in $V$ cc of solution.





    

\end{document}
