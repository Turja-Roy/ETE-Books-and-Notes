\usepackage{enumerate}
\usepackage{framed}
\usepackage{amsmath,amsfonts,amsthm,thmtools,amssymb,mathtools,commath}
\usepackage{tikz}
\usetikzlibrary{mindmap}
\usepackage{caption}
\usepackage{xcolor}
\usepackage[most]{tcolorbox}
\usepackage{hyperref}
\usepackage{cleveref}


%%%%%%%%%%%%%%%%
%  Definition  %
%%%%%%%%%%%%%%%%
\tcbuselibrary{theorems,skins,hooks}
\newtcbtheorem[number within=subsection]{definition}{Definition}%
{
    % theorem style=definition,
    enhanced,
    % breakable,
	before skip=2mm,after skip=2mm, colback=cyan!5,colframe=cyan!80!black,boxrule=0.5mm,
	attach boxed title to top left={xshift=1cm,yshift*=1mm-\tcboxedtitleheight},
	boxed title style={frame code={
					\path[fill=cyan]
					([yshift=-1mm,xshift=-1mm]frame.north west)
					arc[start angle=0,end angle=180,radius=1mm]
					([yshift=-1mm,xshift=1mm]frame.north east)
					arc[start angle=180,end angle=0,radius=1mm];
					\path[left color=cyan!30!black,right color=cyan!30!black,
						middle color=cyan!50!black]
					([xshift=-2mm]frame.north west) -- ([xshift=2mm]frame.north east)
					[rounded corners=1mm]-- ([xshift=1mm,yshift=-1mm]frame.north east)
					-- (frame.south east) -- (frame.south west)
					-- ([xshift=-1mm,yshift=-1mm]frame.north west)
					[sharp corners]-- cycle;
				},interior engine=empty,
		},
	fonttitle=\bfseries,
	title={#2},#1
}{def}

% \newcommand{\definition}[2]{
% \begin{definition}{#1}{}
%     #2
% \end{definition}
% }


%%%%%%%%%%%%%
%  Theorem  %
%%%%%%%%%%%%%
\tcbuselibrary{theorems,skins,hooks}
\newtcbtheorem[number within=subsection]{theorem}{Theorem}%
{
    theorem style=plain,
    enhanced,
    breakable,
    colframe=green!50!black,
    boxrule=1pt,
    titlerule=0mm,
    toptitle=1mm,
    bottomtitle=1mm,
    fonttitle=\bfseries,
    fontupper=\mdseries\itshape,
    coltitle=green!30!black,
    colbacktitle=cyan!15!white,
    colback=green!10,
    description font=\bfseries\sffamily
}{thrm}


%%%%%%%%%%%%%%%
%  Corollary  %
%%%%%%%%%%%%%%%
\tcbuselibrary{theorems,skins}
\newtcbtheorem[number within=tcb@cnt@theorem]{corollary}{Corollary}%
{
    theorem style=plain,
    enhanced,
    colframe=green,
    frame hidden,
    titlerule=0mm,
    toptitle=1mm,
    bottomtitle=1mm,
    fonttitle=\bfseries,
    fontupper=\mdseries\itshape,
    coltitle=green!30!black,
    colbacktitle=cyan!15!white,
    colback=green!10,
    description font=\bfseries\sffamily
}{corl}


%%%%%%%%%%%%%
%  Example  %
%%%%%%%%%%%%%
\tcbuselibrary{theorems,skins,hooks}
\newtcbtheorem[number within=subsection]{example}{Example}%
{
	enhanced,
	breakable,
	colback = gray!5,
	frame hidden,
	boxrule = 0sp,
	borderline west = {2pt}{0pt}{gray},
	sharp corners,
	detach title,
	before upper = \tcbtitle\par\smallskip,
    coltitle=gray!70!black,
	fonttitle = \bfseries\sffamily,
	description font = \mdseries\bfseries
}
{xmp}


%%%%%%%%%%%%%%
%  Exercise  %
%%%%%%%%%%%%%%
\tcbuselibrary{theorems,skins,hooks}
\newtcbtheorem[number within=section]{exercise}{Exercise}%
{
	enhanced,
	breakable,
	colback = gray!15,
	frame hidden,
	boxrule = 0sp,
	borderline west = {2pt}{0pt}{black},
    borderline north = {4pt}{0pt}{black},
	sharp corners,
	% detach title,
	before upper = \tcbtitle\par\smallskip,
    toptitle = 15pt,
    bottomtitle = 10pt,
    coltitle = white,
    colbacktitle = gray!60!black,
	fonttitle = \bfseries\sffamily,
	description font = \mdseries\bfseries
}{exc}

%%%%%%%%%%
%  Note  %
%%%%%%%%%%
\usetikzlibrary{arrows,calc,shadows.blur}
\tcbuselibrary{skins}
\newtcolorbox{note}[1][]{%
	enhanced jigsaw,
	colback=gray!20!white,%
	colframe=gray!80!black,
	size=small,
	boxrule=1pt,
	title=\textbf{Note:-},
	halign title=flush center,
	coltitle=black,
	breakable,
	drop shadow=black!50!white,
	attach boxed title to top left={xshift=1cm,yshift=-\tcboxedtitleheight/2,yshifttext=-\tcboxedtitleheight/2},
	minipage boxed title=1.5cm,
	boxed title style={%
			colback=white,
			size=fbox,
			boxrule=1pt,
			boxsep=2pt,
			underlay={%
					\coordinate (dotA) at ($(interior.west) + (-0.5pt,0)$);
					\coordinate (dotB) at ($(interior.east) + (0.5pt,0)$);
					\begin{scope}
						\clip (interior.north west) rectangle ([xshift=3ex]interior.east);
						\filldraw [white, blur shadow={shadow opacity=60, shadow yshift=-.75ex}, rounded corners=2pt] (interior.north west) rectangle (interior.south east);
					\end{scope}
					\begin{scope}[gray!80!black]
						\fill (dotA) circle (2pt);
						\fill (dotB) circle (2pt);
					\end{scope}
				},
		},
	#1,
}
