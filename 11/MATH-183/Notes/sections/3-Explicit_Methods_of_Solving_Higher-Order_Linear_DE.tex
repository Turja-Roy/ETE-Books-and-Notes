%%%%%%%%%%%%%%%%%%%%%%%%%%%%%%%%%%%%%%%%%%%%%%%%%%%%%%%%%%%%%%%%%%%%%%%%%%%%
%  Explicit Methods of Solving Higher-Order Linear Differential Equations  %
%%%%%%%%%%%%%%%%%%%%%%%%%%%%%%%%%%%%%%%%%%%%%%%%%%%%%%%%%%%%%%%%%%%%%%%%%%%%

\section{Explicit Methods of Solving Higher-Order Linear Differential Equations}

%%%%%%%%%%%%%%%%%%%%%%%%%%%%%%%%%%%%%%%%%%%%%%%%%%%
%  Basic Theory of Linear Differential Equations  %
%%%%%%%%%%%%%%%%%%%%%%%%%%%%%%%%%%%%%%%%%%%%%%%%%%%

\subsection{Basic Theory of Linear Differential Equations}

%%%%%%%%%%%%%%%%%%%%%%%%%%%%%%%%%%%%%%%%%%%%
%  Definition and Basic Existence Theorem  %
%%%%%%%%%%%%%%%%%%%%%%%%%%%%%%%%%%%%%%%%%%%%

\subsubsection{Definition and Basic Existence Theorem}

\begin{definition}{Linear Ordinary Differential Equation of Order $n$}{}
    A \textbf{linear ordinary differential equation of order $n$} in the dependent variable $y$ and the independent variable $x$ is an equation that is in, or can be expressed in the form
    \begin{equation} \label{eq1}
        a_0(x)y^{(n)} + a_1(x)y^{(n-1)}n + \cdots + a_{n-1}(x)y' + a_n(x)y = F(x)
    \end{equation}
    where $a_0$ is not identically zero. In the equation, $a_0, a_1, \cdots, a_n$ and $F$ are continuous real functions on a real interval $a \le x \le b$ and that $a_0(x) \neq 0$ for any $x$ on $a \le x \le b$. The $F(x)$ is called the nonhomogeneous term. If $F$ is identically zero, Equation (\ref{eq1}) reduces to
    \begin{equation} \label{eq2}
        a_0(x)y^{(n)} + a_1(x)y^{(n-1)}n + \cdots + a_{n-1}(x)y' + a_n(x)y = 0
    \end{equation}
    Equation (\ref{eq2}) is a \textbf{homogeneous differential equation of order $n$}.
\end{definition}

\begin{example}{}{}
    The equation \[
        y'' + 3xy' + x^3y = e^x
    \] is a linear ordinary differential equation.\\
    The equation \[
        y''' + xy'' + 3x^2y' - 5y = \sin x
    \] is a linear ODE of third order.
\end{example}

\begin{theorem}{Basic Existence Theorem}{}
    \\\textbf{Hypothesis:}
    \begin{enumerate}
        \item Consider the $n$th-order linear differential equation
            \begin{equation}\tag{3.1.1}
                a_0(x)y^{(n)} + a_1(x)y^{(n-1)}n + \cdots + a_{n-1}(x)y' + a_n(x)y = F(x)
            \end{equation}
        where $a_0, a_1, \cdots, a_n$ and $F$ are real functions on a real interval $a \le x \le b$ and $a_0(x) \neq 0$ for any  $x$ on $a \le x \le b$.
        \item Let $x_0$ be any point of the interval $a \le x \le b$, and let $c_0, c_1, \cdots, c_{n-1}$ be $n$ arbitrary real constants.
    \end{enumerate}
    \textbf{Conclusion:} There exists a unique solution $f$ of (\ref{eq1}) such that \[
        f(x_0) = c_0, f'(x_0) = c_1, \cdots, f^{(n-1)}(x_0) = c_{n-1}
    \]
    and this solution is defined over the entire interval $a \le x \le b$.
\end{theorem}

\begin{example}{
    Consider the initial-value problem
    \[ 2y''' + xy'' + 3x^2y' - 5y = \sin x \]
    \[ y(4)=3 \]
    \[ y'(4)=5 \]
    \[ y''(4)=-\frac{7}{2} \]
}{} \vspace{-10pt}
    Here we have a third-order problem. The coefficients $2, x, 3x^2,$ and $-5$, as well as the nonhomogeneous term $\sin{x}$, are all continuous for all $x \in (-\infty,\infty)$. The point $x_0=4$ certainly belongs to this intervall; the real numbers $c_0, c_1, $ and $c_2$ in this problem are 3, 5, and  $-\frac{7}{2}$ respectively. Theorem 3.1.1 assures is that this problem also has a unique solution which is defined for all $x \in (-\infty,\infty)$
\end{example}

\begin{corollary}{}{}
    \\\textbf{Hypothesis:} Let $f$ be a solution of the nth-order homogeneous linear DE
    \begin{equation}\tag{3.1.2}
        a_0(x)y^{(n)} + a_1(x)y^{(n-1)} + \cdots + a_{n-1}(x)y' + a_n(x)y = 0
    \end{equation}
    such that \[
        f(x_0)=0, f'(x_0)=0, \cdots, f^{(n-1)}(x_0)=0,
    \] where $x_0$ is a point of the interval $a \le x \le b$ in which the coefficients $a_0, a_1, \cdots, a_n$ are all continuous and $a_0(x) \neq 0$. \\~\\
    \textbf{Conclusion:} Then $f(x)=0$ for all $x$ on $a \le x \le b$.
\end{corollary}

\begin{example}{}{}
    The unique solution of $f$ of the third-order homogeneous equation \[
        y''' + 2y'' + 4xy' + x^2y = 0
    \] which is such that \[
        f(2) = f'(2) = f''(2) = 0
    \] is the trivial solution $f$ such that $f(x)=0$ for all $x$.
\end{example}

%%%%%%%%%%%%%%%%%%%%%%%%%%%%%%
%  The Homogeneous Equation  %
%%%%%%%%%%%%%%%%%%%%%%%%%%%%%%
\vspace{20pt}
\subsubsection{The Homogeneous Equation}

We now consider the fundamental results concerning the homogeneous equation
\begin{equation}\tag{3.1.2}
    a_0(x)y^{(n)} + a_1(x)y^{(n-1)} + \cdots + a_{n-1}(x)y' + a_n(x)y = 0
\end{equation}

\begin{theorem}{Basic Theorem on Linear Homogeneous Differential Equations}{}
    \\\textbf{Hypothesis:} Let $f_1, f_2, \cdots, f_m$ be any $m$ solutions of the homogeneous linear differential equation (\ref{eq2}).\\~\\
    \textbf{Conclusion:} Then \[
        c_1f_1 + c_2f_2 + \cdots + c_mf_m
    \] is also a solution of (\ref{eq2}), where $c_1, c_2, \cdots, c_m$ are $m$ arbitrary constants.\\~\\

    In other words: Any linear combination of solutions of the homogeneous linear differential equation (\ref{eq2}) is also a solution of (3.1.2).
\end{theorem}

\begin{definition}{Linear Combination}{}
    If $f_1, f_2, \cdots, f_m$ are $m$ given functions, and $c_1, c_2, \cdots, c_m$ are $m$ constants, then the expression \[
        c_1f_1 + c_2f_2 + \cdots + c_mf_m
    \] is called a linear combination of $f_1, f_2, \cdots, f_m$.
\end{definition}

\begin{example}{}{}
    $e^x, e^{-x}, e^{2x}$ are solutions of \[
        y''' - 2y'' - y' + 2y = 0
    \] Theorem 3.1.3 states that the linear combination $c_1e^x + c_2e^{-x} + c_3x^{2x}$ is also a solution for any constants $c_1, c_2, c_3$. For example, the particular linear combination \[
        2e^x - 3e^{-x} + \frac{2}{3}e^{2x}
    \] is a solution.
\end{example}

\begin{definition}{Linear Dependence}{}
    The $n$ functions $f_1, f_2, \cdots, f_n$ are called \textit{linearly dependent} on $a \le x \le b$ if there exist constants $c_1, c_2, \cdots, c_n$, not all zero, such that \[
        c_1f_1(x) + c_2f_2(x) + \cdots + c_mf_m(x) = 0
    \] for all $x$ such that $a \le x \le b$.
\end{definition}

\begin{definition}{Linear Independence}{}
    The $n$ functions $f_1, f_2, \cdots, f_n$ are called linearly independent on the interval $a \le x \le b$ if the relation \[
        c_1f_1(x) + c_2f_2(x) + \cdots + c_nf_n(x) = 0
    \] for all $x$ such that $a \le x \le b$ implies that \[
        c_1 = c_2 = \cdots = c_n = 0
    \] In other words, the only linear combination of $f_1, f_2, \cdots, f_n$ that is identically zero on $a \le x \le b$ is the trivial linear combination \[
    0\cdot f_1 + 0\cdot f_2 + \cdots + 0\cdot f_n
    \]
\end{definition}

\begin{theorem}{Linearly Independent Solutions of nth-Order Linear Differential Equation}{}
    The nth-order homogeneous linear differential equation
    \begin{equation}\tag{3.1.2}
        a_0(x)y^{(n)} + a_1(x)y^{(n-1)} + \cdots + a_{n-1}(x)y' + a_n(x)y = 0
    \end{equation}
    always possesses $n$ solutions that are linearly independent. Further, if $f_1, f_2, \cdots f_n$ are $n$ linearly independent solutions of (\ref{eq2}), then every solution $f$ of (3.1.2) can be expressed as a linear combination \[
        c_1f_1(x) + c_2f_2(x) + \cdots + c_nf_n(x) = 0
    \] of these $n$ linearly independent soulutions by proper choice of the constants $c_1, c_2, \cdots, c_n$.
\end{theorem}

\begin{example}{}{}
    We have observed that $\sin x$ and $\cos x$ are solutions of
    \begin{equation}
        y'' + y = 0
    \end{equation}
    for all $x \in (-\infty, \infty)$. Further, we can show that these two solutions are linearly independent. Suppose $f$ is any solution of (3.1.3). Then by Theorem 3.1.4 $f$ can be expressed as a certain linear combination $c_1\sin x + c_2\cos x$ of the two linearly independent solutions  $\sin x$ and $\cos x$ by proper choice of $c_1$ and $c_2$. That is, there exist two particular constants $c_1$ and $c_2$ such that
    \begin{equation}
        f(x) = c_1\sin x + c_2\cos x
    \end{equation}
    for all $x \in (-\infty,\infty)$. For example, it can be easily verified that $f(x) = \sin(x+\pi/6)$ is a solution of the equation (3.1.3). Since \[
        \sin \left( x + \frac{\pi}{6} \right) = \sin x \cos \frac{\pi}{6} + \cos x \sin \frac{\pi}{6} = \frac{\sqrt{3}}{2}\sin x + \frac{1}{2}\cos x,
    \] we see that the solution $\sin(x + \pi/6)$ can be expressed as the linear combination \[
        \frac{\sqrt{3}}{2}\sin x + \frac{1}{2}\cos x
    \] of the two linearly independent solutions $\sin x$ and $\cos x$. Here, $c_1 = \sqrt{3}/2$ and $c_2 = 1/2$
\end{example}

\begin{definition}{Fundamental Set of Solutions}{}
    If $f_1, f_2, \cdots, f_n$ are $n$ linearly independent solutions of the nth-order homogeneous linear differential equation
    \begin{equation}\tag{3.1.2}
        a_0(x)y^{(n)} + a_1(x)y^{(n-1)} + \cdots + a_{n-1}(x)y' + a_n(x)y = 0
    \end{equation}
    on $a \le x \le b$, then the set $f_1, f_2, \cdots, f_n$ is called a fundamental set of solutions of (\ref{eq2}) and the function $f$ defined by \[
        f(x) = c_1f_1(x) + c_2f_2(x) + \cdots + c_nf_n(x), \quad a \le x \le b,
    \] where $c_1, c_2, \cdots, c_n$ are arbitrary constants, is called a general solution of (\ref{eq2}) on  $a \le x \le b$.
\end{definition}

Therefore, if we can find $n$ linearly independent solutions of (\ref{eq2}), we can at once write the general solution of (3.1.2) as a general linear combination of these $n$ solutions.

\begin{example}{}{}
    The solutions $e^x, e^{-x},$ and $e^{2x}$ of \[
        y''' - 2y'' + y' + 2y = 0
    \] may be shown to be linearly independent for all $x \in (-\infty, \infty)$. Thus, $e^x, e^{-x},$ and $e^{2x}$ constitute a fundamental set of the given DE, and its general solution may be expressed as the linear combination \[
        c_1e^x + e^{-x} + c_3e^{2x}
    \] where $c_1$, $c_2$, and $c_3$ are arbitrary constants. We can write this as \[
        y = c_1e^x + e^{-x} + c_3e^{2x}
    \]
\end{example}

\begin{definition}{Wronskian}{}
    Let $f_1, f_2, \cdots, f_3$ be $n$ real functions each of which has an $(n-1)$th derivative on a real interval $a \le x \le b$. The determinant \[
        W(f_1, f_2, \cdots, f_n) = 
        \begin{vmatrix} 
            f_1 & f_2 & \cdots & f_n \\
            f_1' & f_2' & \cdots & f_n' \\
            \vdots & \vdots & \ddots & \vdots \\
            f_1^{(n-1)} & f_2^{(n-1)} & \cdots & f_n^{(n-1)}
        \end{vmatrix}
    \] is called the Wronskian of these $n$ functions. We observe that $W(f_1, f_2, \cdots, f_n)$ is itself a real function defined on $a \le x \le b$. Its value at $x$ is denoted by  $W(f_1, f_2, \cdots, f_n)(x)$ or by $W[ f_1(x), f_2(x), \cdots, f_n(x) ]$.
\end{definition}

\begin{theorem}{}{}
    The $n$ solutions $f_1, f_2, \cdots, f_n$ of the nth-order homogeneous linear DE (\ref{eq2}) are linearly independent on $a \le x \le b$ if and only if the Wronskian of $f_1, f_2, \cdots, f_n$ is different from zero for some $x$ on the interval $a \le x \le b$.
\end{theorem}

\begin{theorem}{}{}
    The Wronskian of $n$ solutions $f_1, f_2, \cdots, f_n$ of (\ref{eq2}) is either identically zero on $a \le x \le b$ or else is never zero on $a \le x \le b$.
\end{theorem}

\begin{example}{}{}
    We can apply Theorem 3.1.2 to show that the solutions $\sin x$ and $\cos x$ of \[
        y'' + y = 0
    \] are linearly independent. We find that \[
        W(\sin x, \cos x) = 
        \begin{vmatrix}
            \sin x & \cos x \\
            \cos x & -\sin x
        \end{vmatrix} 
        = -\sin^2x - \cos^2x = -1 \neq 0
    \] for all real $x$. Thus, since $W(\sin x, \cos x) \neq 0$ for all real $x$, we conclude that $\sin x$ and $\cos x$ are indeed linearly independent solutions of the given DE on every real interval.
\end{example}

\begin{example}{}{}
    The solutions $e^{x}, e^{-1}$, and $e^{2x}$ of \[
        y''' - 2y'' - y' + 2y = 0
    \] are linearly independent on every real interval, for \[
        W(e^x, e^{-1}, e^{2x}) =
        \begin{vmatrix}
            e^{x} & e^{-1} & e^{2x} \\
            e^{x} & -e^{-1} & 2e^{2x} \\
            e^{x} & e^{-1} & 4e^{2x}
        \end{vmatrix}
        = e^{2x}
        \begin{vmatrix}
            1 & 1 & 1 \\
            1 & -1 & 2 \\
            1 & 1 & 4
        \end{vmatrix} 
        = -6e^{2x} \neq 0
    \] for all real $x$.
\end{example}


%%%%%%%%%%%%%%%%%%%%%%%%
%  Reduction of Order  %
%%%%%%%%%%%%%%%%%%%%%%%%
\vspace{20pt}
\subsubsection{Reduction of Order}

\begin{theorem}{}{}
    \\\textbf{Hypothesis:} Let $f$ be a nontrivial solution of the nth-order homogeneous linear differential equation
    \begin{equation}\tag{3.1.2}
        a_0(x)y^{(n)} + a_1(x)y^{(n-1)} + \cdots + a_{n-1}(x)y' + a_n(x)y = 0
    \end{equation}
    \textbf{Conclusion:} The transformation $y = f(x)v$ reduces Equation (\ref{eq2}) to an $(n-1)$th-order homogeneous linear different equation in the dependent variable $w = dv/dx$.
\end{theorem}

According to this theorem, id one nonzero solution of the nth-order homogeneous linear DE (\ref{eq2}) is known, then by making the appropriate transformation we may reduce the given equation to another homogeneous linear DE that is one order lower than the original. We'll now investigate the second-order ($n=2$) DE in detal.\\

Suppose $f$ is a known nontrivial solution of the second-order homogeneous linear equation
\begin{equation}
    a_0(x)y'' + a_1(x)y' + a_2(x)y = 0
\end{equation}
Let's make the transformation
\begin{equation}
    y = f(x)v
\end{equation}
, where $f$ is the known solution of (3.1.5) and $v$ is a function of $x$ that will be determined. Then, differentiating, we obtain
\begin{equation}
    y' = f(x)v' + f'(x)v,
\end{equation}
\begin{equation}
    y'' = f(x)v'' + 2f'(x)v' + f''(x)v.
\end{equation}

Substituting (3.1.6), (3.1.7), and (3.1.8) into (3.1.5), we obtain \[
    a_0(x)[ y'' = f(x)v'' + 2f'(x)v' + f''(x)v ] + a_1(x)[ y' = f(x)v' + f'(x)v ] + a_2(x)f(x)v = 0
\] or \[
a_0(x)f(x)v'' + [ 2a_0(x)f'(x) + a_1f(x) ]v' + [ a_0(x)f''(x) + a_1(x)f'(x) + a_2(x)f(x) ]v = 0
\]
Since $f$ is a solution of (3.1.5), the coefficient of $v$ is zero, and so the last equation reduces to \[
    a_0(x)f(x)v'' + [ 2a_0(x)f'(x) + a_1(x)f(x) ]v' = 0
\]
Letting $w = v'$, this becomes
\begin{equation}
    a_0(x)f(x) \: \frac{dw}{dx} + [ 2a_0(x)f'(x) + a_1(x)f(x) ]w = 0
\end{equation}

This is a first-order homogeneous linear DE in the dependent variable $w$. The equation is separable; thus, assuming $f(x) \neq 0$ and $a_0(x) \neq 0$, we may write \[
    \frac{dw}{w} = -2 \left[ 2 \frac{f'(x)}{f(x)} + \frac{a_1(x)}{a_0(x)} \right] \: dx
\]
Thus integrating, we obtain \[
    \ln|w| = -\ln[f(x)]^2 - \int{ \frac{a_1(x)}{a_0(x)} } \: d{x} + \ln|c|
\] or \[
w = \frac{ c\cdot \exp\left[ -\int{ \frac{a_1(x)}{a_0(x)} } \: d{x} \right] }{[ f(x) ]^2}
\]
This is the general solution of Equation (3.1.9); choosing the particular solution for which $c=1$, recalling that $dv/dw = w$, and integrating again, we now obtain \[
    v = \int{ \frac{ \exp\left[ -\int{ \frac{a_1(x)}{a_0(x)} } \: d{x} \right] }{[ f(x) ]^2} } \: d{x} 
\]
Finally, from (3.1.6), we obtain \[
    y = f(x) \int{ \frac{ \exp\left[ -\int{ \frac{a_1(x)}{a_0(x)} } \: d{x} \right] }{[ f(x) ]^2} } \: d{x} 
\]
If we denote the right side of the function by $g$, which is a solution of (3.1.5), we can write the general solution of (3.1.5) as the following linear combination \[
    c_1f + c_2g
\] where $c_1$ and $c_2$ are arbitrary constants.

We also observe that $g$ and $f$ are linearly independent, since \[
    W(f,g)(x) =
    \begin{vmatrix}
        f(x) & g(x) \\
        f'(x) & g'(x)
    \end{vmatrix} =
    \begin{vmatrix}
        f(x) & f(x)v \\
        f'(x) & f(x)v' + f'(x)v
    \end{vmatrix}
    = [f(x)]^2v' = \exp\left[ -\int{\frac{a_1(x)}{a_0(x)}} \: d{x} \right] \neq 0
\]

\begin{theorem}{}{}
    \\\textbf{Hypothesis:} Let $f$ be a nontrivial solution of the second-order homogeneous linear differential equation
    \begin{equation}
        a_0(x)y'' + a_1(x)y' + a_2(x)y = 0
    \end{equation}
    \textbf{Conclusion 1:} The transformation $y = f(x)v$ reduces (3.1.10) to the first-order linear homogeneous DE \[
        a_0(x)f(x)\frac{dw}{dx} + [ 2a_0(x)f'(x) + a_1(x)f(x) ]w = 0
    \] in the dependent variable $w$, where $w = v'$ \\
    \textbf{Conclusion 2:} The particular solution \[
        w = \frac{\exp\left[ -\int{ \frac{a_1(x)}{a_0(x)} } \: d{x} \right]}{[f(x)]^2}
    \] of equation (3.1.10) gives rise to the function $v$, where \[
    v(x) = \int{ \frac{\exp\left[ -\int{\frac{a_1(x)}{a_0(x)}} \: d{x} \right]}{[f(x)]^2} } \: d{x}
    \] The function $g(x)$ defined by $g(x) = f(x)v(x)$ is then a solution of the second-order equation (3.1.10). \\
    \textbf{Conclusion 3:} The original known solution $f$ and the new solution $g$ are linearly independent solutions of (3.1.10), and hence the general solution of (3.1.10) is \[
        c_1f + c_2g
    \] where $c_1$ and $c_2$ are arbitrary constants.
\end{theorem}

\begin{example}{
        Given that $y=x$ is a solution of
        \begin{equation}\tag{A}
            (x^2+1)y'' - 2xy' + 2y = 0
        \end{equation}
        find a linearly independent solution by reducing the order.
    }{}
    First applying the transformation \[
        y = xv
    \] Then \[
        y' = xv' + v, \text{ and } y'' = xv'' + 2v'
    \] Substituting these into (A) we obtain \[
    (x^2+1)(xv''+2v') - 2x(xv'+v) + 2xv = 0
    \] \[ x(x^2+1)v'' + 2v' = 0 \]
    Letting $w = v'$,
    \[ x(x^2+1)\frac{dw}{dx} + 2w = 0 \]
    \[ \int{\frac{dw}{w}} = -\int{\frac{2dx}{x(x^2+1)}} = \int{\left( -\frac{2}{x} + \frac{2x}{x^2+1} \right)} \: d{x} \]
    \[ \ln|w| = \ln|x^2+1| - \ln|x|^2 + \ln|c| \]
    \[ w = \frac{c(x^2+1)}{x^2} \]
    Letting $c = 1$,
    \[ v(x) = \int{\left( 1 + \frac{1}{x^2} \right)} \: d{x} = x - \frac{1}{x} \]
    \[ \therefore g = f(x)v(x) = x\cdot\left( x-\frac{1}{x} \right) = x^2-1 \]
    Hence, the general solution of (A) is \[
        \boxed{ y = c_1x + c_2(x^2-1) }
    \]
\end{example}


%%%%%%%%%%%%%%%%%%%%%%%%%%%%%%%%%
%  The Nonhomogeneous Equation  %
%%%%%%%%%%%%%%%%%%%%%%%%%%%%%%%%%
\vspace{20pt}
\subsubsection{The Nonhomogeneous Equation}
\begin{equation}\tag{3.1.1}
    a_0(x)y^{(n)} + a_1(x)y^{(n-1)} + \cdots + a_{n-1}(x)y' + a_n(x)y = F(x)
\end{equation}

\begin{theorem}{}{}
    \\\textbf{Hypothesis:}
    \begin{enumerate}
        \item Let $v$ be any solution of the given (nonhomogeneous) nth-order linear differential equation (\ref{eq1})
        \item Let $u$ be any solution of the corresponding homogeneous equation
            \begin{equation} \tag{3.1.2}
                a_0(x)y^{(n)} + a_1(x)y^{(n-1)} + \cdots + a_{n-1}(x)y' + a_n(x)y = 0
            \end{equation}
    \end{enumerate}
    \textbf{Conclusion:} Then $u + v$ is also a solution of the given (nonhomogeneous) equation (\ref{eq1})
\end{theorem}

\begin{theorem}{}{}
    \\\textbf{Hypothesis:}
    \begin{enumerate}
        \item Let $y_p$ be a given solution of the nth-order nonhomogeneous linear equation (\ref{eq1}) involving no arbitrary constants.
        \item Let \[
            y_c = c_1y_1 + c_2y_2 + \cdots + c_ny_n
        \] be the general solution of the corresponding homogeneous equation (\ref{eq2})
    \end{enumerate}
    \textbf{Conclusion:} Then every solution $\phi$ of the nth-order nonhomogeneous equation (\ref{eq1}) can be expressed in the form \[
        y_c + y_p
    \] that is, \[
        c_1y_1 + c_2y_2 + \cdots + c_ny_n + y_p
    \] for suitable choice of the $n$ arbitrary constants $c_1, c_2, \cdots, c_n$.
\end{theorem}

\begin{definition}{}{}
    Consider the nth-order nonhomogeneous linear DE
    \begin{equation} \tag{3.1.1}
        a_0(x)y^{(n)} + a_1(x)y^{(n-1)} + \cdots + a_{n-1}(x)y' + a_n(x)y = F(x)
    \end{equation}
    and the corresponding homogeneous equation
    \begin{equation} \tag{3.1.2}
        a_0(x)y^{(n)} + a_1(x)y^{(n-1)} + \cdots + a_{n-1}(x)y' + a_n(x)y = 0
    \end{equation}
    \textbf{Complimentary Function:} $y_c =$ general solution of (\ref{eq2})\\
    \textbf{Particular Solution:} $y_p =$ any solution of (\ref{eq1}) involving no arbitrary constant\\
    \textbf{General Solution of (\ref{eq1}):} $y_c + y_p$
\end{definition}

\begin{example}{}{}
    Consider the DE \[
        y'' + y = x
    \] The complimentary function \[
        y_c = c_1\sin x + c_2\cos x
    \] of the corresponding homogeneous DE \[
        y'' + y = 0
    \] A particular integral is \[
        y_p = x
    \] Thus the general solution is \[
        y = y_c + y_p = c_1\sin x + c_2\cos x + x
    \]
\end{example}

\begin{theorem}{}{}
    \\\textbf{Hypothesis:}
    \begin{enumerate}
        \item Let $f_1$ be a particular integral of
            \begin{equation}
                a_0(x)y^{(n)} + a_1(x)y^{(n-1)} + \cdots + a_{n-1}(x)y' + a_n(x)y = F_1(x)
            \end{equation}
        \item Let $f_2$ be a particular integral of
            \begin{equation}
                a_0(x)y^{(n)} + a_1(x)y^{(n-1)} + \cdots + a_{n-1}(x)y' + a_n(x)y = F_2(x)
            \end{equation}
    \end{enumerate}
    \textbf{Conclusion:} Then $k_1f_1 + k_2f_2$ is a particular integral of
    \begin{equation}
        a_0(x)y^{(n)} + a_1(x)y^{(n-1)} + \cdots + a_{n-1}(x)y' + a_n(x)y = k_1F_1(x) + k_2F_2(x)
    \end{equation}
\end{theorem}


%%%%%%%%%%%%%%%%%%%%%%%%%%%%%%%%%%%%%%%%%%%%%%%%%%%%%%%%%%%%%%%%
%  The Homogeneous Linear Equation with Constant Coefficients  %
%%%%%%%%%%%%%%%%%%%%%%%%%%%%%%%%%%%%%%%%%%%%%%%%%%%%%%%%%%%%%%%%
\vspace{20pt}
\subsection{The Homogeneous Linear Equation with Constant Coefficients}

%%%%%%%%%%%%%%%%%%
%  Introduction  %
%%%%%%%%%%%%%%%%%%

\subsubsection{Introduction}
In this section we'll consider the special case of the nth-order homogeneous linear differential equation in which all the coefficients are real constants. That is, we'll deal with the following equation
\begin{equation} \label{eq3}
    a_0y^{(n)} + a_1y^{(n-1)} + \cdots + a_{n-1}y' + a_ny = 0
\end{equation}
where $a_0, a_1, \cdots, a_{n-1}, a_n$ are real constants.

The DE (\ref{eq3}) requires a function $f$ having the property such that if it and its various derivatives are each multiplied by certain constants and the resultant products are then added, the result will be equal to zero for the appropriate domain of this equation.\\~\\
Therefore, we need a function such that its derivatives are constant multiples of itself. That is, having the following property \[
    \dv[k]{}{x} f(x) = cf(x)
\]
We all know such a function is $e^{mx}$, and \[
    \dv[k]{}{x} e^{mx} = m^ke^{mx}
\]
Thus we'll seek solutions of (\ref{eq3}) of the form $y=e^{mx}$.\\~\\
Suppose $y=e^{mx}$ is a solution for certain $m$. Then we have:
\begin{align*}
    y' &= me^{mx} \\
    y'' &= m^2e^{mx} \\
    \vdots & \\
    y^{(n)} &= m^ne^{mx}
\end{align*}
Substituting in (\ref{eq3}) we obtain \[
    a_0m^ne^{mx} + a_1m^{n-1}e^{mx} + \cdots + a_{n-1}me^{mx} + a_ne^{mx} = 0
\] or \[
    e^{mx}( a_0m_n + a_1m^{n-1} + \cdots + a_{n-1}m + a_n ) = 0
\]
Since $e^{mx} \neq 0$, we obtain the polynomial equation in the unknown $m$ :
\begin{equation} \label{eq4}
    a_0m_n + a_1m^{n-1} + \cdots + a_{n-1}m + a_n = 0
\end{equation}
This equation is called the \textbf{Auxiliary Equation} or the \textbf{Characteristic Equation} of the given differential equation (\ref{eq3}).\\~\\
Note that (\ref{eq4}) is formally obtained from (\ref{eq3}). If $y=e^{mx}$ is a solution of (\ref{eq3}), then the constant $m$ must satisfy (\ref{eq4}). Hence, to solve (\ref{eq3}), we write the auxiliary equation (\ref{eq4}) and solve it for $m$.


%%%%%%%%%%%%%%%%%%%%%%%%%%%%%%%%%
%  Case 1: Distinct Real Roots  %
%%%%%%%%%%%%%%%%%%%%%%%%%%%%%%%%%
\vspace{20pt}
\subsubsection{Case 1: Distinct Real Roots}

If the AE of (\ref{eq3}) has the $n$ distinct real roots $m_1, m_2, \cdots, m_n$, then $e^{m_1x}, e^{m_2x}, \cdots, e^{m_nx}$ are $n$ distinct solutions of (\ref{eq3}).

\begin{theorem}{}{}
    Consider the nth-order homogeneous linear differential equation (\ref{eq3}) with constant coefficients. If the auxiliary equation (\ref{eq4}) has the $n$ distinct real roots \[
        m_1, m_2, \cdots, m_n
    \], then the general solution of (\ref{eq3}) is \[
        y = c_1e^{m_1x} + c_2e^{m_2x} + \cdots + c_ne^{m_nx}
    \], where $c_1, c_2, \cdots, c_n$ are arbitrary constants.
\end{theorem}

\begin{example}{
    Solve the differential equation \[
        y'' - 3y' + 2y = 0
\] }{}
    The auxiliary equation is \[
        m^2 - 3m + 2y = 0
    \]\[ (m-1)(m-2) = 0 \]
    \[ m = 1,2 \]
    The roots are real and distinct. Thus $e^{x}$ and $e^{2x}$ are the solutions, and the general solution is \[
        y = c_1e^{x} + c_2e^{2x}
    \]
    We can also verify that the solutions are linearly independent. Their Wronskian : \[
        W(e^{x}, e^{2x}) =
        \begin{vmatrix}
            e^{x} & e^{2x} \\
            e^{x} & 2e^{2x}
        \end{vmatrix} = e^{3x} \neq 0
    \]
\end{example}
























