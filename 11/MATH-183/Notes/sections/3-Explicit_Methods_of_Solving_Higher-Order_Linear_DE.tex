%%%%%%%%%%%%%%%%%%%%%%%%%%%%%%%%%%%%%%%%%%%%%%%%%%%%%%%%%%%%%%%%%%%%%%%%%%%%
%  Explicit Methods of Solving Higher-Order Linear Differential Equations  %
%%%%%%%%%%%%%%%%%%%%%%%%%%%%%%%%%%%%%%%%%%%%%%%%%%%%%%%%%%%%%%%%%%%%%%%%%%%%

\section{Explicit Methods of Solving Higher-Order Linear Differential Equations}

%%%%%%%%%%%%%%%%%%%%%%%%%%%%%%%%%%%%%%%%%%%%%%%%%%%
%  Basic Theory of Linear Differential Equations  %
%%%%%%%%%%%%%%%%%%%%%%%%%%%%%%%%%%%%%%%%%%%%%%%%%%%

\subsection{Basic Theory of Linear Differential Equations}

%%%%%%%%%%%%%%%%%%%%%%%%%%%%%%%%%%%%%%%%%%%%
%  Definition and Basic Existence Theorem  %
%%%%%%%%%%%%%%%%%%%%%%%%%%%%%%%%%%%%%%%%%%%%

\subsubsection{Definition and Basic Existence Theorem}

\begin{definition}{Linear ODE and Homogeneous DE of Order $n$}{}
    A \textbf{linear ordinary differential equation of order $n$} in the dependent variable $y$ and the independent variable $x$ is an equation that is in, or can be expressed in the form
    \begin{equation}
        a_0(x)y^{(n)} + a_1(x)y^{(n-1)}n + \cdots + a_{n-1}(x)y' + a_n(x)y = F(x)
    \end{equation}
    where $a_0$ is not identically zero. In the equation, $a_0, a_1, \cdots, a_n$ and $F$ are continuous real functions on a real interval $a \le x \le b$ and that $a_0(x) \neq 0$ for any $x$ on $a \le x \le b$. The $F(x)$ is called the nonhomogeneous term. If $F$ is identically zero, Equation (3.1.1) reduces to
    \begin{equation}
        a_0(x)y^{(n)} + a_1(x)y^{(n-1)}n + \cdots + a_{n-1}(x)y' + a_n(x)y = 0
    \end{equation}
    Equation (3.1.2) is a \textbf{homogeneous differential equation of order $n$}.
\end{definition}

\begin{example}{}{}
    The equation \[
        y'' + 3xy' + x^3y = e^x
    \] is a linear ordinary differential equation.\\
    The equation \[
        y''' + xy'' + 3x^2y' - 5y = \sin x
    \] is a linear ODE of third order.
\end{example}

\begin{theorem}{Basic Existence Theorem}{}
    \\\textbf{Hypothesis:}
    \begin{enumerate}
        \item Consider the $n$th-order linear differential equation
            \begin{equation}
                a_0(x)y^{(n)} + a_1(x)y^{(n-1)}n + \cdots + a_{n-1}(x)y' + a_n(x)y = F(x)
            \end{equation}
        where $a_0, a_1, \cdots, a_n$ and $F$ are real functions on a real interval $a \le x \le b$ and $a_0(x) \neq 0$ for any  $x$ on $a \le x \le b$.
        \item Let $x_0$ be any point of the interval $a \le x \le b$, and let $c_0, c_1, \cdots, c_{n-1}$ be $n$ arbitrary real constants.
    \end{enumerate}
    \textbf{Conclusion:} There exists a unique solution $f$ of (3.1.3) such that \[
        f(x_0) = c_0, f'(x_0) = c_1, \cdots, f^{(n-1)}(x_0) = c_{n-1}
    \]
    and this solution is defined over the entire interval $a \le x \le b$.
\end{theorem}

\begin{example}{
    Consider the initial-value problem
    \[ 2y''' + xy'' + 3x^2y' - 5y = \sin x \]
    \[ y(4)=3 \]
    \[ y'(4)=5 \]
    \[ y''(4)=-\frac{7}{2} \]
}{} \vspace{-10pt}
    Here we have a third-order problem. The coefficients $2, x, 3x^2,$ and $-5$, as well as the nonhomogeneous term $\sin{x}$, are all continuous for all $x \in (-\infty,\infty)$. The point $x_0=4$ certainly belongs to this intervall; the real numbers $c_0, c_1, $ and $c_2$ in this problem are 3, 5, and  $-\frac{7}{2}$ respectively. Theorem 3.1.1 assures is that this problem also has a unique solution which is defined for all $x \in (-\infty,\infty)$
\end{example}

\begin{corollary}{}{}
    \\\textbf{Hypothesis:} Let $f$ be a solution of the nth-order homogeneous linear DE
    \begin{equation}
        a_0(x)y^{(n)} + a_1(x)y^{(n-1)} + \cdots + a_{n-1}(x)y' + a_n(x)y = 0
    \end{equation}
    such that \[
        f(x_0)=0, f'(x_0)=0, \cdots, f^{(n-1)}(x_0)=0,
    \] where $x_0$ is a point of the interval $a \le x \le b$ in which the coefficients $a_0, a_1, \cdots, a_n$ are all continuous and $a_0(x) \neq 0$. \\~\\
    \textbf{Conclusion:} Then $f(x)=0$ for all $x$ on $a \le x \le b$.
\end{corollary}

\begin{example}{}{}
    The unique solution of $f$ of the third-order homogeneous equation \[
        y''' + 2y'' + 4xy' + x^2y = 0
    \] which is such that \[
        f(2) = f'(2) = f''(2) = 0
    \] is the trivial solution $f$ such that $f(x)=0$ for all $x$.
\end{example}

\pagebreak
%%%%%%%%%%%%%%%%%%%%%%%%%%%%%%
%  The Homogeneous Equation  %
%%%%%%%%%%%%%%%%%%%%%%%%%%%%%%

\subsubsection{The Homogeneous Equation}

We now consider the fundamental results concerning the homogeneous equation
\begin{equation}
    a_0(x)y^{(n)} + a_1(x)y^{(n-1)} + \cdots + a_{n-1}(x)y' + a_n(x)y = 0
\end{equation}

\begin{theorem}{Basic Theorem on Linear Homogeneous Differential Equations}{}
    \\\textbf{Hypothesis:} Let $f_1, f_2, \cdots, f_m$ be any $m$ solutions of the homogeneous linear differential equation (3.1.5).\\~\\
    \textbf{Conclusion:} Then \[
        c_1f_1 + c_2f_2 + \cdots + c_mf_m
    \] is also a solution of (3.1.5), where $c_1, c_2, \cdots, c_m$ are $m$ arbitrary constants.\\~\\

    In other words: Any linear combination of solutions of the homogeneous linear differential equation (3.1.5) is also a solution of (3.1.5).
\end{theorem}

\begin{definition}{Linear Combination}{}
    If $f_1, f_2, \cdots, f_m$ are $m$ given functions, and $c_1, c_2, \cdots, c_m$ are $m$ constants, then the expression \[
        c_1f_1 + c_2f_2 + \cdots + c_mf_m
    \] is called a linear combination of $f_1, f_2, \cdots, f_m$.
\end{definition}

\begin{example}{}{}
    $e^x, e^{-x}, e^{2x}$ are solutions of \[
        y''' - 2y'' - y' + 2y = 0
    \] Theorem 3.1.3 states that the linear combination $c_1e^x + c_2e^{-x} + c_3x^{2x}$ is also a solution for any constants $c_1, c_2, c_3$. For example, the particular linear combination \[
        2e^x - 3e^{-x} + \frac{2}{3}e^{2x}
    \] is a solution.
\end{example}

\begin{definition}{Linear Dependence}{}
    The $n$ functions $f_1, f_2, \cdots, f_n$ are called \textit{linearly dependent} on $a \le x \le b$ if there exist constants $c_1, c_2, \cdots, c_n$, not all zero, such that \[
        c_1f_1(x) + c_2f_2(x) + \cdots + c_mf_m(x) = 0
    \] for all $x$ such that $a \le x \le b$.
\end{definition}

\begin{definition}{Linear Independence}{}
    The $n$ functions $f_1, f_2, \cdots, f_n$ are called linearly independent on the interval $a \le x \le b$ if the relation \[
        c_1f_1(x) + c_2f_2(x) + \cdots + c_nf_n(x) = 0
    \] for all $x$ such that $a \le x \le b$ implies that \[
        c_1 = c_2 = \cdots = c_n = 0
    \] In other words, the only linear combination of $f_1, f_2, \cdots, f_n$ that is identically zero on $a \le x \le b$ is the trivial linear combination \[
    0\cdot f_1 + 0\cdot f_2 + \cdots + 0\cdot f_n
    \]
\end{definition}

\begin{theorem}{Linearly Independent Solutions of n-th Order Linear Differential Equation}{}
    The n-th order homogeneous linear differential equation
    \begin{equation}
        a_0(x)y^{(n)} + a_1(x)y^{(n-1)} + \cdots + a_{n-1}(x)y' + a_n(x)y = 0
    \end{equation}
    always possesses $n$ solutions that are linearly independent. Further, if $f_1, f_2, \cdots f_n$ are $n$ linearly independent solutions of (3.1.6), then every solution $f$ of (3.1.6) can be expressed as a linear combination \[
        c_1f_1(x) + c_2f_2(x) + \cdots + c_nf_n(x) = 0
    \] of these $n$ linearly independent soulutions by proper choice of the constants $c_1, c_2, \cdots, c_n$.
\end{theorem}

\begin{example}{}{}
    We have observed that $\sin x$ and $\cos x$ are solutions of
    \begin{equation}
        y'' + y = 0
    \end{equation}
    for all $x \in (-\infty, \infty)$. Further, we can show that these two solutions are linearly independent. Suppose $f$ is any solution of (4.7). Then by Theorem 3.1.4 $f$ can be expressed as a certain linear combination $c_1\sin x + c_2\cos x$ of the two linearly independent solutions  $\sin x$ and $\cos x$ by proper choice of $c_1$ and $c_2$. That is, there exist two particular constants $c_1$ and $c_2$ such that
    \begin{equation}
        f(x) = c_1\sin x + c_2\cos x
    \end{equation}
    for all $x \in (-\infty,\infty)$. For example, it can be easily verified that $f(x) = \sin(x+\pi/6)$ is a solution of the equation (3.1.7). Since \[
        \sin \left( x + \frac{\pi}{6} \right) = \sin x \cos \frac{\pi}{6} + \cos x \sin \frac{\pi}{6} = \frac{\sqrt{3}}{2}\sin x + \frac{1}{2}\cos x,
    \] we see that the solution $\sin(x + \pi/6)$ can be expressed as the linear combination \[
        \frac{\sqrt{3}}{2}\sin x + \frac{1}{2}\cos x
    \] of the two linearly independent solutions $\sin x$ and $\cos x$. Here, $c_1 = \sqrt{3}/2$ and $c_2 = 1/2$
\end{example}

\begin{definition}{Fundamental Set of Solutions}{}
    If $f_1, f_2, \cdots, f_n$ are $n$ linearly independent solutions of the n-th order homogeneous linear differential equation
     \begin{equation}
        a_0(x)y^{(n)} + a_1(x)y^{(n-1)} + \cdots + a_{n-1}(x)y' + a_n(x)y = 0
    \end{equation}
    on $a \le x \le b$, then the set $f_1, f_2, \cdots, f_n$ is called a fundamental set of solutions of (3.1.9) and the function $f$ defined by \[
        f(x) = c_1f_1(x) + c_2f_2(x) + \cdots + c_nf_n(x), \quad a \le x \le b,
    \] where $c_1, c_2, \cdots, c_n$ are arbitrary constants, is called a general solution of (3.1.9 on  $a \le x \le b$.
\end{definition}

Therefore, if we can find $n$ linearly independent solutions of (3.1.9), we can at once write the general solution of (3.1.9) as a general linear combination of these $n$ solutions.

\begin{example}{}{}
    The solutions $e^x, e^{-x},$ and $e^{2x}$ of \[
        y''' - 2y'' + y' + 2y = 0
    \] may be shown to be linearly independent for all $x \in (-\infty, \infty)$. Thus, $e^x, e^{-x},$ and $e^{2x}$ constitute a fundamental set of the given DE, and its general solution may be expressed as the linear combination \[
        c_1e^x + e^{-x} + c_3e^{2x}
    \] where $c_1$, $c_2$, and $c_3$ are arbitrary constants. We can write this as \[
        y = c_1e^x + e^{-x} + c_3e^{2x}
    \]
\end{example}

\begin{definition}{Wronskian}{}
    Let $f_1, f_2, \cdots, f_3$ be $n$ real functions each of which has an $(n-1)$th derivative on a real interval $a \le x \le b$. The determinant \[
        W(f_1, f_2, \cdots, f_n) = 
        \begin{vmatrix} 
            f_1 & f_2 & \cdots & f_n \\
            f_1' & f_2' & \cdots & f_n' \\
            \vdots & \vdots & \ddots & \vdots \\
            f_1^{(n-1)} & f_2^{(n-1)} & \cdots & f_n^{(n-1)}
        \end{vmatrix}
    \] is called the Wronskian of these $n$ functions. We observe that $W(f_1, f_2, \cdots, f_n)$ is itself a real function defined on $a \le x \le b$. Its value at $x$ is denoted by  $W(f_1, f_2, \cdots, f_n)(x)$ or by $W[ f_1(x), f_2(x), \cdots, f_n(x) ]$.
\end{definition}





















