%%%%%%%%%%%%%%%%%%%%%%%%%%%%
%  DE and their Solutions  %
%%%%%%%%%%%%%%%%%%%%%%%%%%%%

\section{Differential Equations and Their Solutions}

%%%%%%%%%%%%%%%%%%%%%%%%%%
%  Classification of DE  %
%%%%%%%%%%%%%%%%%%%%%%%%%%

\subsection{Classification of Differential Equations}

\begin{definition}{Differential Equation}
    
    Differential equation is an equation involving derivatives of one or more dependent variables with respect to one or more independent variables.
\end{definition}

\begin{definition}{Ordinary Differential Equation}
    
    A differential equation involving ordinary derivatives of one or more dependent variables with respect to a single independent variable is called an ordinary differential equation.
\end{definition}
\begin{example}{Ordinary Differential Equations:}
    
    \begin{equation}
        \frac{dy}{dx} + xy \left( \frac{d}{dx} \right)^2 = 0
    \end{equation}
    \begin{equation}
        \frac{d^4 x}{dt^4} + 5 \frac{d^2 x}{dt^2} + 3x = \sin{t}
    \end{equation}
\end{example}

\begin{definition}{Partial Differential Equation}
    
    A differential equation involving partial derivatives of one or more dependent variables with respect to more than one independent variables is called an partial differential equation.
\end{definition}
\begin{example}{Partial Differential Equations:}
    
    \begin{equation}
        \pdv{v}{s} + \pdv{v}{t} = v
    \end{equation}
    \begin{equation}
        \pdv[2]{u}{x} + \pdv[2]{u}{y} + \pdv[2]{u}{z} = 0
    \end{equation}
\end{example}

\begin{definition}{Order and Degree of Differential Equations}
    
    \textbf{Order of DE:} The order of the highest ordered derivative involved in a differential equation is called the order of the differential equation. \\~\\
    \textbf{Degree of DE:} The power of the highest order derivative involved in a differential equation is called the degree of the differential equation.
\end{definition}

\begin{definition}{Linearity of Differential Equations}
    
    If the dependent variable  and its various derivatives occur to the first degree only, the DE is a linear DE. Otherwise it's a non-linear DE. \[
        a_0(x)\dv[n]{y}{x} + a_1(x)\dv[n-1]{y}{x} + \cdots + a_{n-1}(x)\dv{y}{x} + a_n(x)y = b(x)
    \] Linear DE can also be classified as linear with \textit{constant} and \textit{variable} coefficients.
\end{definition}

\begin{example}{Ordinary Differential Equations: Orders, Degree, Linearity}
    
    \begin{align*}
        \dv[3]{y}{x} - 3\dv[2]{y}{x} + 3\dv{y}{x} - 6y = \sin{x} &\text\;\;\;\;\;\text{3rd ord 1st deg Lin} \\
        \dv[2]{y}{x} + \left( \dv{y}{x} \right)^2 + y = 0 &\text\;\;\;\;\;\text{2nd ord 1st deg Non-Lin} \\
        y = x\dv{y}{x} + \sqrt{1 + \dv[2]{y}{x} } &\text\;\;\;\;\;\text{2nd ord 1st deg Non-Lin} \\
        \dv[4]{x}{t} + t^2\dv[3]{x}{t} + \dv{y}{x} = \sin{t} &\text\;\;\;\;\;\text{4th ord 1st deg Lin} \\
        \dv[2]{y}{x} + 5\dv{y}{x} + 6y^2 = 0 &\text\;\;\;\;\;\text{2nd ord 1st deg Non-Lin} \\
        \dv[2]{y}{x} + 5 \left( \dv{y}{x} \right)^3 + 6y = 0 &\text\;\;\;\;\;\text{2nd ord 1st deg Non-Lin} \\
        \dv[2]{y}{x} + 5y\dv{y}{x} + 6y = 0 &\text\;\;\;\;\;\text{2nd ord 1st deg Lin}
    \end{align*}
\end{example}


%%%%%%%%%%%%%%%
%  Solutions  %
%%%%%%%%%%%%%%%

\subsection{Solutions}

%%%%%%%%%%%%%%%%%%%%%%%%%%%%
%  A. Nature of Solutions  %
%%%%%%%%%%%%%%%%%%%%%%%%%%%%

\subsubsection{Nature of Solutions}
An nth-order Differential Equation:
\begin{equation}
    F \left[ x, y, \dv{y}{x}, \cdots, \dv[n]{y}{x} \right] = 0
\end{equation}

\begin{definition}{Explicit solution}
    
    $f$ is an explicit solution of (1.2.1) if \[
        \forall x \in I, F \left[ x, f(x), f'(x), \cdots, f^{(n)}(x) \right] = 0
    \] where $I$ is a real interval.
\end{definition}

\begin{definition}{Implicit solution}
    
    $g(x,y)=0$ is an implicit solution if this relation defines at least one real function $f(x)$ on an interval $I$ such that $f$ is an explicit solution of (1.2.1)
\end{definition}

\begin{example}{Explicit and Implicit Solutions}
    
    \[ x^2 + y^2 - 25 = 0 \;\;:\;\; \text{ Implicit solution } \]
    \[ 2x + 2y\dv{y}{x} = 0 \]
    \[ x + y\dv{y}{x} = 0 \;\;:\;\; \text{ Differential Equation } \]
    \[ y = \pm \sqrt{25 - x^2} \;;\; -5 \le x \le 5 \;\;:\;\; \text{ Explicit solution } \]
\end{example}

%%%%%%%%%%%%%%%%%%%%%%%%%%%%%
%  B. Methods of Solutions  %
%%%%%%%%%%%%%%%%%%%%%%%%%%%%%

\subsubsection{Methods of Solution}

The study of a Differential Equation consists of $3$ phases:
\begin{enumerate}
    \item Formulation of DE from the given physical situation.
    \item Solutions of DE, evaluating the arbitrary constants from the given condition.
    \item Physical interpretation of the solution.
\end{enumerate}

\begin{example}{
    Show that the function $f(x) = e^x+2x^2+6x+7$ is a solution to the DE $\dv[2]{y}{x} - 3\dv{y}{x} + 2y = 4x^2$ 
}

    \[ f(x) = e^x + 2x^2 + 6x + 7 \]
    \[ f'(x) = e^x + 4x + 6 \]
    \[ f"(x) = e^x + 4 \]
    \begin{align*}
        \dv[2]{y}{x} - 3\dv{y}{x} + 2y &= (e^x+4) - 3(e^x+4x+6) + 2(e^x+2x^2+6x+7) \\
        &= 0\cdot e^x + 0\cdot x + (4-18+14) + 4x^2 \\
        &= 4x^2
    \end{align*} \qed
\end{example}

\begin{example}{
    Show that the function $f(x) = \dfrac{1}{1+x^2}$ is a solution to the DE $(1+x^2)\dv[2]{y}{x} + 4\dv{y}{x} + 2y = 0$ 
}
    
    \[ f(x) = \frac{1}{1+x^2} \]
    \[ (1+x^2)f(x) = 1 \]
    \[ (1+x^2)f'(x) + 2xf(x) = 0 \]
    \[ (1+x^2)f''(x) + 2xf'(x) + 2xf'(x) + 2f(x) = 0 \]
    \[
        (1+x^2) \dv[2]{y}{x} + 4x \dv{y}{x} + 2y = 0
    \] \qed
\end{example}

\begin{example}{
    Show that the function $y = (2x^2+2e^{3x}+3)e^{-2x}$ satisfies the DE \[
        \dv{y}{x} + 2y = 6e^x + 4xe^{-2x}
    \]}
    
    \[ y = ( 2x^2 + 2e^{3x} + 3 ) e^{-2x} \]
    \[ y_1 = ( 4x + 6e^{3x} )e^{-2x} - ( 2x^2 + 2e^{3x} + 3 )2e^{-2x} \]
    \[ y_1 = 4xe^{-2x} + 6e^x - 2y \]
    \[
        \dv{y}{x} + 2y = 6e^x + 4e^{-2x}
    \] \qed
\end{example}



%%%%%%%%%%%%%%%%%%%%%%%%%%%%%%%%%%%%%%%%%%%%%%%%%%%%%%%%%%%%%%%%%%%%
%  Initial value, Boundary value problems, Existence of Solutions  %
%%%%%%%%%%%%%%%%%%%%%%%%%%%%%%%%%%%%%%%%%%%%%%%%%%%%%%%%%%%%%%%%%%%%

\subsection{Initial-Value and Boundary-Value Problems, and Existence of Solutions}

%%%%%%%%%%%%%%%%%%%%
%  A. IVP and BVP  %
%%%%%%%%%%%%%%%%%%%%

\subsubsection{Initial-value Problems and Boundary-value Problems}

One of the most frequently encountered type of problems in Differential Equations involves both a DE and one or more supplementary conditions which the solution of the given DE must satisfy. \\~\\

\begin{definition}{IVP and BVP}
    
    Consider the first-order DE \[
        \dv{y}{x} = f(x,y)
    \] where $f$ is a continuous function of $x$ and $y$ in some domain $D$ of the $xy$ plane; and let $(x_0,y_0)$ be a point of $D$. The \textbf{initial-value problem} associated with the DE is to find a solution $\phi$ of the DE, defined on some real interval containing $x_0$, and satisfying the initial condition \[
        \phi(x_0) = y_0
    \] In the customary abbreviated notation, this initial-value problem may be written \[
        \dv{y}{x} = f(x,y)
    \] \[
        y(x_0) = y_0
    \] \\~\\

    If the conditions relate to two different $x$ values (the extreme or boundary values), the proble is called a \textbf{Two-Point Boundary-Value Problem} or simply a \textbf{Boundary-Value Problem (BVP)}.
\end{definition}

\begin{example}{Find the solution of the DE $\dv{y}{x} = 2x$ such that $\forall x\in I, f'(x) = 2x$ and  $f(1)=4$}
    
    \[ \dv{y}{x} = 2x \]
    \[ \int {\dv{y}{x}} \: d{x} = \int {2x} \: d{x} \]
    \[ y = x^2 + c \]
    Substituting $y=4$ and $x=1$,
    \[ 4 = 1 + c \text{ or } c = 3 \]
    \[ \therefore \text{ Solution: } y^2 = x + 3 \] \qed
\end{example}

\begin{example}{$\dv{y}{x} = -\dfrac{x}{y}$, $y(3)=4$}
    
    \[ x + y \dv{y}{x} = 0 \]
    \[ \int{x} \: d{x} + \int{y \dv{y}{x}} \: d{x} = 0 \]
    \[ \frac{x^2}{2} + \frac{y^2}{2} = c' \]
    \[ x^2 + y^2 = c \]

    Substituting $x=3$ and $y=4$,
    \[ 16^2 + 3^2 = c \text{ or } c = 25 \]
    \[ \therefore \text{ Solution: } x^2 + y^2 - 25 = 0 \]
\end{example}

%%%%%%%%%%%%%%%%%%%%%%%%%%%%%%%
%  B. Existence of Solutions  %
%%%%%%%%%%%%%%%%%%%%%%%%%%%%%%%

\subsubsection{Existence of Solutions}

Not all initial-value and boundary-value problems have solutions. For example,\[
    \dv[2]{y}{x} + y = 0
\] \[ y(0) = 1 \;\;,\;\; y(\pi) = 5 \]
has no solutions! Thus arises the question of \textit{existence} of solutions. We can say, every initial-value problem that satisfies definition (1.3.1) has \textit{at least one} solution. However, there arises another question. Can a problem have more than one solution? \\
Let's consider the initial-value problem
\[ \dv{y}{x} = y^{1/3} \;\;;\;\;y(0) = 0 \]
One may verify that the functions $f_1$ and $f_2$ defined, respectively, by \[
    \forall x\in \mathbb{R} ,\;\; f_1(x) = 0
\] and \[
    f_2(x) = (\frac{2}{3}x)^{3/2}, \;\; x \ge 0; \;\; f_2(x) = 0, \;\; x \le 0
\] are both solutions of this initial-value problem. In fact, this problem has infinitely many solutions. Hence, we can state that the initial-value problem need not have a \textit{unique} solution. In order to ensure uniqeness, some additional requiremnt must certainly be imposed.

\begin{theorem}{Basic Existence and Uniqueness Theorem}
    
    \\\textbf{Hypothesis: }
    Consider the differential equation
    \begin{equation}
        \dv{y}{x} = f(x,y)
    \end{equation}
    where
    \begin{itemize}
        \item The function $f$ is a continuous function of $x$ and $y$ in some domain $D$ of the xy plane, and
        \item The partial derivative $\pdv{f}{y}$ is also a continuous function of $x$ and $y$ in $D$; and let $(x_0,y_0)$ be a point in $D$.
    \end{itemize}

    \textbf{Conclusion: }
    There exists a unique solution $\phi$ of the differential equation (1.3.1), defined on some interval $|x-x_0| \le h$, where $h$ is sufficiently small, that satisfies the condition
    \[
        \phi(x_0) = y_0
    \]
\end{theorem}

\begin{example}{
        Show that \[
            y = 4e^{2x} + 2e^{-3x}
        \] is a solution of the initial-value problem \[
        \dv[2]{y}{x} + \dv{y}{x} - 6y = 0
        \] \[ y(0) = 6 \]
        \[ y'(0) = 2 \]
        Is $y=2e^{2x} + 4e^{-3x}$ also a solution of this problem? Explain why or why not.
    }
    
    \[ y = 4e^{2x} + 2e^{-3x} \]
    \[ y_1 = 8e^{2x} - 6e^{-3x} \]
    \[ y_2 = 16e^{2x} + 18e^{-3x} \]
    \begin{align*}
        y_2 + y_1 - 6y &= (16e^{2x} + 18e^{-3x}) + (8e^{2x} - 6e^{-3x}) - 6(4e^{2x} + 2e^{-3x}) \\
        &= 0\cdot e^{2x} + 0\cdot e^{-3x} \\
        &= 0
    \end{align*}
    The solution also satisfies $y(0)=6$ and  $y'(0)=2$ \\~\\

    Now,
    for $y = 2e^{2x} + 4e^{-3x}$,
    \[ y_1 = 4e^{2x} - 12e^{-3x} \;\;;\;\; y_2 = 8e^{2x} + 36e^{-3x} \]
    \begin{align*}
        y_2 + y_1 - 6y &= (8e^{2x} + 36e^{-3x}) + (4e^{2x} - 12e^{-3x}) - 6(2e^{2x} + 4e^{-3x}) \\
        &= 0\cdot e^{2x} + 0\cdot e^{-3x} \\
        &= 0
    \end{align*}
    However, in this case,
    \[ y(0) = 6 \;\;;\;\; y'(0) = -8 \]
    As we can see, this solution doesn't satisfy the initial-value problem. Hence $y=2e^{2x} + 4e^{-3x}$ is not a solution of this problem.
\end{example}

\begin{example}{
        Given that every solution of \[
            x^3 \dv[3]{y}{x} - 3x^2 \dv[2]{y}{x} + 6x \dv{y}{x} - 6y = 0
        \] may be written in the form $y = c_1x + c_2x^2 + c_3x^3$ for some choice of the arbitrary constants $c_1$, $c_2$, and $c_3$, solve the initial-value problem consisting of the above DE plus the three conditions \\
        \[ y(2)=0 \;\;,\;\; y'(2)=2 \;\;,\;\; y''(2)=6 \]
    }
    
    \[ y = c_1x + c_2x^2 + c_3x^3 \]
    \begin{equation}
        y(2) = 0 \text{ or, } 8c_3 + 4c_2 + 2c_1 = 0
    \end{equation}
    \[ y' = c_1 + 2c_2x + 3c_3x^2 \]
    \begin{equation}
        y'(2) = 2 \text{ or, } 12c_3 + 4c_2 + c_1 = 2
    \end{equation}
    \[ y'' = 0 + 2c_2 + 6c_3x \]
    \begin{equation}
        y''(2) = 6 \text{ or, } 12c_3 + 2c_2 + 0c_1 = 6
    \end{equation}
    Solving (1.3.1), (1.3.2), and (1.3.3) we get,
    \[ c_1 = 2 \;\;,\;\; c_2 = -3 \;\;,\;\; c_3 = 1 \]
    \[ \therefore \text{ Solution: } y = 2x - 3x^2 + x^3 \]
\end{example}
