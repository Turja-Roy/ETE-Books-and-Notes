%%%%%%%%%%%%%%%%%%%%%%%%%%%%%%%%%%%%%%%%%%%%%%%%%%%%%%%%%%
%  1st ord eqn for which Exact Solutions are Obtainable  %
%%%%%%%%%%%%%%%%%%%%%%%%%%%%%%%%%%%%%%%%%%%%%%%%%%%%%%%%%%

\section{First Order Equations for Which Exact Solutions Are Obtainable}

%%%%%%%%%%%%%%%%%%%%%
%  Exact DE and IF  %
%%%%%%%%%%%%%%%%%%%%%

\subsection{Exact Differential Equations and Integrating Factors}

%%%%%%%%%%%%%%%%%%%%%%%%%%%%%%%%%%%%%%%%%
%  A. Standard Forms of First-Order DE  %
%%%%%%%%%%%%%%%%%%%%%%%%%%%%%%%%%%%%%%%%%

\subsubsection{Standard Forms of First-Order Differential Equations}
The first-order differential equations may be expressed in either the \textbf{Derivative Form}
\begin{equation}
    \dv{y}{x} = f(x,y)
\end{equation}
or the \textbf{Differential Form}
\begin{equation}
    M(x,y) \: d{x} + N(x,y) \: d{y} = 0
\end{equation}

\begin{example}{Standard Forms}
    
    The equation \[
        \dv{y}{x} = \frac{x^2 + y^2}{x-y}
    \]
    is the form (2.1.1). It may be written as \[
        (x^2 + y^2) \: dx + (y-x) \: dy = 0
    \] which is of the form (2.1.2). \\
    Again, the equation \[
        (\sin x + y) \: dx + (x + 3y) \: dy = 0
    \] is of the form (2.1.2), which can also be written as \[
    \dv{y}{x} = - \frac{\sin x + y}{x + 3y}
    \]
\end{example}

%%%%%%%%%%%%%%%%%%%%%%%%%%%%%%%%%%%%%
%  B. Exact Differential Equations  %
%%%%%%%%%%%%%%%%%%%%%%%%%%%%%%%%%%%%%

\subsubsection{Exact Differential Equations}

\begin{definition}{Exact Differential}
    
    Let $F$ be a function of two real variables such that  $F$ has continuous first partial derivatives in a domain $D$. The total differential $dF$ of the function $F$ is defined by the formula
    \[
        d{F(x,y)} = \pdv{F(x,y)}{x} \: dx + \pdv{F(x,y)}{y} \: dy
    \]
    for all $(x,y) \in D$.
\end{definition} \\~\\

Comparing $dF(x,y)$ with the form (2.1.2), we get
\[
    \pdv{F(x,y)}{x} = M(x,y) \;\;\; and \;\;\;
    \pdv{F(x,y)}{y} = N(x,y)
\]

\begin{example}{}
    
    Let $F$ be a function \[
        F(x,y) = xy^2 + 2x^3y
    \] for all real $(x,y)$. Then
    \[
        \pdv{F(x,y)}{x} = y^2 + 6x^2y ,\;\;\;
        \pdv{F(x,y)}{y} = 2xy + 2x^3
    \]
    and the total differential $dF$ is defined by \[
        dF(x,y) = (y^2 + 6x^2y) \: dx + (2xy + 2x^3) \: dy
    \]
    for all real $(x,y)$
\end{example}

\begin{definition}{Exact Differential Equation}
    
    The expression
    \begin{equation}
        M(x,y) \: d{x} + N(x,y) \: d{y}
    \end{equation}
    is called an exact differential in a domain $D$ if there exists a function $F$ of two real variables such that this expression equals the total differential $dF(x,y)$ for all $(x,y) \in D$. \\
    That is, expression (2.1.3) is an exact differential in $D$ if there exists a function $F$ such that
\[
    \pdv{F(x,y)}{x} = M(x,y) \;\;\; and \;\;\;
    \pdv{F(x,y)}{y} = N(x,y)
\]
for all $(x,y) \in D$. \\
If $M(x,y) \: d{x} + N(x,y) \: d{y}$ is an exact differential, then the differential equation
\[ M(x,y) \: d{x} + N(x,y) \: d{y} = 0 \]
is called an \textbf{Exact Differential Equation}.
\end{definition}

\begin{theorem}{Exact Differential Equation}
    
    \begin{enumerate}
        \item If the DE $M(x,y) \: d{x} + N(x,y) \: d{y} = 0$ is exact in $D$, then
            \[ \forall (x,y)\in D, \;\;\; \pdv{M(x,y)}{y} = \pdv{N(x,y)}{x} \]
        \item Conversely, if 
            \[ \forall (x,y)\in D, \;\;\; \pdv{M(x,y)}{y} = \pdv{N(x,y)}{x} \]
        then the DE is exact in $D$.
    \end{enumerate}
\end{theorem} \\~\\

\underline{\textbf{Proof (1):}}
\begin{align*}
    \pdv{F(x,y)}{x} = M(x,y) \;\;\;&,\;\;\; \pdv{F(x,y)}{y} = N(x,y)  \\
    \frac{\partial^2 F(x,y)}{\partial x \partial y} = \pdv{M(x,y)}{y} \;\;\;&,\;\;\; \frac{\partial^2 F(x,y)}{\partial x \partial y} = \pdv{N(x,y)}{x}
\end{align*}
\[
    \because \frac{\partial^2 F(x,y)}{\partial y \partial x} = \frac{\partial^2 F(x,y)}{\partial x \partial y}
\] \[
    \therefore \pdv{M(x,y)}{y} = \pdv{N(x,y)}{x} \qed
\]

%%%%%%%%%%%%%%%%%%%%%%%%%%
%  Solution of Exact DE  %
%%%%%%%%%%%%%%%%%%%%%%%%%%
 
\subsubsection{The Solution of Exact Differential Equations}

\begin{theorem}{Solution of Exact DE}
    
    \\If $M(x,y) \: d{x} + N(x,y) \: d{y} = 0$ is exact in domain $D$, then
    \[
        \forall(x,y)\in D, \exists F(x,y) : \pdv{F(x,y)}{x} = M(x,y) \;\;\;and\;\;\;
        \pdv{F(x,y)}{y} = N(x,y)
    \]
    Then the equation may be written \[
        \pdv{F(x,y)}{x} \: d{x} + \pdv{F(x,y)}{y} \: d{y} = 0
    \] or simply, \[
        dF(x,y) = 0
    \]
    Here, $F(x,y) = c$ is a one-parameter family of solutions of this DE, where $c$ is an arbitrary constant.
\end{theorem}

\begin{example}{Solve the equation \[
        (3x^2 + 4xy) \: d{x} + (2x^2 + 2y) \: d{y} = 0
\]}
    
    \textbf{Standard Method:}
    \begin{align*}
        \pdv{F(x,y)}{x} &= M(x,y) = 3x^2 + 4xy \\
        F(x,y) &= \int (3x^2 + 4xy) \: \partial{x} + \phi(y) \\
        &= x^3 + 2x^2y + \phi(y) \\
    \end{align*}
    Again, \[
        \pdv{F(x,y)}{y} = 2x^2 + \pdv{\phi(y)}{y} = 2x^2 + 2y
    \] \[
        \dv{\phi(y)}{y} = 2y
    \] \[
    \int \dv{\phi(y)}{y} \: d{y} = \int 2y \: d{y}
    \] \[
        \phi(y) = y^2 + c_0
    \]
    Thus, we get \[
        F(x,y) = x^3 + 2x^2y + y^2 + c_0
    \] Hence, a one-parameter family of the solution is $F(x,y) = c_1$ or, \[
        x^3 + 2x^2y + y^2 + c_0 = c_1
    \]\[
        \boxed{x^3 + 2x^2y + y^2 = c}
    \]
    \textbf{Method of Grouping:}
    \begin{align*}
        (3x^2 + 4xy) \: d{x} + (2x^2 + 2y) \: d{y} &= 0 \\
        3x^2 \: d{x} + (4xy \: d{x} + 2x^2 \: d{y}) + 2y \: d{y} &= 0 \\
        d(x^3) + d(2x^2y) + d(y^2) &= d(c) \\
        \boxed{x^3 + 2x^2y + y^2 = c}
    \end{align*}
\end{example}

\begin{example}{Solve the initial-value problem \[
        (2x\cos y + 3x^2y) \: d{x} + (x^3 - x^2\sin y - y) \: d{y} = 0 \;\;;\;\; y(0) = 2
\]}

    \begin{align*}
        (2x\cos y \: dx - x^2\sin y \: d{y}) + (3x^2y \: d{x} + x^3 \: d{y}) - y \: d{y} &= 0 \\
        d(x^2 \cos y) + d(x^3y) + d(\frac{y^2}{2}) &= d(c_1) \\
        \boxed{ 2x^2 \cos y + x^3y + y^2 = c }
    \end{align*}
\end{example}











